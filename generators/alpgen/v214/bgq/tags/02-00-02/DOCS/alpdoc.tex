 \documentclass[paper]{JHEP3}
\usepackage{epsfig}
%\usepackage{axodraw}
\jot 7pt
% Roberto macros
\newcommand{\qu}{u}
\newcommand{\qd}{d}
\newcommand{\qubar}{\bar u}
\newcommand{\qdbar}{\bar d}
\newcommand{\bfig}{\begin{center}\begin{picture}}
\newcommand{\efig}[1]{\end{picture}\\{\small #1}\end{center}}
\newcommand{\flin}[2]{\ArrowLine(#1)(#2)}
\newcommand{\wlin}[2]{\DashLine(#1)(#2){2.5}}
\newcommand{\zlin}[2]{\DashLine(#1)(#2){5}}
\newcommand{\glin}[3]{\Photon(#1)(#2){2}{#3}}
\newcommand{\lin}[2]{\Line(#1)(#2)}
\newcommand{\noi}{\noindent}
\newcommand{\sof}{\SetOffset}
\newcommand{\bmip}[2]{\begin{minipage}[t]{#1pt}\bfig(#1,#2)}
\newcommand{\emip}[1]{\efig{#1}\end{minipage}}
\newcommand{\putk}[2]{\Text(#1)[r]{$p_{#2}$}}
\newcommand{\putp}[2]{\Text(#1)[l]{$p_{#2}$}}
\newcommand{\bq}{\begin{equation}}
\newcommand{\eq}{\end{equation}}
\newcommand{\bqa}{\begin{eqnarray}}
\newcommand{\eqa}{\end{eqnarray}}
\newcommand{\nl}{\nonumber \\}
\newcommand{\eqn}[1]{eq. (\ref{#1})}
\newcommand{\eqs}[1]{eqs. (\ref{#1})}
\newcommand{\ibidem}{{\it ibidem\/},}
\newcommand{\vpb}{}
\newcommand{\p}[1]{{\scriptstyle{\,(#1)}}}
\newcommand{\gev}{\mbox{GeV}}
\newcommand{\tev}{\mbox{TeV}}
\newcommand{\mev}{\mbox{MeV}}
\setcounter{section}{0}
\setcounter{subsection}{0}
\setcounter{paragraph}{0}
\setcounter{subparagraph}{0}
\setcounter{equation}{0}
\setcounter{figure}{-1}
\setcounter{table}{0}
\setcounter{footnote}{0}
\setcounter{mpfootnote}{0}
% MLM's macros
\def    \eqnum          #1{(\ref{#1})}       %equation # in round parenthesis
\def    \scite          #1{$^{\cite{#1}}$}     %superscript biblio ref
%--------------------------------------------
%       SET PAGE SIZE
%       \evensidemargin 0.0in
        \oddsidemargin -1cm
        \textwidth 16.5cm
        \textheight 22cm
        \hoffset=0cm
        \headsep -0.5in
        \newdimen\eqskip
        \newdimen\txtskip
        \eqskip=25pt
        \txtskip=25pt
        \baselineskip=\txtskip
        \parskip 5pt plus 1pt
        \floatsep 0cm
        \textfloatsep 0.2cm
        \renewcommand{\textfraction}{0.3}
        \renewcommand{\topfraction}{0.7}
        \newdimen\mysep                
        \newdimen\hmysep
        \mysep=-0.4cm
        \hmysep=-0.4cm
%        \mysep=0cm
%\begin{document}
       
  \newcommand{\ccaption}[2]{
    \begin{center}
    \parbox{0.85\textwidth}{
      \caption[#1]{\small{{#2}}}
      }
    \end{center}
    }
\newcommand{\BS}{\bigskip}
% MATH SYMBOLS
\def    \be             {\begin{equation}}
\def    \ee             {\end{equation}}
\def    \ba             {\begin{eqnarray}}
\def    \ea             {\end{eqnarray}}
\def    \nn             {\nonumber}
\def    \=              {\;=\;}
\def    \frac           #1#2{{#1 \over #2}}
\def    \ret            {\\[\eqskip]}
\def    \ie             {{\em i.e.\/} }
\def    \eg             {{\em e.g.\/} }
%\def \lsim{\mathrel{\vcenter
%     {\hbox{$<$}\nointerlineskip\hbox{$\sim$}}}}
%\def \gtrsim{\mathrel{\vcenter
%     {\hbox{$>$}\nointerlineskip\hbox{$\sim$}}}}
\def    \bentarrow      {\:\raisebox{1.1ex}{\rlap{$\vert$}}\!\rightarrow}
\def    \rd             {{\mathrm d}}    
\def    \Im             {{\mathrm{Im}}}  
\def    \bra#1          {\mbox{$\langle #1 |$}}
\def    \ket#1          {\mbox{$| #1 \rangle$}}

% UNITS                 
\def    \kev            {\mbox{$\mathrm{keV}$}}
\def    \mev            {\mbox{$\mathrm{MeV}$}}
\def    \gev            {\mbox{$\mathrm{GeV}$}}

% PARTICLE SYMBOLS
\def    \nubar   {\bar{\nu}}
\def    \ubar   {\bar{u}}
\def    \dbar   {\bar{d}}
\def    \qbar   {\bar{q}}
\def    \bbar   {\bar{b}}
\def    \cbar   {\bar{c}}
\def    \cpbar   {\bar{c}'}
\def    \tbar   {\bar{t}}
\def    \Qbar   {\overline{Q}}

% KINEMATICAL VARIABLES 
\def    \mq             {\mbox{$m_Q$}}  
\def    \mZ             {\mbox{$m_Z$} }
\def    \mZsq             {\mbox{$m_Z^2$} }
\def    \mW             {\mbox{$m_W$} }
\def    \mWsq             {\mbox{$m_W^2$} }
\def    \mH             {\mbox{$m_H$} }
\def    \mHsq             {\mbox{$m_H^2$} }
\def    \mt             {\mbox{$m_t$}}  
\def    \mtsq             {\mbox{$m_t^2$}}  
\def    \mb             {\mbox{$m_b$}}  
\def    \mbsq             {\mbox{$m_b^2$}}  
\def    \mqq            {\mbox{$m_{Q\bar Q}$}}
\def    \mqqsq          {\mbox{$m^2_{Q\bar Q}$}}
\def    \pt             {\mbox{$p_T$}}
\def    \et             {\mbox{$E_T$}}
\def    \xt             {\mbox{$x_T$}}
\def    \xtsq           {\mbox{$x_T^2$}}
\def    \ptsq           {\mbox{$p^2_T$}}
\def    \ptbsq           {\mbox{$p^2_{T,b}$}}
\def    \ptbbsq           {\mbox{$p^2_{T,\bar{b}}$}}
\def    \ptWsq           {\mbox{$p^2_{T,W}$}}
\def    \ptZsq           {\mbox{$p^2_{T,Z}$}}
\def    \pttsq           {\mbox{$p^2_{T, t}$}}
\def    \pttbsq           {\mbox{$p^2_{T, \bar{t}}$}}
\def    \mT             {\mbox{$m_T$}}
\def    \mTsq           {\mbox{$m^2_T$}}
\def    \etsq           {\mbox{$E^2_T$}}
        
% QCD PARAMETERS                                      
\newcommand     \MSB            {\ifmmode {\overline{\rm MS}} \else 
                                 $\overline{\rm MS}$  \fi}
\def    \muf            {\mbox{$\mu_{\rm F}$}}
\def    \mug            {\mbox{$\mu_\gamma$}}
\def    \mufsq          {\mbox{$\mu^2_{\rm F}$}}
\def    \mur            {{\mbox{$\mu_{\rm R}$}}}
\def    \mursq          {\mbox{$\mu^2_{\rm R}$}}
\def    \mul            {{\mu_\Lambda}}
\def    \mulsq          {\mbox{$\mu^2_\Lambda$}}

\def    \bzero          {\mbox{$b_0$}}
\def    \as             {\ifmmode \alpha_s \else $\alpha_s$ \fi}
\def    \aem             {\mbox{$\alpha_{em}(\mZ)$}}
\def    \asb            {\mbox{$\alpha_s^{(b)}$}}
\def    \assq           {\mbox{$\alpha_s^2$}}
\def \oacube {\mbox{$ {\cal O}(\alpha_s^3)$}}
\def \oaemacube {\mbox{$ {\cal O}(\alpha\alpha_s^3)$}}
\def \oafour {\mbox{$ {\cal O} (\alpha_s^4)$}}
\def \oatwo {\mbox{$ {\cal O} (\alpha_s^2)$}}
\def \oaematwo {\mbox{$ {\cal O}(\alpha \alpha_s^2)$}}
\def \oaemas {\mbox{$ {\cal O}(\alpha \alpha_s)$}} 
\def \oas   {\mbox{$ {\cal O}(\alpha_s)$}}
\def\slash#1{{#1\!\!\!/}}
\def\rt1{\raisebox{-1ex}{\rlap{$\; \rho \to 1 \;\;$}}
\raisebox{.4ex}{$\;\; \;\;\simeq \;\;\;\;$}}
\def\ltap{\raisebox{-.5ex}{\rlap{$\,\sim\,$}} \raisebox{.5ex}{$\,<\,$}}
\def\gtap{\raisebox{-.5ex}{\rlap{$\,\sim\,$}} \raisebox{.5ex}{$\,>\,$}} 

\newcommand\LambdaQCD{\Lambda_{\scriptscriptstyle \rm QCD}}

\def\naive{na\"{\i}ve}
\def\asp{{\alpha_s}\over{\pi}}
\def\half{\frac{1}{2}}
\def\herwig{{\small HERWIG}}
\def\isajet{{\small ISAJET}}
\def\pythia{{\small PYTHIA}}
\def\grace{{\small GRACE}}
\def\amegic{{\small AMEGIC++}}
\def\vecbos{{\small VECBOS}}
\def\madgraph{{\small MADGRAPH}}
\def\comphep{{\small CompHEP}}
\def\ALPHA{{\small ALPHA}}
\def\ppbar{\mbox{$p \bar{p}$}}
\def\met{$\rlap{\kern.2em/}E_T$}
%%%%%%%%%%%%%%%%%%%%%%%%%%%%%%%%%%%%%%%%%%%%%%%%%%%%%%%%%%%%%%%%%%%%%%

\title{ 
      ALPGEN, a generator for 
      hard multiparton processes 
      in hadronic collisions \thanks{The work of MLM and FP is
        supported in part by the EU Fourth Framework Programme
        ``Training and Mobility of Researchers'', Network ``Quantum
        Chromodynamics and the Deep Structure of Elementary
        Particles'', contract FMRX--CT98--0194 (DG 12 -- MIHT). MM and
        RP acknowledge the financial support of the European Union
        under contract HPRN-CT-2000-00149. ADP is supported by a
        M.Curie fellowship, contract HPMF-CT-2001-01178.}}
\vfill                                                       
\author{
  Michelangelo L. MANGANO, Fulvio PICCININI\thanks{On leave 
          of absence from INFN Sezione di Pavia, Italy.},
and Antonio D. POLOSA\\
CERN, Theoretical Physics Division, CH~1211 Geneva 23, Switzerland
\\ E-mail: \email{michelangelo.mangano@cern.ch, fulvio.piccinini@cern.ch,
 antonio.polosa@cern.ch}}
\author{Mauro MORETTI\\
Dipartimento di Fisica, Universit\`{a} di 
Ferrara, and INFN, Ferrara, Italy \\
E-mail: \email{mauro.moretti@fe.infn.it}}
\author{Roberto PITTAU \\ 
Dipartimento di Fisica, 
Universit\`{a} di Torino, and INFN, Torino, Italy \\
E-mail: \email{roberto.pittau@to.infn.it}}
\vskip -0.5truecm

\abstract{ This paper presents a new event generator, ALPGEN,
  dedicated to the study of multiparton hard processes in hadronic
  collisions. The code performs, at the leading order in QCD and EW
  interactions, the calculation of the exact matrix elements for a
  large set of parton-level processes of interest in the study of the
  Tevatron and LHC data.  The current version of the code describes
  the following final states: $(W\to f\bar{f}') Q\Qbar+ N$~jets ($Q$
  being a heavy quark, and $f=\ell,q$), with $N\le 4$;
  $(Z/\gamma^{*}\to f\bar{f}) \, Q\Qbar+N$ ~jets ($f=\ell,\nu$), with
  $N\le 4$; $(W\to f\bar{f}') + \mbox{charm} + N$~jets ($f=\ell,q$,
  $N\le 5$); $(W\to f\bar{f}') + N$~jets ($f=\ell,q$) and
  $(Z/\gamma^{*}\to f\bar{f})+ N$~jets ($f=\ell,\nu$), with $N\le 6$;
  $nW+mZ+j\gamma+lH+N$~jets, with $n+m+j+l+N\le 8$, $N\le3$, including all
  2-fermion decay modes of $W$ and $Z$ bosons, with spin correlations;
  $Q\Qbar+N$~jets, with $t\to b f\bar{f}'$ decays and relative spin
  correlations included where relevant, and $N\le 6$; $Q\Qbar
  Q'\Qbar'+N$~jets, with $Q$ and $Q'$ heavy quarks (possibly equal)
  and $N\le 4$; $H Q \Qbar+N$~jets, with $t\to b f\bar{f}'$ decays and
  relative spin correlations included where relevant and $N\le 4$; 
  $m H + N$~jets, with $1 \leq m \leq 4$ and $m + N \leq 8$; 
  $(W\to f\bar{f}') + m \gamma + N$~jets ($f=\ell,q$), with 
  $0 \leq m \leq 2$; $(W\to f\bar{f}') Q\Qbar + m \gamma + N$~jets ($Q$
  being a heavy quark, $0 \leq m \leq 2$ and $f=\ell,q$), with $N\le 3$; 
  $N$~jets, with $N\le 6$; $N\gamma+M$ jets, with $N\ge 1$, $N+M\le 8$
  and $M\le 6$; 
  $Q\Qbar + m \gamma +N$~jets, with $t\to b f\bar{f}'$ decays and 
  relative spin correlations included where relevant, and $m + N \le 6$;
  single-top, namely: 1) $t(\bar t)+N$~jets, with $N\le 2$, 
  2) $t(\bar t) {\bar b}(b)+N$~jets, with $N\le 1$, 3) 
  $t(\bar t)+W+N$~jets, with $N\le 1$,  4) $t(\bar t) {\bar b}(b)+W+N$~jets, 
  with $N\le 1$. All four channels include $t\to b f\bar{f}'$, 
  $W\to f\bar{f}'$ decays and relative spin correlations where relevant. 
  Parton-level events are generated, providing full information on
  their colour and flavour structure, enabling the evolution of the
  partons into fully hadronised final states.
  New features specific to V2.0 are given in an appendix.}

\preprint{CERN-TH/2002-129, FTN/T-2002/06\\ hep-ph/0206293 \\Version update:
  V2.1, December 2006}
%\keywords{Hadronic Collisions, Jets, Monte Carlo}

\begin{document}
\section{Introduction}
\label{sec:intro}
The large energies available in current and forthcoming hadronic
colliders make final states with several hard and well separated jets
a rather common phenomenon.  These multijet final states can originate
directly from hard QCD radiative processes~\cite{Mangano:1991by}, or
from the decay of massive particles, such as for example $W$ and $Z$
gauge bosons, top quarks, Higgs bosons, supersymmetric particles, etc.
Whether our interest is in accurate measurements of top quarks or in
the search for more exotic states~\cite{Gianotti:2002xx}, multijet
final states always provide an important observable, and the study of
the backgrounds due to QCD is an essential part of any experimental
analysis.

Several examples of calculation of multijet cross-sections in hadronic
collisions exist in the literature. Some of them are included in
parton-level Monte Carlo (MC) event generators, where final states
consisting of hard and well isolated partons are generated. Among the
most used and best documented examples are
PAPAGENO~\cite{Hinchliffe:1993de} (a compilation of several partonic
processes), VECBOS~\cite{Berends:1991ax} (for production of $W/Z$
bosons in association with up to 4 jets), NJETS~\cite{Berends:1989ie}
(for production of up to 6 jets). The range of jet multiplicities
calculable in practice for purely QCD processes in hadronic collisions
has recently been extended in~\cite{Draggiotis:2002hm}, where new
techniques~\cite{Draggiotis:2000sh} to deal with the complexity of the
multijet flavour configurations have been implemented in the
calculation of up to 7 jets.  Finally, programs for the automatic
generation of user-specified parton-level processes exist and have
been used in the literature for the calculation of many important
reactions in hadronic collisions: \madgraph~\cite{Stelzer:1994ta},
\comphep~\cite{Pukhov:1999gg}, \grace~\cite{Ishikawa:1993qr} and
\amegic~\cite{Krauss:2001iv}.

In order to use these results in practical analyses of the
experimental data, the calculations need to be completed with the
treatment of the higher-order corrections leading to the development
of partonic cascades, and with the subsequent transformation of the
partons into observable hadrons.  MC programs such as
\herwig~\cite{Marchesini:1988cf}, \pythia~\cite{Sjostrand:1994yb} or
\isajet~\cite{Paige:1998xm} are available to carry out these last two
steps. The consistent combination of the parton-level calculations
with the partonic evolution given by the shower MC programs is the
subject of extensive work.  Several approaches to this problem have
been proposed in the case of low-order processes, where one is
interested in final states with at most one extra jet relative to a
given Born-level configuration (for example $W,Z$ plus
jet~\cite{Seymour:1995df} and $t\bar{t}$ +
jet~\cite{Corcella:1998rs}).  In these cases, the proposed algorithms
correct the probability for hard-emission estimated by the
approximation of the shower-evolution programs, using the value of the
exact real-emission higher-order matrix element. In other
studies~\cite{Dobbs:2001gb,Grace:2003npb,Frixione:2002ik}, limited 
for the time
being to the case of jet emission in association with vector boson and
heavy quark~\cite{Frixione:2003ei} pairs, the full set of virtual and
real NLO corrections to the partonic matrix elements has been merged
with the \herwig\ MC. See in particular~\cite{Frixione:2002ik} for a
complete discussion of the problem of matching NLO parton level and
leading-logarithmic (LL) shower generators.  Some of the problems
raised by a consistent matching of NLO parton level calculations and
next-to-leading logarithm (NLL) shower evolution are discussed
in~\cite{Collins:2000qd}.

In the case of large jet multiplicities, the complexity of the matrix
element evaluation and of its singularity structure prevents so far
the application of these approaches. 
Recently, a new strategy has been introduced~\cite{Catani:2001cc}
 for the merging of
multijet matrix elements with the shower development. This involves a
reweighting of the matrix element weights with Sudakov form factors,
and the veto of shower emissions in regions of phase-space already
covered by the parton-level configurations. After the shower
evolution, samples of different
parton-level multiplicity can then be combined together to obtain
inclusive samples of arbitrary jet multiplicity, double counting being
limited to subleading effects. A complete application of this ideas
has been achieved for inclusive hadronic final states in $e^+e^-$
collisions, and a proposal has been formulated for the extension to
the hadron collider case~\cite{Krauss:2002up}. Explicit
implementations are being developed~\cite{kraussetal}.
 
 We discussed in~\cite{Caravaglios:1999yr} a theoretical framework for
the evaluation and generation of events in such a way as to enable the
subsequent perturbative evolution using a shower MC program.
In~\cite{Mangano:2001xp} we presented a complete application to the
case where a $W$ boson is produced in association with a heavy quark
pair, plus up to 4 additional light partons. The relative MC code
allows a complete description of these final states, from the leading
order (LO) matrix element computation to the perturbative evolution
and hadronization carried out using \herwig.

We recently extended the application of the ideas contained
in~\cite{Caravaglios:1999yr}, completing a library of MC codes for
hadronic collisions including the following new processes: $WQ\Qbar+
N$~jets and $Z/\gamma^{*} \, Q\Qbar+N$ ~jets ($Q$ being a heavy
quark), with $N\le 4$; $W+\mbox{charm}+ N$~jets; 
$W+ N$~jets and $Z+ N$~jets ($N\le 6$);
$nW+mZ+lH+N$~jets, with $n+m+l+N\le 8$ and $N\le3$; $Q\Qbar+N$~jets
($N\le 6$); $Q\Qbar Q'\Qbar'+N$~jets, with $Q$ and $Q'$ heavy quarks
(possibly equal) and $N\le 4$; $H Q \Qbar+N$~jets ($N\le 4$);
$N$~jets, with $N\le 6$; $N\gamma+M$ jets, with $N\ge 1$, $N+M\le 8$ 
and $M\le 6$; single-top, namely: 1) $t(\bar t)+N$~jets, with $N\le 2$, 
2) $t(\bar t) {\bar b}(b)+N$~jets, with $N\le 1$, 3) 
$t(\bar t)+W+N$~jets, with $N\le 1$,  4) $t(\bar t) {\bar b}(b)+W+N$~jets, 
with $N\le 1$. 
Details on the decay mode options for the
various unstable particles will be  given below. For all of these
processes, an interface to both \herwig\ and \pythia\ is provided. In the
present work, we document the contents and use of this library. The
emphasis is on the code itself and not on the physics approach, which
is  discussed in more detail in~\cite{Mangano:2001xp}. Numerical results for
some benchmark processes are nevertheless presented.

Independent work on the merging of parton-level calculations with
shower MC's (for the specific case of \pythia) has been pursued by the
\comphep\ group~\cite{Belyaev:2000wn}, by the \grace\ 
group~\cite{Sato:2001ae} and, more recently, 
in~\cite{Kersevan:2002dd,Tsuno:2002ae,Maltoni:2002qb}. 
To the best of our knowledge, a large fraction of the matrix element
calculations documented in this paper have never been performed before.
%All the above approaches are limited to final states with at most 
%4 partons. 

Section~\ref{sec:general} reviews the general structure of the codes,
covering both the aspects of the  parton-level calculations, and
of the shower evolution.  Section~\ref{sec:hard} discusses the
features of the implementation of each individual hard process in our
library.  A technical Appendix will provide more explicit details on
the programs and their use.

The library containing all codes described in this work can be
downloaded from the following URL: {\tt
  http://mlm.home.cern.ch/mlm/alpgen} \,.
The users can find here the latest updates of the code, as well as a
detailed history of the code changes.

\section{The general structure of the program}
\label{sec:general}
The program consists of several building blocks (see the Appendix for
details). A section of the code library defines the overall
infrastructure of the generator, implementing the logical sequence of
operations in a standard set of subroutine calls; this part is
independent of the hard process selected and, among other things, it
includes the algorithms needed for the evaluation of the matrix
elements, for the evaluation of the parton densities and for the
bookkeeping and histogramming of the results.  Each hard process has
viceversa a separate set of code elements, which are specific to it.
These include the process initialization, the phase-space generation,
the extraction of flavour and colour structure of the event, and the
default analysis routines.  Each hard process corresponds to a
specific executable, obtained by linking the relative
process-dependent code elements with the process-independent ones.  In
addition to the above, a section of the code library deals with the
shower evolution. As explained below, the shower evolution is
performed as an independent step, following the generation of a sample
of unweighted parton-level events. The code elements relative to this
phase of the computation include the \herwig\ code and the algorithm
needed to transform the partonic input into a format which can be
interpreted and processed by \herwig. We adopt the formatting standard
proposed in the so called Les Houches accord~\cite{Boos:2001cv}.  

As alluded to above, the program has two main modes of operation. In the
first mode the code performs the parton-level calculation of the matrix
elements relative to the selected hard processes, generating weighted
events.  Each weighted event is analysed on-line in a routine where
the kinematics of the event can be studied, and histograms filled. The
user has direct access to this analysis routine, and can adapt it to
his needs.  At the end of the run, differential distributions are
obtained. A histogramming package is included in the code library,
which generates a graphic output in the form of topdrawer plots.
This mode of running can also be used in order to easily get total
cross sections in presence of some overall generation cuts (e.g. rates
for production of jets above a given threshold), without looking at
any particular differential distribution.

In the second mode of operation the code generates unweighted parton
level events and stores them to a file, for subsequent evolution via
the parton-shower part of the program. The generation of parton-level
events, and their shower evolution, are performed in two different
phases by different code elements. Since the generation of unweighted
parton-level events is typically by far the most CPU intensive
component of the calculation, the storage of unweighted events allows
to build up event data sets which can then be used efficiently for
studies of hadronization systematics or realistic detector
simulations. In this mode of operation the matrix-element calculation
generates all the flavour and colour information necessary for the
complete shower evolution. The kinematical,
flavour and colour data for a given event are stored in a file, and
are read in by the shower MC, which will process the event.
In the rest of this Section we present in more detail these two running
modes of the code.

\subsection{Parton-level generation and cross section evaluation}
\label{sec:xsec}
In a nutshell, the calculation of the cross section for a given hard
process is performed in the following steps:
\begin{itemize}
\item The parameters required to define the hard process are passed to
  the code. These include the selection of jet multiplicity, the
  mass of possible heavy quarks, rapidity and transverse momentum
  cuts, etc.
\item A first phase-space integration cycle is performed, with the goal of
 exploring how the cross-section is distributed in phase-space and
  among the possible contributing  subprocesses. Event by event, the
  following steps take place:
\begin{itemize}
  \item one subprocess (see later) is randomly selected; 
  \item a point in phase-space is randomly selected, consistent with
    the required kinematical acceptance cuts;
  \item the initial-state parton luminosity is evaluated for the
    chosen subprocess, and 
   one among the possible flavour configurations is selected (see later);
  \item
    spin and colour for each parton are randomly assigned;
  \item the matrix element is evaluated, and the weight of the event
    is obtained after inclusion of the phase-space and parton-luminosity
    factors. A bookkeeping of the weights is kept for each individual
   subprocess and phase-space subvolume.
\end{itemize}
\item At the end of the first integration iteration, a map of the
  cross-section distribution among the different subprocesses and in
  phase-space is available. It will be used for subsequent
  integration cycles, where the phase-space and subprocess random sampling
  will be weighted by the respective probability distributions.
\item After the completion of a series of warm-up integration cycles (whose
  number is specified at the beginning of the run by the user), the
  optimised integration grids are stored in a file. A final
  large-statistics run is then performed. After the generation of each
  event, its kinematics is analysed and histograms are filled.
\end{itemize}
We shall now discuss in more detail the individual steps outlined
above and the ingredients of the calculation.

\subsubsection{Selection of the subprocess}
The calculation of the cross section for  multiparton final
states involves typically the sum over a large set of subprocesses
and flavour configurations. For example, in the case of $WQ\Qbar+2$
jets we have, among others, the following subprocesses:
\be q\qbar' \to W Q\Qbar g g, \;
q g \to W Q\Qbar g  q' , \;
g q \to  W Q\Qbar g  q' , \;
gg \to W Q\Qbar q \qbar' , \;
q\qbar' \to W Q \Qbar q'' \qbar''  , \; 
{\mbox{ etc.}}
\ee
Each of these subprocesses receives contributions from several
possible flavour configurations (e.g. $u\bar{d} \to W Q \Qbar gg$ ,
$u\bar{s} \to W Q \Qbar gg$, etc.). Our subdivision in subprocesses is
designed to allow to sum the contribution of different flavour
configurations by simply adding trivial factors such as parton
densities or CKM factors, which factorize out of a single, flavour
independent,  matrix element. For example the overall
contribution from the first process in the above list is given by 
\be \label{eq:ckm}
\left [ u_1\bar{d}_2\cos^2\theta_c + u_1\bar{s}_2\sin^2\theta_c +
  c_1\bar{s}_2\cos^2\theta_c + c_1\bar{d}_2\sin^2\theta_c \right ]
\times \vert M(q\qbar' \to W Q\Qbar g g)\vert^2 \; , 
\ee 
where $q_i=f(x_i),\; i=1,2$,  are the parton densities for the
quark flavour $q$. 
Contributions from
charge-conjugate or isospin-rotated states can also be summed up,
after trivial momentum exchanges. 
For example, the same matrix element
calculation is used to describe  the four events:
\ba
u(p_1) \dbar(p_2) &\to& b(p_3) \bbar(p_4) g(p_5) g(p_6) e^+(p_5)
\nu(p_6)
\nn \\
\ubar(p_1) d(p_2) &\to& \bbar(p_3) b(p_4) g(p_5) g(p_6) e^-(p_5)
\nubar(p_6)
\nn \\
\dbar(p_1) u(p_2) &\to& \bbar(p_3) b(p_4) g(p_5) g(p_6) \nu(p_5)
e^+(p_6)
\nn \\
d(p_1) \ubar(p_2) &\to& b(p_3) \bbar(p_4) g(p_5) g(p_6) \nubar(p_5)
e^-(p_6) \; . \nn
\ea
Event by event, the flavour configuration for the
assigned subprocess is then selected with a probability
proportional to the relative size of the individual contributions to
the luminosity, weighted by the Cabibbo angles. 

Typically, only few among all possible subprocesses give a substantial
contribution to the cross section. For each event, instead of summing
the weight of all subprocesses, we calculate only one. This is
selected with uniform probability during the first integration cycle,
and a record is kept of each individual
contribution to the cross-section. In subsequent integration
iterations, the accumulated rates of the single channels are used to
weight their selection probabilities. This will significantly improve
the CPU performance of the code, and the unweighting efficiency.

Due to the rapid growth in the number of subprocesses when quarks are
added~\cite{Draggiotis:2000sh}, we limited ourselves to processes with
at most 2 pairs of light quarks (plus pairs of heavy quarks, when
required). In all of the cases considered this is not, however, a
significant limitation on the accuracy of the results.  The full list
of subprocesses available for each hard process is given in
Section~\ref{sec:hard}.

\subsubsection{Phase-space sampling}
The phase-space generation is optimised for each individual hard
process, using generation variables which are most suitable to the
application of typical hadron collider selection cuts.  The
phase-space is mapped with a multidimensional grid, and during the
integration a record is kept of the total weight accumulated within
each bin of the grid.  To further contribute to the efficiency of the
phase-space sampling, independent grids are employed to sample
different subprocesses. In particular, we shall associate one
phase-space grid to each of the following initial states:
\begin{enumerate}
\item $q\qbar$, $q\qbar'$ and charge conjugates
\item $qg$ and $\qbar g$
\item $gq$ and $g \qbar$
\item $gg$
\item $qq$, $qq'$ and  charge conjugates.
\end{enumerate}
For example, the processes $q \qbar'  \to WQ\Qbar  gg$ and 
 $q \qbar'  \to WQ\Qbar  q \qbar$ share the same phase-space
 grid.
 
\subsubsection{Matrix element calculation}
The calculation of the LO matrix elements for the selected hard
process is performed using the \ALPHA\cite{Caravaglios:1995cd}
algorithm, extended to QCD interactions as described
in~\cite{Caravaglios:1999yr,Mangano:2001xp}.  As explained in detail
in~\cite{Mangano:2001xp}, use of the \ALPHA\ algorithm is in our view
essential in order to cope with the complexity of the problem.  All
mass effects are included in the case of massive quarks.  The
calculations are done after having selected, on an event-by-event
basis, polarization, flavour and colour configurations, in order to be
able to provide the shower MC's with the information necessary for the
shower evolution. The sum over polarizations and colours is performed
by summing over multiple events, in a MC fashion.  The choice of
colour basis and the strategy for the determination of the colour
flows necessary for the coherent shower evolution are described
in~\cite{Mangano:2001xp}. 
To the best of our knowledge, a large fraction of the matrix element
calculations documented in this paper have never been performed before
using other calculational tools.  An independent implementation of the
\ALPHA\ algorithm has been exploited for the evaluation of multijet
processes in hadronic collisions
in~\cite{Draggiotis:1998gr,Draggiotis:2002hm}.
 
\subsubsection{PDF sets and  $\mathbf{\alpha_s}$}
 The code library includes a choice among some of the most recent PDF
 parameterizations. They can be selected at the beginning of the run
 through the variable {\tt ndns}, which is  mapped as
 follows\footnote{In addition to these default sets, we
 include in the package the full group of 40 CTEQ6M sets which allow
 the determination of PDF systematics errors~\cite{Pumplin:2002vw}.}:
\def\cteqa {1 & CTEQ4M~\cite{Lai:1996mg} & $[0.116]_2$}
\def\cteqb {2& CTEQ4L~\cite{Lai:1996mg}    &  $[0.116]_2$ }
\def\cteqc {3& CTEQ4HJ~\cite{Lai:1996mg}   & $[0.116]_2$ } 
\def\cteqd {4& CTEQ5M~\cite{Lai:2000wy}    &  $[0.118]_2$ }
\def\cteqe {5& CTEQ5L~\cite{Lai:2000wy}    & $[0.127]_1$ }
\def\cteqf {6& CTEQ5HJ~\cite{Lai:2000wy}   & $[0.118]_2$ } 
\def\cteqg {7& CTEQ6M~\cite{Pumplin:2002vw}& $[0.118]_2$  }
\def\cteqh {8& CTEQ6L~\cite{Pumplin:2002vw} & $[0.118]_2$  }
\def\cteqhp {9& CTEQ6L1~\cite{Stump:2003yu} & $[0.130]_1$  }
\def\cteqsy {10+XX& CTEQ61.XX~\cite{Stump:2003yu}& $[0.118]_2$  }
\def\mrsa{101 & MRST99-1~\cite{Martin:1999ww} & $[0.1175]_2$ }
\def\mrsb{102 & MRST01-1~\cite{Martin:2001es}  & $[0.119]_2$ } 
\def\mrsc{103 & MRST01-2~\cite{Martin:2001es} & $[0.117]_2$ } 
\def\mrsd{104 & MRST01-3~\cite{Martin:2001es} & $[0.121]_2$ } 
\def\mrse{105 & MRST01J~\cite{Martin:2001es} & $[0.121]_2$ } 
\def\mrsf{106 & MRSTLO~\cite{Martin:2002dr} & $[0.130]_1$ }
\def\dummypdf{& & }
\begin{center}
{\small
\begin{tabular}{lll|lll}
{\tt ndns} & PDF & $[\as(\mZ)]_{n_{\rm loop}}$ &
{\tt ndns} & PDF & $[\as(\mZ)]_{n_{\rm loop}}$ \\ \hline
\cteqa & \mrsa \\
\cteqb & \mrsb \\
\cteqc & \mrsc \\
\cteqd & \mrsd \\
\cteqe & \mrse \\
\cteqf & \mrsf \\
\cteqg & \dummypdf \\
\cteqh & \dummypdf \\
\cteqhp & \dummypdf \\
\cteqsy & \dummypdf \\
\end{tabular}
}
\end{center}
where $XX=00,01,\dots,40$ refers to the numbering scheme given
in~\cite{Stump:2003yu}.
In the case of NLO sets we use the 2-loop expression for $\as$: 
\be
\as(Q)=\frac{1}{b_5 \, \log(Q^2/\Lambda_5^2)} - \frac{b'_5 \,
  \log\log(Q^2/\Lambda_5^2) }{b_5^2 \,  \log^2(Q^2/\Lambda_5^2)} \; ,
\ee 
valid for $Q>m_b\equiv4.5$~GeV, 
where $b_5$ and $b'_5$ are the 1- and 2-loop coefficients of the QCD
$\beta$ function, respectively, for 5 flavours. Threshold matching is
applied in the case of $Q<m_b$. In the case of LO sets (such as
CTEQ*L or MRSTLO) we follow the prescriptions used by the authors in
performing the PDF fits. These vary from set to set. For example,  set
CTEQ5L and CTEQ6L1 are
 fitted using a LO expression for $\as$, while CTEQ6L used
the NLO evolution. The table above lists the values of $\as(\mZ)$
corresponding to the various sets, and indicates whether these values
(and the relative evolution to different renormalization scales)
correspond to the 1 or 2 loop formulation.

As more recent sets of PDFs are added to the library, their list can
be obtained by interactively running the code in $\tt imode=3$.

\subsubsection{Electroweak couplings}
In the current version of the \ALPHA\ code the input couplings are
derived from the standard $SU(3)\times SU(2)_L \times U(1)_Y$ tree
level Lagrangian.  The choice of input EW parameters deserves a short
discussion. In \ALPHA, the couplings of the electroweak (EW) and Higgs
($H$) bosons (including the respective selfcouplings) are parametrised
in terms of the $SU(2)$ coupling strength $g$, of the weak mixing
angle $\sin\theta_W$, of the electromagnetic fine structure constant
$\alpha_{em}$, and of the masses of $W$, $Z$ and $H$. We therefore
have a total of 6 parameters needed to specify the value of general EW
matrix elements. If we want to preserve gauge invariance we are
however allowed to use only four independent parameters (plus fermion
masses in the Yukawa sector). Treating $\mH$ as a free parameter,
gauge invariance at the tree level demands that the remaining 5
parameters satisfy the following tree-level relationships: \ba
\cos \theta_W & = & \frac {\mW} {\mZ} \\
e &=& g \; \sin\theta_W \; . \ea The Higgs self-couplings and Yukawa
couplings to fermions are furthermore given by: \ba
\lambda_{hhh} & = & \frac{g \, \mHsq}{4\mW} \\
\lambda_{hhhh} & = & \frac{g^2 \, \mHsq}{32 \, M^2_W} \\
y_f & = & \frac{gm_f}{\sqrt{2}\mW} \; . \ea As a consequence, we cannot
assign to the input parameters the values which are known from the
current accurate experimental measurements, since these values are
only consistent with the radiatively corrected versions of the above
relations.  This is not a merely formal issue: any tiny violation of
the tree-level gauge relationships among the model parameters leads to
violations of the equivalence theorem and leads to unphysical
corrections to the tree-level cross-sections scaling like
$(E_V/M_V)^{2n}$, $n$ being the number of on/off-shell heavy gauge
bosons appearing in the relevant diagrams and $E_V$, $M_V$ their
energy and mass respectively.  In view of the large center of mass
energy available in current and future hadron collisions, these
spurious corrections could be numerically large, leading to the wrong
high-energy behaviour of the cross sections.

The current version of the code provides four choices for the setting
of EW parameters. These choices are controlled by the variable {\tt
  iewopt}, set at running time (default values for this variable are
provided in the code, and are listed in the following sections
describing the individual hard processes). The different options are
listed here below;  the numerical values of the calculated parameters are
obtained using the following set of inputs: $\mW=80.419$, $\mZ=91.188$,
$\sin^2\theta_W=0.231$, $\aem=1/128.89$, $G_F=1.16639\times 10^{-5}$ :
\begin{itemize}
\item[~] {\tt iewopt=0}.  Inputs: $\aem$, $G_{F}$,
  $\sin^2 \theta_W$. Extracted:
  \ba &&  g=\sqrt{4\pi\aem}/\sin\theta_W = 0.6497, \quad
         \mW=g/\sqrt{4\sqrt{2} G_F} = 79.98, \label{eq:iew0} \\
      &&   \mZ = \mW/\cos\theta_W = 91.20 \nn
  \ea
\item[~] {\tt iewopt=1}.  Inputs: $\mW$, $G_{F}$,
  $\sin^2 \theta_W$. Extracted:
  \ba && \mZ=\mW/\cos\theta_W = 91.705, \quad 
  g= ( 4\sqrt{2} \, G_F)^{1/2} \mW =0.6532,  \label{eq:iew1} \\ 
     && \aem = (g \; \sin\theta_W)^2/4\pi = 1/127.51 \nn 
  \ea
% This option is the default for the processes $W+$ jets and $WQ\Qbar+$
% jets, which only depend on  $\mW$ and $g$.
\item[~] {\tt iewopt=2}.  Inputs: $\mZ$, $\aem$,
  $\sin^2 \theta_W$. Extracted:
  \be \mW=\mZ\cos\theta_W = 79.97 , \quad 
  g=\sqrt{4\pi\aem}/\sin\theta_W = 0.6497  \label{eq:iew2} 
  \ee
% This option is the default for the processes $Z+$ jets and $ZQ\Qbar+$
% jets, which are insensitive to $\mW$.
\item[~] {\tt iewopt=3}.  Inputs: $\mZ$, $\mW$,
  $G_F$. Extracted:
  \ba && \sin^2\theta_W = 1- (\mW/\mZ)^2 = 0.2222 , \quad 
  g= ( 4\sqrt{2} \, G_F)^{1/2} \mW =0.6532,  \label{eq:iew3} \\
      &&  \aem = (g \; \sin\theta_W)^2/4\pi = 1/132.5 \nn \; .
  \ea
\end{itemize}
As a default, in all processes we employ {\tt iewopt=3}. We verified
that alternative options generate changes in the rates by at most few percent.
Gauge and Higgs boson widths are calculated at tree level after the
couplings have been selected. With the exception of the class of
processes involving several gauge bosons, which will be discussed in detail
Section~\ref{sec:nw} and where we set boson widths to 0,  \ALPHA\  
uses fixed widths in the propagators. 


\subsection{Unweighting and Shower evolution}
\label{sec:shower}
The starting point for the processing of events through the shower
evolution is the generation of a sample of unweighted events.
This generation takes place through a two-step procedure (more details
are given in the Appendix):
\begin{enumerate}
\item to start with, a run of the parton-level code is performed as
  described in Section~\ref{sec:xsec}. Selecting the running mode option
  {\tt imode=1}, weighted events are stored in a file. To limit the
  size of the file, instead of saving all the event information, we
  simply store the seed of the first random number used in the
  generation of the event, in addition to the event weight.
\item at the end of the generation of the weighted event sample, the
  unweighting is performed by running the code once more, using a
  running mode option {\tt imode=2}. In this running mode, the code
  will sequentially read the events stored in the file, and will
  perform the unweighting using the knowledge of the maximum weight of
  the sample, and of the weight of each individual event. When an
  event is selected by the unweighting, the seed of the random number
  is uploaded and all the information about the event (kinematics,
  flavours, spins and colours) is automatically reconstructed. The
  colour flow for the event is then selected according to the
  algorithm described in~\cite{Mangano:2001xp}: the subamplitudes
  corresponding to all colour flows compatible with the colour state
  of the event are first evaluated using \ALPHA; one of them
  is then randomly extracted with a probability proportional to the
  squared modulus of the relative subamplitude. The momenta of the
  particles, together with their flavour and with the colour flow
  information, are then written to a file, which at the end of the run
  will contain the complete sample of unweighted events.
\end{enumerate}
At this point we are ready to process the events through the shower
evolution. The stored events can be read by the chosen shower MC, the
kinematics, colour and flavour information for each event being
translated into the event format established by the Les Houches
convention~\cite{Boos:2001cv}.

\section{The available hard processes}
\label{sec:hard}

\subsection{$\mathbf{WQ\Qbar+}$~jets}
\label{sec:wqq}
Code for the associated production of $W$, heavy quark ($Q=c,b$ or
$t$) pairs, and jets.  
The physics content of the  $WQ\Qbar+$~jets code has been described in detail
in~\cite{Mangano:2001xp}, where some phenomenological applications are
also presented. We use the notation $W$ as a short hand; what is
actually calculated is the matrix element for a fermion-antifermion final
state. All spin correlations and finite width effects are therefore
accounted for.  
The quoted cross sections refer to a single lepton
family; in the flavour assignment, the code selects by default an
electron. Different flavours can be selected during the unweighting
phase, covering all possible leptonic decays, as well as inclusive
quark decays (for more details see the Appendix~\ref{app:topdec}).
In the case $Q=t$, the top quark is left undecayed.
The EW parameters are fixed by default using the option {\tt iewopt=3}
(see eq.~(\ref{eq:iew1})). Only the leading-order EW diagrams are
included in the calculation.

 The subprocesses considered include all configurations
with up to 2 light-quark pairs; they are listed in
Table~\ref{tab:wqq}, following the notation employed in the
code. 
\begin{table}
\begin{center}
\begin{tabular}{ll|ll|ll}
{\tt jproc} & subprocess & {\tt jproc} & subprocess & {\tt jproc} &
subprocess \\ 
1 &  $q\qbar' \to W Q\Qbar$ 
&2 &  $q g \to q' W Q\Qbar$ 
&3 &  $g q \to q' W Q\Qbar$ 
\\
4 &  $gg \to q \qbar' W Q\Qbar$ 
&5 &  $q\qbar' \to W Q \Qbar q'' \qbar'' $ 
&6 &  $qq'' \to W Q \Qbar q' q'' $ 
\\
7 &  $q'' q \to W Q \Qbar q' q'' $ 
&8 &  $q\qbar \to W Q \Qbar q' \qbar'' $ 
&9 &  $q\qbar' \to W Q \Qbar q \qbar $ 
\\
10 &  $\qbar' q\to W Q \Qbar q \qbar $ 
&11 &  $q\qbar \to W Q \Qbar q \qbar' $ 
&12 &  $q\qbar \to W Q \Qbar q' \qbar $ 
\\
13 &  $q q \to W Q \Qbar q q' $ 
&14 &  $q q' \to W Q \Qbar q q $ 
&15 &  $q q' \to W Q \Qbar q' q' $ 
\\
16 &  $q g \to W Q \Qbar q' q''\qbar'' $ 
&17 &  $g q \to W Q \Qbar q' q''\qbar'' $ 
&18 &  $q g \to W Q \Qbar q q \qbar' $ 
\\
19 &  $q g \to W Q \Qbar q' q \qbar $ 
&20 &  $g q \to W Q \Qbar q q \qbar' $ 
&21 &  $g q \to W Q \Qbar q' q \qbar $ 
\\
22 &  $q g \to W Q \Qbar q' q' \qbar' $ 
&23 &  $g q \to W Q \Qbar q' q' \qbar' $ 
&24 &  $g g \to W Q \Qbar q \qbar' q'' \qbar'' $ 
\\
25 &  $g g \to W Q \Qbar q \qbar q \qbar' $ 
& &
& &
\end{tabular}
\ccaption{}{\label{tab:wqq} Subprocesses included in the $WQ\Qbar+$jets
  code. Additional final-state gluons are not explicitly 
  shown here but are included in the code if the requested light-jet
  multiplicity ($N\le 4$) exceeds the number of indicated final-state partons.
  For example, the subprocess {\tt jproc=1} in the case of 2 light jets
  will correspond to the final state  $q\qbar' \to W Q\Qbar g g$.
  The details can be found in the subroutine {\tt selflav} of
  the file {\tt wqqlib/wqq.f}.}
\end{center}
\end{table}
For all processes, the charge-conjugate ones are always understood.
The above list fully covers all the possible processes with up to 3
light jets in addition to the heavy quarks. In the case of 4 extra
jets, we do not calculate processes with 3 light-quark pairs. Within
the uncertainties of the LO approximation, these can be safely
neglected~\cite{Berends:1991ax}.

The selection of the flavour takes place at run time, when the user is
required to input the heavy quark mass. Masses below 3~GeV are
associated to charm, between 3 and 10 to the bottom, and above 10 to
the top quark. 
As a default, the code generates kinematical configurations defined by
cuts applied to the following variables (the cuts related to the heavy
quarks are only applied in the case of $b$, while top quarks are
always generated without cuts):
\begin{itemize}
\item $\pt^{\rm jet}$, $\eta^{\rm jet}$, $\Delta R_{jj}$
\item $\pt^{ b}$, $\eta^{ b}$, $\Delta R_{b\bbar}$ 
\item $\pt^{\rm \ell}$, $\eta^{\rm \ell}$, $\pt^{\rm \nu}$, $\Delta
  R_{\ell j}$ ,
\end{itemize}
where $\Delta R_{ab}=\sqrt{ [(\eta_a - \eta_b)^2+(\phi_a-\phi_b)^2 ]}$.
The  cut values can be provided by the user at run
time. Additional cuts can be supplied by the user in the 
routine {\tt usrcut} contained in the user file {\tt wqqwork/wqqusr.f}.

In the code initialization phase, 
the user can select among 4 continuous choices for the parametrization
of the factorization and renormalization scale $Q$: a real input
parameter ({\tt qfac}) allows to vary the overall scale of $Q$,
$Q={\tt qfac}\times Q_0$, while the preferred functional form for
$Q_0$ is selected through the integer input parameter {\tt
  iqopt}:
{\renewcommand{\arraystretch}{1.2}
\begin{center}
\begin{tabular}{l||l|l|l|l|l|l|}
{\tt iqopt} & 0 & 1 & 2 & 3 & 4 & 5\\  \hline
$Q_0^2$ & 1 & $m_W^2+ \sum \mTsq$ & $m_W^2$ & $m_W^2+\pt_W^2$ & $\sum
\mTsq$ & $H_T$ 
\end{tabular}
\end{center}
}
where $\mT$ is the transverse mass defined as $\mTsq=m^2+\ptsq$,
and the sum $\sum \mTsq $ extends to all final
state partons (including the heavy quarks, excluding the $W$ decay products).
$H_T$ is the sum of the transverse momenta of all jets and gauge boson
decay products.
The option  {\tt iqopt}=0 allows the user to freeze the scale to an
arbitrary value, which can be selected specifying the numerical value
of {\tt qfac}.   
 
Some numerical benchmark results are given in
Table~\ref{tab:wbbxs} and ~\ref{tab:wttxs}. 
In the case of the minimal jet multiplicity, the results agree with
previous calculations (see e.g.~\cite{Kunszt:1984ri,Mangano:1993kp}).
The following scale and cuts are used:
\ba \label{eq:wqq1}
&& Q^2 = \mWsq + \ptWsq,
\\
        && \pt^{\rm jet}>20~\gev, \quad \vert \eta_j\vert < 2.5, \quad \Delta
        R_{jj} >0.7
\\
        && \pt^{b}>20~\gev, \quad \vert \eta_b \vert < 2.5, \quad \Delta
        R_{b\bbar} >0.7, \Delta R_{bj} >0.7 \; .
\label{eq:wqq2}
\ea
Here and in the following Sections we shall use
the PDF set CTEQ5L. The quoted errors reflect the statistical accuracy
of the integrations. We never tried to go beyond the percent level, to
concentrate the CPU resources on the most computationally demanding channels.
Results for the FNAL Tevatron refer to \ppbar\
collisions at $\sqrt{S}=2$~TeV, those for the LHC refer to $pp$
collisions at $\sqrt{S}=14$~TeV.

{\renewcommand{\arraystretch}{1.2}
\begin{table}
\begin{center}
\begin{tabular}{||l|l|l|l|l|l||}\hline
 & $N = 0$  & $N = 1$ & $N = 2$ & $N = 3$ & $N = 4$
                 \\  \hline
LHC (pb)   & 2.222(4) & 3.013(9) & 1.83(1) & 0.831(8) & 0.307(5)
\\ \hline
FNAL (fb)  &  332.2(7) & 86.2(4) & 18.3(2) & 3.17(3) & 0.44(3)
\\ \hline
\end{tabular}            
\ccaption{}{\label{tab:wbbxs} $\sigma(b \bbar \ell \nu + N~{\rm jets})$
at the Tevatron and 
at the LHC. Parameters and cuts are given
in eqs.~(\ref{eq:wqq1}-\ref{eq:wqq2}). }
\end{center}
\end{table}}

{\renewcommand{\arraystretch}{1.2}
\begin{table}
\begin{center}
\begin{tabular}{||l|l||}\hline
LHC (fb)   & 61.1(4)
\\ \hline
FNAL (fb)  &  1.55(1)
\\ \hline
\end{tabular}            
\ccaption{}{\label{tab:wttxs} $\sigma(t \tbar \ell \nu)$
at the Tevatron and 
at the LHC.}
\end{center}
\end{table}}



\subsection{$\mathbf{Z/\gamma^{*} \, Q\Qbar+}$~jets}
\label{sec:zqq}
We use the notation $Z/\gamma^{*}$ as a short hand; what is actually
calculated is the matrix element for a lepton-pair final state. All
spin correlations and finite width effects are therefore accounted
for. When the final state $\ell^+ \ell^-$ is selected, the
interference between intermediate $Z$ and $\gamma^{*}$ is also included.
The quoted cross sections refer to a single lepton family; in the
flavour assignement, the code selects by default the $e^+ e^-$ pair.
In the case of the final state $\nu \bar\nu$ the quoted cross sections
include the decays to all 3 neutrino flavours, although we always
label the neutrinos as $\nu_e$.  In the case $Q=t$, the top quark is
left undecayed.
The EW parameters are fixed by default using the option {\tt iewopt=3}
(see eq.~(\ref{eq:iew2})).

 The subprocesses considered include all configurations
with up to 2 light-quark pairs; they are listed in
Table~\ref{tab:zqq}, following the notation employed in the
code.
\begin{table}
\begin{center}
\begin{tabular}{ll|ll|ll}
{\tt jproc} & subprocess & {\tt jproc} & subprocess & {\tt jproc} &
subprocess \\ 
1 &  $u\ubar \to Z Q\Qbar$ &  
2 &  $d \dbar  \to Z Q\Qbar$ &  
3 &  $g g \to Z Q\Qbar$ \\
4 &  $g u  \to u Z Q\Qbar$ &  
5 &  $g d  \to d Z Q\Qbar$ &  
6 &  $ ug  \to u Z Q\Qbar$ \\
7 &  $ dg  \to d Z Q\Qbar$ &  
8 &  $g g \to u \ubar Z Q\Qbar$ &  
9 &  $g g \to d \dbar Z Q\Qbar$ \\
10 &  $u\ubar \to u \ubar Z Q\Qbar$ &  
11 &  $d\dbar \to d \dbar Z Q\Qbar$ &  
12 &  $uu     \to u u     Z Q\Qbar$ \\  
13 &  $dd     \to d d     Z Q\Qbar$ &  
14 &  $u\ubar \to u'\ubar' Z Q\Qbar$ &  
15 &  $d\dbar \to d' \dbar' Z Q\Qbar$ \\  
16 &  $uu'     \to u u'     Z Q\Qbar$ &  
17 &  $u\ubar' \to u\ubar' Z Q\Qbar$ &  
18 &  $dd'     \to dd'     Z Q\Qbar$ \\  
19 &  $d\dbar' \to d\dbar' Z Q\Qbar$ &  
20 &  $u\ubar \to d\dbar Z Q\Qbar$ &  
21 &  $d\dbar \to u\ubar Z Q\Qbar$ \\  
22 &  $ud     \to u d     Z Q\Qbar$ &  
23 &  $du     \to  du     Z Q\Qbar$ &  
24 &  $u\dbar \to u\dbar Z Q\Qbar$ \\  
25 &  $d\ubar \to d\ubar Z Q\Qbar$ &  
26 &  $u\ubar \to b\bbar Z Q\Qbar$ &  
27 &  $d\dbar \to b\bbar Z Q\Qbar$ \\  
28 &  $gg     \to b\bbar Z Q\Qbar$     & 
29 &  $gu     \to u u \ubar Z  Q\Qbar $ &
30 &  $ug     \to u u \ubar Z   Q\Qbar$      \\
31 &  $gu     \to  u u' \ubar' Z    Q\Qbar $ &
32 &  $ug     \to u u' \ubar' Z  Q\Qbar $   &
33 &  $gu     \to u d \dbar Z   Q\Qbar$    \\
34 &  $ug     \to u d \dbar Z  Q\Qbar $    &
35 &  $gu     \to u b \bbar Z  Q\Qbar $    &
36 &  $ug     \to u b \bbar Z  Q\Qbar $    \\
37 &  $gd     \to d d \dbar Z  Q\Qbar $    &
38 &  $dg     \to d d \dbar Z  Q\Qbar $    &
39 &  $gd     \to d d' \dbar' Z  Q\Qbar$   \\
40 &  $dg     \to d d' \dbar' Z  Q\Qbar $  &
41 &  $gd     \to d u \ubar Z  Q\Qbar $    &
42 &  $dg     \to d u \ubar Z  Q\Qbar $    \\
43 &  $gd     \to d b \bbar Z Q\Qbar  $   &
44 &  $dg     \to d b \bbar Z  Q\Qbar $   & & 
\end{tabular}
\ccaption{}{\label{tab:zqq} Subprocesses included in the $Z/\gamma^*
  Q\Qbar+$jets code. It is always understood that quarks $u$ and $d$
  represent generic light quarks of type up or down. The $Z$ in the table
  stands for a neutral $\ell^+ \ell^-$ ($\nu \bar\nu$) lepton pair.
  Additional final-state gluons are not explicitly shown here but are
  included in the code if the requested light-jet multiplicity ($N\le
  4$) exceeds the number of indicated final-state partons.  For
  example, the subprocess {\tt jproc=1} in the case of 2 light jets
  will correspond to the final state $u\ubar \to Z Q\Qbar g g$.  The
  details can be found in the subroutine {\tt selflav} of the file
  {\tt zqqlib/zqq.f}.}
\end{center}
\end{table}
For each process, the charge-conjugate ones are always understood.
The above list fully covers all the possible processes with up to 3
light jets in addition to the heavy quarks. In the case of 4 or more
extra jets, we do not calculate processes with 3 light-quark pairs. As
in the case of associated production with a $W$, we expect that,
within the uncertainties of the LO approximation, these can be safely
neglected.

As a default, the code generates kinematical configurations defined by
cuts applied to the following variables (the cuts related to the heavy
quarks are only applied in the case of $b$, while top quarks are
always generated without cuts):
\begin{itemize}
\item $\pt^{\rm jet}$, $\eta^{\rm jet}$, $\Delta R_{jj}$
\item $\pt^{\rm Q}$, $\eta^{\rm Q}$, $\Delta R_{Q\Qbar}$ 
\item $\pt^{\ell}$, $\eta^{\ell}$, $\Delta R_{\ell j}$, $m(\ell^+\ell^-)$, 
for $\ell^+ \ell^-$ final states
\item missing $E_T$ for $\nu \bar\nu$ final states.  
\end{itemize}
The cut on the dilepton invariant mass allows to optimise the sampling
of the DY mass spectrum if the user is interested in events off the
$Z$ peak. Additional cuts can be supplied by the user in an
appropriate routine.  The choice of factorization and renormalization
scale is similar to what given for the $WQ\Qbar$ processes, with the
$W$ mass replaced by the mass of the DY pair.
Some benchmark results are given in Table~\ref{tab:zqqxs}, obtained
for the following set of cuts and scale choice:
\ba \label{eq:zqq1}
&& Q^2 = \mZsq + \ptZsq,
\quad
80~\gev \leq m_{ll} \leq 100~\gev
\\
        && \pt^{\rm jet}>20~\gev, \quad \vert \eta_j\vert < 2.5, \quad \Delta
        R_{jj} >0.7
\\
        && \pt^{b}>20~\gev, \quad \vert \eta_b \vert < 2.5, \quad \Delta
        R_{b\bbar} >0.7, R_{bj} >0.7 \; .
\label{eq:zqq2}
\ea
{\renewcommand{\arraystretch}{1.2}
\begin{table}
\begin{center}
\begin{tabular}{||l|l|l|l|l|l||}\hline
%& & & \\
 & $N = 0$  & $N = 1$ & $N = 2$ & $N = 3$ & $N = 4$\\ 
%& & & \\ 
\hline
%& & & \\ 
LHC, (fb) & 1840(5)  & 1085(4) & 444(3) &  154(2) & 44(1) \\ 
%& & &\\ 
\hline
%& & & \\ 
FNAL, (fb) & 49.3(1) & 13.18(5) & 2.57(2) & 0.400(4) & 0.0511(5) \\ 
%& & &\\ 
\hline
\end{tabular}            
\ccaption{}{\label{tab:zqqxs}
$\sigma(\ell^+ \ell^- b \bbar + N~{\rm jets})$
at the Tevatron and at the LHC. Parameters and cuts are given
in eqs.~(\ref{eq:zqq1}-\ref{eq:zqq2}).}
\end{center}
\end{table}}

\subsection{$\mathbf{W+}$~jets}
\label{sec:wjets}
As in the previous cases, we use the notation $W$ as a short hand; what is
actually calculated is the matrix element for a lepton+neutrino final
state. All spin correlations and finite width effects are therefore
accounted for. The quoted cross sections refer to a single lepton
family; in the flavour assignment, the code selects by default an
electron.
The EW parameters are fixed by default using the option {\tt iewopt=3}
(see eq.~(\ref{eq:iew1})).

 The subprocesses considered include all configurations
with up to 2 light-quark pairs; they are listed in
Table~\ref{tab:wjets}, following the notation employed in the
code.
\begin{table}
\begin{center}
\begin{tabular}{ll|ll|ll}
{\tt jproc} & subprocess & {\tt jproc} & subprocess & {\tt jproc} &
subprocess \\ 
1 &  $q\qbar' \to W  $ 
&2 &  $q g \to q' W  $ 
&3 &  $g q \to q' W  $ 
\\
4 &  $gg \to q \qbar' W  $ 
&5 &  $q\qbar' \to W  q'' \qbar'' $ 
&6 &  $qq'' \to W  q' q'' $ 
\\
7 &  $q'' q \to W  q' q'' $ 
&8 &  $q\qbar \to W  q' \qbar'' $ 
&9 &  $q\qbar' \to W  q \qbar $ 
\\
10 &  $\qbar' q\to W  q \qbar $ 
&11 &  $q\qbar \to W  q \qbar' $ 
&12 &  $q\qbar \to W  q' \qbar $ 
\\
13 &  $q q \to W  q q' $ 
&14 &  $q q' \to W  q q $ 
&15 &  $q q' \to W  q' q' $ 
\\
16 &  $q g \to W  q' q''\qbar'' $ 
&17 &  $g q \to W  q' q''\qbar'' $ 
&18 &  $q g \to W  q q \qbar' $ 
\\
19 &  $q g \to W  q' q \qbar $ 
&20 &  $g q \to W  q q \qbar' $ 
&21 &  $g q \to W  q' q \qbar $ 
\\
22 &  $q g \to W q' q' \qbar' $ 
&23 &  $g q \to W q' q' \qbar' $ 
&24 &  $g g \to W q \qbar' q'' \qbar'' $ 
\\
25 &  $g g \to W q \qbar q \qbar' $ 
& &
& &
\end{tabular}
\ccaption{}{\label{tab:wjets} Subprocesses included in the $W+$jets
  code. Additional final-state gluons are not explicitly 
  shown here but are included in the code if the requested light-jet
  multiplicity ($N\le 6$) exceeds the number of indicated final-state partons.
  For example, the subprocess {\tt jproc=1} in the case of 2 jet
  will correspond to the final state  $q\qbar' \to W g g$.
  The details can be found in the subroutine {\tt selflav} of
  the file {\tt wjetlib/wjet.f}.}
\end{center}
\end{table}
For each process, the charge-conjugate ones are always understood.
The above list fully covers all the possible processes with up to 3
light jets in addition to the heavy quarks. In the case of 4 extra
jets, we do not calculate processes with 3 light-quark pairs. Within
the uncertainties of the LO approximation, these can be safely
neglected~\cite{Berends:1991ax}.

As a default, the code generates kinematical configurations defined by
cuts applied to the following variables:
\begin{itemize}
\item $\pt^{\rm jet}$, $\eta^{\rm jet}$, $\Delta R_{jj}$
\item $\pt^{\rm Q}$, $\eta^{\rm Q}$, $\Delta R_{Q\Qbar}$ 
\item $\pt^{\rm \ell}$, $\eta^{\rm \ell}$, $\pt^{\rm \nu}$, $\Delta
  R_{\ell j} \; . $
\end{itemize}
The respective threshold values can be provided by the user at run
time. Additional cuts can be supplied by the user in an appropriate
routine. 
The choice of scale follows the same conventions as for the $WQ\Qbar$
case.


Some benchmark results are given in Table~\ref{tab:wjxs}, obtained
for the following set of cuts and scale choice:
\ba \label{eq:wj1}
&& Q^2 = \mWsq + \ptWsq,
\\
        && \pt^{\rm jet}>20~\gev, \quad \vert \eta_j\vert < 2.5, \quad \Delta
        R_{jj} >0.7 \; .
\label{eq:wj2}
\ea

{\renewcommand{\arraystretch}{1.2}
\begin{table}
\begin{center}
\begin{tabular}{||l|l|l|l|l|l|l|l||}\hline
 & $N = 0$ & $N = 1$ & $N = 2$  & $N = 3$ & $N = 4$ & $N = 5$ & $N = 6$
\\ \hline
LHC (pb) & 18068(4) & 3412(4) & 1130(2) & 342.9(1.4) & 100.6(1.4) & 27.6(4) & 7.14(15)
                 \\  \hline
FNAL (pb)  & 2087.0(6) & 225.8(2) & 37.3(2) &  5.66(6) & 0.745(4) & 0.0864(15) & 0.0086(2)
\\ \hline
\end{tabular}           
\ccaption{}{\label{tab:wjxs} $\sigma(\ell \nu + N~{\rm jets})$
at the Tevatron and 
at the LHC. Parameters and cuts are given
in eqs.~(\ref{eq:wj1}-\ref{eq:wj2}).}
\end{center}
\end{table}}
For processes with up to 4 jets, we verified the numerical agreement
with the results of the \vecbos\ code~\cite{Berends:1991ax}.


\subsection{$\mathbf{W+c+}$~jets}
\label{sec:wcjets}
The processes included in this code are a subset of the ones 
treated in section~\ref{sec:wjets}. Here the final states consisting 
exclusively of one $c$ (or $\bar c$) quark in association with a 
$W$ and additional light jets 
are singled out. Events with $charm$ quark pairs are not included, and
should be generated using the {\tt wqq} processes. Likewise we have
not included processes with a charm in
the intial state.
All spin correlations and finite width effects in the fermionic decay
of the $W$ are 
accounted for. The $W$ decay modes can be selected when running 
the code with {\tt imode=2} to produce unweighted events, as discussed
in the appendix B.4. 
The EW parameters are fixed by default using the option {\tt iewopt=3}
(see eq.~(\ref{eq:iew1})).

The subprocesses considered include all configurations
with up to 2 light-quark pairs (where ``light'' includes the charm); 
they are listed in
Table~\ref{tab:wcjets}, following the notation employed in the
code.
\begin{table}
\begin{center}
\begin{tabular}{ll|ll}
{\tt jproc} & subprocess & {\tt jproc} & subprocess \\ 
1 &  $g c' \to W c  $ 
&2 &  $c' g \to W c  $ 
\\
3 &  $gg \to W c \cbar'  $ 
%&4 &  $qq'' \to W  q' q'' (q'/q''=c) $ 
&4 &  $c'q \to W  c q $ 
\\
5 &  $q c' \to W  c q $ 
&6 &  $q\qbar \to W  q' \qbar'' (q'/q''=c) $
\\
%7 &  $q\qbar \to W  q \qbar' (q/q'=c) $ 
7 &  $c' \cbar' \to W  c' \bar{c} $ 
%&8 &  $q\qbar \to W  q' \qbar (q/q'=c) $ 
&8 &  $c'\cbar' \to W  c \cbar' $ 
\\
%9 &  $q q \to W  q q' (q/q'=c)$ 
9 &  $c' c' \to W  c' c$ 
&10 &  $c' g \to W  c q''\qbar'' $ 
\\
11 &  $g c' \to W  c q''\qbar'' $ 
&12 &  $c' g \to W  c' c' \cbar $ 
\\
13 &  $c' g \to W  c c' \cpbar $ 
&14 &  $g c' \to W  c' c' \cbar $ 
\\
15 &  $g c' \to W  c c' \cpbar $ 
&16 &  $g g \to W  c \cbar' q'' \qbar'' $ 
\\
17 &  $g g \to W  c' \cpbar c' \cbar $ 
& &
\end{tabular}
\ccaption{}{\label{tab:wcjets} Subprocesses included in the $W+c+$jets
  code. The symbol $c'$ refers to either of the two Cabibbo partners
  of the charm quark, $d$ or $s$.
  Additional final-state gluons are not explicitly 
  shown here but are included in the code if the requested light-jet
  multiplicity ($N\le 5$) exceeds the number of indicated final-state partons.
  For example, the subprocess {\tt jproc=1} in the case of 2 jet
  will correspond to the process  $g c' \to W c g g$.
  The details can be found in the subroutine {\tt selflav} of
  the file {\tt wcjetlib/wcjet.f}.}
\end{center}
\end{table}
For each process, the charge-conjugate ones are always understood.
The above list fully covers all the possible processes with up to 2
light jets in addition to the heavy quarks. In the case of 3 extra
jets, we do not calculate processes with 3 light-quark pairs. As in
the case of $W$+jet production, we expect these to be negligible.

As a default, the code generates kinematical configurations defined by
cuts applied to the following variables:
\begin{itemize}
\item $\pt^{\rm jet}$, $\eta^{\rm jet}$, $\Delta R_{jj}$
\item $\pt^{\rm \ell}$, $\eta^{\rm \ell}$, $\pt^{\rm \nu}$, $\Delta
  R_{\ell j} \; ,$ 
\end{itemize}
where the cuts on $c$-quarks are the same as for light jets. 
The respective threshold values can be provided by the user at run
time. Additional cuts can be supplied by the user in an appropriate
routine. 
The choice of scale follows the same conventions as for the $WQ\Qbar$
case.
Some benchmark results are given in Table~\ref{tab:wcjxs}, obtained
for the following set of cuts and scale choice:
\ba \label{eq:wcj1}
&& Q^2 = \mWsq + \ptWsq,
\\
        && \pt^{\rm jet}>20~\gev, \quad \vert \eta_j\vert < 2.5, \quad \Delta
        R_{jj} >0.7 \; .
\label{eq:wcj2}
\ea

{\renewcommand{\arraystretch}{1.2}
\begin{table}
\begin{center}
\begin{tabular}{||l|l|l|l|l|l|l||}\hline
 & $N = 0$ & $N = 1$ & $N = 2$  & $N = 3$ & $N = 4$ & $N = 5$            \\ \hline
LHC (pb)    & 418.9(8) & 183.7(8) & 55.6(3) & 14.9(1)& 3.65(3)& 0.848(7) \\ \hline
FNAL (pb)   & 8.74(2)  & 2.39(1)&  0.360(2)& 0.0422(2)& 0.00414(2)& 0.000353(2)\\ \hline
\end{tabular}           
\ccaption{}{\label{tab:wcjxs} $\sigma(\ell \nu + c +N~{\rm jets})$
at the Tevatron and 
at the LHC. Parameters and cuts are given
in eqs.~(\ref{eq:wcj1}-\ref{eq:wcj2}).}
\end{center}
\end{table}}

\subsection{$\mathbf{Z/\gamma^{*}+}$~jets}
\label{sec:zjets}
We use the notation $Z/\gamma^{*}$ as a short hand; what is actually
calculated is the matrix element for a charged lepton or neutrino pair
final state. All spin correlations and finite width effects are
therefore accounted for. When the final state $\ell^+ \ell^-$ is
selected, the interference between intermediate $Z$ and $\gamma^{*}$ is
also included. The quoted cross sections refer to a single lepton
family; in the flavour assignment, the code selects by default the
$e^+ e^-$ pair. In the case of the final state $\nu \bar\nu$ the
quoted cross sections include the decays to all 3 neutrino flavours,
although we always label the neutrinos as $\nu_e$.
The EW parameters are fixed by default using the option {\tt iewopt=3}
(see eq.~(\ref{eq:iew2})).

All subprocesses with up to 2 light quark pairs are included. 
This means that the cross-sections with up to 3 final-state partons are 
exact. The emission of additional hard gluons can however be
calculated, and the current version of the code works with up to a 6
final-state jets.
The subprocesses considered are listed in Table~\ref{tab:zjets}.

\begin{table}[h]
\begin{center}
\vskip .3cm
\begin{tabular}{ll|ll|ll}
{\tt jproc} & subprocess & {\tt jproc} & subprocess & {\tt jproc} &
subprocess \\ 
1  &  $\qu \qubar \to   Z$ 
&2 &  $\qd \qdbar \to   Z$ 
&3 &  $ g \qu \to \qu Z $ 
\\
4  &  $g \qd  \to \qd Z$ 
&5 &  $\qu g  \to \qu Z $ 
&6 &  $\qd g  \to \qd Z $ 
\\
7  &  $g g  \to \qu \qubar Z$ 
&8 &  $g g  \to \qd \qdbar Z$ 
&9 &  $\qu \qubar \to \qu \qubar Z$ 
\\
10  &  $\qd \qdbar \to \qd \qdbar Z  $ 
&11 &  $ \qu \qu   \to \qu \qu Z$ 
&12 &  $ \qd \qd   \to \qd \qd Z$ 
\\               
13 &   $\qu \qubar \to \qu' \qubar' Z $ 
&14 &  $\qd \qdbar \to \qd' \qdbar' Z  $ 
&15 &  $\qu \qu' \to \qu \qu' Z$ 
\\
16  &  $\qu \qubar'  \to \qu \qubar' Z $ 
&17 &  $\qd \qd'     \to \qd \qd' Z $ 
&18 &  $ \qd \qdbar' \to \qd \qdbar' Z $ 
\\
19 &   $\qu \qubar \to \qd \qdbar Z $ 
&20 &  $\qd \qdbar \to \qu \qubar Z$ 
&21 &  $\qu \qd    \to \qu \qd Z$ 
\\               
22 & $\qd \qu    \to \qu \qd Z $ 
& 23 & $\qu \qdbar \to \qu \qdbar Z$ 
& 24 & $\qd \qubar \to \qd \qubar Z$
\\ 
25 & $g \qu     \to \qu \qu  \qubar Z   $
& 26 & $\qu g   \to \qu \qu  \qubar Z   $
& 27 & $g \qu   \to \qu \qu' \qubar' Z   $
\\
28 & $\qu g     \to \qu \qu' \qubar' Z   $
& 29 & $g \qu   \to \qu \qd  \qdbar Z   $
& 30 & $\qu g   \to \qu \qd  \qdbar Z   $
\\
31 & $g \qd     \to \qd \qd  \qdbar Z   $
& 32 & $\qd g   \to \qd \qd  \qdbar Z  $
& 33 & $g \qd   \to \qd \qd' \qdbar' Z   $
\\
34 & $\qd g     \to \qd \qd' \qdbar' Z   $
& 35 & $g \qd   \to \qd \qu  \qubar Z   $
& 36 & $\qd g   \to \qd \qu  \qubar Z   $
\end{tabular}
\ccaption{}{\label{tab:zjets} Subprocesses included in the
  $Z/\gamma^{*}+$jets code.  It is always understood that quarks $u$ and
  $d$ represent generic quarks of type up or down. The $Z$ in the
  table stands for a neutral $\ell^+ \ell^-$ ($\nu \bar\nu$) lepton
  pair.  The complex conjugate processes are also understood.
  Additional final-state gluons are not explicitly shown here but are
  included in the code if the requested jet multiplicity ($N\le 6$)
  exceeds the number of indicated final-state partons.  For example,
  the subprocess {\tt jproc=1} in the case of 2  jets will
  correspond to the final state $u\ubar \to Z  g g$.  The
  details can be found in the subroutine {\tt selflav} of the file
  {\tt zjetlib/zjet.f}.}
\end{center}
\end{table}

As a default, the code generates kinematical configurations defined by
cuts applied to the following variables:
\begin{itemize}
\item $\pt^{\rm jet}$, $\eta^{\rm jet}$, $\Delta R_{jj}$
\item $\pt^{\ell}$, $\eta^{\ell}$, $\Delta R_{\ell j}$ for $\ell^+ \ell^-$ 
final states
\item missing $E_T$ for $\nu \bar\nu$ final states.  
\end{itemize}
Additional cuts can be supplied by the user in an appropriate routine.
The choice of scale follows the same conventions as for the $ZQ\Qbar$
case.

Some benchmark results are given in Table~\ref{tab:zjxs}, obtained
for the following set of cuts and scale choice:
\ba \label{eq:zj1}
&& Q^2 = \mZsq + \ptZsq,
\\
        && \pt^{\rm jet}>20~\gev, \quad \vert \eta_j\vert < 2.5, \quad \Delta
        R_{jj} >0.7
\\
        && 80~\gev \leq m_{ll} \leq 100~\gev \; .
\label{eq:zj2}
\ea

{\renewcommand{\arraystretch}{1.2}
\begin{table}
\begin{center}
\begin{tabular}{||l|l|l|l|l|l|l|l||}\hline
 & $N = 0$  & 
$N = 1$ & $N = 2$ & $N = 3$ & $N = 4$ & $N = 5$ & $N = 6$ \\ 
\hline
LHC (pb)  &  1526(1) &  320.9(5) & 104.6(2) & 31.6(2) & 9.4(2) & 
2.51(4) & 0.65(2) \\ 
\hline
FNAL (pb)  & 179.4(2) & 21.44(2) & 3.36(1) & 0.489(2) & 
0.0630(3) & 0.00700(4)& 0.000690(6) \\ 
\hline
\end{tabular}            
\ccaption{}{\label{tab:zjxs} $\sigma(\ell^+ \ell^- + N~{\rm jets})$
at the Tevatron and at LHC. Parameters and cuts are given
in eqs.~(\ref{eq:wj1}-\ref{eq:wj2}).}
\end{center}
\end{table}}

\subsection{$\mathbf{ nW+mZ+j\gamma+lH+k}$ jets}
\label{sec:nw}
The code computes all processes where EW gauge bosons and/or Higgs
particles are produced on shell plus (up to 3) additional light jets.
The limitations in the number of final state particles is therefore
the following: $ k \le 3$ and $n+m+j+l+k \le 8$. It is intended that at
least one EW gauge boson or Higgs particle appear in the final state 
 ($n+m+j+l>0$). All contributions from
QCD and EW processes are included. In particular, all processes of
the gauge-boson-fusion type are present when the number of final state
quarks is at least 2 (e.g. $qq\to qqH$).
All gauge bosons are decayed to fermion pairs, taking therefore into account 
all spin correlations among the decay products by means of exact 
matrix elements. More information on gauge-invariance issues in
presence of decays are discussed below, and details on the choice of
the decay modes can 
be found in Appendix~\ref{app:topdec}. The Higgs boson decays will be soon available. 

In the case of $m=l=j=0$
($n=l=j=0$) the code reproduces the results of the programs described in 
Sections~\ref{sec:wjets} and~\ref{sec:zjets}, up to the following effects:
final-state finite-width corrections, 
are absent here since the gauge bosons are kept on shell; branching
ratios depend on the selected decay mode for the $Z$, while they are
kept to 1 in the case of the $W$ bosons, whose final states are
selected during the unweighting phase (see Appendix B4); 
the $\gamma^* \to \ell^+\ell^- $ contributions are only present in the code
of Section~\ref{sec:zjets}; EW production of jets ($W\to jj$, $Z/\gamma^*
\to jj$, plus EW boson exchanges between quarks) are included here.

The EW parameters are fixed by default using the option {\tt iewopt=3}
(see eq.~(\ref{eq:iew3})). In this way we are guaranteed that all gauge
boson masses are consistent with the measured values. The coupling
strengths extracted from the fixed inputs, however, will differ by few
percents from their best values. The user can select the option 
 {\tt iewopt=0}
(see eq.~(\ref{eq:iew0})), where the couplings are matched to the
radiatively corrected best values, at the price of working with gauge
boson masses which differ by few percents from the measured
masses. Both schemes are equally consistent at the LO. We verified
that cross-sections obtained using the two schemes differ from each
other at the level of few percent, a negligible effect compared to
the large uncertainties related to the choice of factorization and
renormalization scales. 


The subprocesses considered are listed in Table~\ref{tab:WZH}, where a
distinction is made according to the number of final state $W$'s.  If
$n$ is odd, the whole Cabibbo structure of the matrix element is
correctly taken into account. If $n$ is even, we work in the $\cos
\theta_c= 1$ approximation. The reason of this choice is that, in the
latter case, different Cabibbo structures may interfere at the
amplitude level. The quantity ${\cal F}^{\pm}$ in Table~\ref{tab:WZH}
stands for $nW+mZ+j\gamma+lH$ with $n$ odd, while ${\cal F}^{0}$ stands for
$nW+mZ+j\gamma+lH$ when $n$ is even.

\begin{table}
\begin{center}
\vskip .3cm
{\bf $\mathbf{n}$ odd}
\vskip .3cm
\begin{tabular}{ll|ll|ll}
{\tt jproc} & subprocess & {\tt jproc} & subprocess & {\tt jproc} &
subprocess \\ 
1  &  $q\qbar' \to {\cal F}^{\pm} $ 
&2 &  $g q \to q' {\cal F}^{\pm} $ 
&3 &  $q g \to q' {\cal F}^{\pm} $ 
\\
4 &   $gg \to q \qbar' {\cal F}^{\pm} $ 
&5 &  $q\qbar' \to {\cal F}^{\pm} q'' \qbar'' $ 
&6 &  $qq'' \to {\cal F}^{\pm} q' q'' $ 
\\
7 &  $q'' q \to {\cal F}^{\pm} q' q'' $ 
&8 &  $q''\qbar'' \to {\cal F}^{\pm} q \qbar' $ 
&9 &  $q\qbar' \to {\cal F}^{\pm} q \qbar $ 
\\
10 &  $\qbar' q\to {\cal F}^{\pm} q \qbar $ 
&11 &  $q\qbar \to {\cal F}^{\pm} q \qbar' $ 
&12 &  $q\qbar \to {\cal F}^{\pm} q' \qbar $ 
\\
 13 &  $q q  \to {\cal F}^{\pm} q q' $ 
&14 &  $q q' \to {\cal F}^{\pm} q q $ 
&15 &  $q q' \to {\cal F}^{\pm} q' q' $ 
\\
16  &  $q g \to {\cal F}^{\pm} q' q''\qbar'' $ 
&17 &  $g q \to {\cal F}^{\pm} q' q''\qbar'' $ 
&18 &  $q g \to {\cal F}^{\pm} q q \qbar' $ 
\\
19 &  $q g \to  {\cal F}^{\pm} q' q \qbar $ 
&20 &  $g q \to {\cal F}^{\pm} q q \qbar' $ 
&21 &  $g q \to {\cal F}^{\pm} q' q \qbar $ 
\end{tabular}
\vskip .3cm
{\bf $\mathbf{n}$ even}
\vskip .3cm
\begin{tabular}{ll|ll|ll}
{\tt jproc} & subprocess & {\tt jproc} & subprocess & {\tt jproc} &
subprocess \\ 
1  &  $q\qbar \to   {\cal F}^0$ 
&2 &  $g q    \to q {\cal F}^0$ 
&3 &  $q g    \to q {\cal F}^0$ 
\\
4  &  $gg     \to q \qbar {\cal F}^0$ 
&5 &  $qq     \to q q     {\cal F}^0$ 
&6 &  $qq     \to q' q'   {\cal F}^0$ 
\\
7  &  $q q'  \to  q q'      {\cal F}^0$ 
&8 &  $q q'' \to  q q''     {\cal F}^0$ 
&9 &  $q q'  \to  q'' q'''  {\cal F}^0$ 
\\
10  &  $q \qbar \to   q \qbar     {\cal F}^0$ 
&11 &  $q \qbar \to   q' \qbar'   {\cal F}^0$ 
&12 &  $q \qbar \to   q'' \qbar'' {\cal F}^0$ 
\\
13 &   $q \qbar'  \to   q \qbar' {\rm~or~} \qbar q' {\cal F}^0$ 
&14 &  $q \qbar'' \to   q \qbar''                  {\cal F}^0$ 
&15 &  $q \qbar'  \to   q'' \qbar'''               {\cal F}^0$ 
\\
16  &  $q  \qbar'' \to q' \qbar''' {\cal F}^0$ 
&17 &  $g q  \to q q \qbar {\cal F}^0 $ 
&18 &  $q g  \to q q \qbar {\cal F}^0 $ 
\\
19 &   $g q  \to q q' \qbar' {\cal F}^0$ 
&20 &  $q g  \to q q' \qbar' {\cal F}^0$ 
&21 &  $g q  \to q q'' \qbar''{\cal F}^0 $ 
\\
  22 & $q g  \to q q'' \qbar'' {\cal F}^0$ 
& 23 & $g q  \to q' q' \qbar   {\cal F}^0$ 
& 24 & $q g  \to q' q' \qbar   {\cal F}^0$ 
\\
  25 & $g q  \to q' q'' \qbar''' {\cal F}^0$ 
& 26 & $q g  \to q' q'' \qbar''' {\cal F}^0$ 
& &
\end{tabular}
\ccaption{}{\label{tab:WZH} Subprocesses included in the 
$nW+mZ+j\gamma+lH+$jets
  code, for the cases n=odd ($nW+mZ+j\gamma+lH={\cal F}^{\pm}$)
and n=even ($nW+mZ+j\gamma+lH={\cal F}^0$).
It is always understood that quarks $q$ and $q'$ belong to the same
iso-doublet, while $q$ and $q''$ belong to different iso-doublets.
  Additional final-state
  gluons are not explicitly indicated but are included in the
  code. 
  The details can be found in the subroutine {\tt selflav} of
  the file {\tt vbjetlib/vbjet.f}.}
\end{center}
\end{table}

As a default, the code generates kinematical configurations defined by
cuts applied to the following variables:
\begin{itemize}
\item $\pt^{\rm jet}$, $\eta^{\rm jet}$, $\Delta R_{jj} \; $
\item $\pt^\gamma$, $\eta^\gamma$, $\Delta R_{\gamma j}$, 
      $\Delta R_{\gamma \gamma} \, .$
\end{itemize}
Additional cuts can be supplied by the user in an appropriate
routine. Cuts on  the leptons or on  the Higgs decay products
should be imposed in the analysis routine, using the momentum and
flavour information as specified in Appendix~\ref{app:topdec}.

In the code initialization phase, the user can select among 4
continuous choices for the parametrization of the factorization and
renormalization scale $Q$: a real input parameter ({\tt qfac}) allows
to vary the overall scale of $Q$, $Q={\tt qfac}\times Q_0$, while the
preferred functional form for $Q_0$ is selected through an integer
input parameter ({\tt iqopt}=0,1,2,3).  In more detail:
{\renewcommand{\arraystretch}{1.2}
\begin{center}
\begin{tabular}{l||l|l|l|l|}
{\tt iqopt} & 0 & 1 & 2 & 3 \\  \hline
$Q_0^2$ & 1 & $\sum m_B^2+ \sum \mTsq$ & $(\sum m_B)^2$ & ${\hat{s}}$ 
\end{tabular}
\end{center}
}
where $M_B$ represent the EW boson masses (vectors as well as
Higgses), $\mT$ is the transverse mass defined as $\mTsq=m^2+\ptsq$,
and the sum $\sum \mTsq $ extends to all final
state jets.

Some benchmark results are given in
Tables~\ref{tab:wwxs}-\ref{tab:whxs}, using the scale $Q = \sum M_B$,
with the sum extended over all bosons in the final state.

In order to deal, in a gauge invariant way, with the problem of the 
unstable bosons appearing in the intermediate states, we computed 
the matrix element by using zero-width propagators and by
cutting away events around the mass of the unstable particle 
$M$ in such a way to keep the area of a Breit-Wigner distribution.
In other words, for a particle of mass 
$M$ and width $\Gamma$, our effective cut $s_0$ 
is determined by the equation  
\bqa
\int_{-\infty}^{M^2-s_0} ds \frac{1}{(s-M^2)^2} = 
\int_{-\infty}^{M^2} ds \frac{1}{(s-M^2)^2+\Gamma^2 M^2} \; .
\eqa
that gives gives $s_0= \frac{2\Gamma M}{\pi}$. In practice the 
program discards events for which $|s-M^2| < {\tt winsize}\cdot \Gamma 
\cdot M$, with the adjustable parameter ${\tt winsize}$ set, by
default, to the value ${\tt winsize}= \frac{2}{\pi}$.
The described procedure gives sensible results 
when the portion of resonance cut away is of the order of a few GeV. 
Otherwise, holes start becoming visible in the distributions.
This is the case for $Z$'s and $W$'s vector bosons. But problems arise
for very heavy Higgses. For this reason we put a protection in the code
to avoid runs when $\Gamma_H > 10$ GeV.

We checked that our algorithm reproduces, within few percents, 
the results obtained by using a fudge approach, defined by setting
$\Gamma= 0$ in the matrix element and by multiplying
the final result by the factor 
$$ \frac{1}{1+\left(\frac{M\Gamma}{s-M^2}\right)^2}\; .$$

%In order to deal, in a gauge invariant way, with the problem of the 
%appearing instable bosons in the intermediate states, 
%we used a fudge approach.
%For any given instable particle of mass 
%$M$ and width $\Gamma$, we used 
%$\Gamma= 0$ in the matrix element computation and multiplied
%the result, whenever necessary and only in a range of invariant mass 
%$s$ such that $|\sqrt s- M| \le \Gamma/2$, by the factor 
%$$ \frac{1}{1+\left(\frac{M\Gamma}{s-M^2}\right)^2}\,.$$

In the case of 3 gauge boson production, we verified the agreement
with the results shown in ref.~\cite{Haywood:1999qg}, 
up to the $WZZ$ channel, whose
rate is erroneously reported in the tables of 
Section 5.33\footnote{A. Ghinkulov, private communication.}. In the
case of four boson production, a previous calculation has been
documented in \cite{Barger:1989cp}. A comparison of the results of our
code with the numbers presented in that paper shows however some
discrepancies. 

As a consistency check of our calculations for large multiplicities of
gauge bosons, we verified that the production rates for multiple gauge
bosons when the Higgs mass is above the threshold for diboson decay
are well approximated by the incoherent sum of the processes mediated
by an on-shell Higgs, plus those where the Higgs contribution 
is suppressed in the
intermediate states. In the tables we show, this is for example seen
in the comparison of the $WWjj$ vs $Hjj$ or $VVV$ vs $HV$
 rates at \mH=200~GeV.

{\renewcommand{\arraystretch}{1.2}
\begin{table}
\begin{center}
\begin{tabular}{||l|l|l|l||}\hline
& $WW$ & $WZ$ & $ZZ$ 
                 \\  \hline
LHC (pb)   & 74.8(5) & 28.1(2) & 10.85(6) 
\\ \hline
FNAL (pb)  & 8.51(5) & 2.45(1) & 1.027(5)
\\ \hline
\end{tabular}            
\ccaption{}{\label{tab:wwxs} Diboson production 
at the Tevatron and at the LHC. }
\end{center}
\end{table}}


{\renewcommand{\arraystretch}{1.2}
\begin{table}
\begin{center}
\begin{tabular}{||l|l|l|l|l||}\hline
 $WW+$jets & $jj$, central & 
            $jjj$, central 
& $jj$, fwd 
& $jjj$,fwd 
                 \\  \hline
LHC, \mH=120 (pb)   & 18.6(2) &8.16(9) &0.256(4) & 0.365(9)
\\ \hline	     	    
LHC, \mH=200 (pb)   & 18.9(2) &8.3(1)  &0.546(24)& 0.389(9)
\\ \hline	     	    
FNAL, \mH=120 (fb)  & 336(1)  &49.1(2) &0.201(1) & 0.0789(3) \\ \hline
FNAL, \mH=200 (fb)  & 364(2)  &54.9(8) &0.415(4) & 0.096(2) \\ \hline
\end{tabular}            
\ccaption{}{\label{tab:wwjxs} Associated production of $WW$ and jets,
at the Tevatron and at the LHC. In all cases, $\et_j>20~\gev$ and
$\Delta R_{jj}>0.7$. The {\em central} configurations correspond to
all jets with $\vert \eta_j \vert<2.5$. The {\em fwd} configurations
have two jets in opposite rapidity hemispheres 
with $2.5 < \vert \eta_j \vert<5$, plus a central jet in the $jjj$ case.}
\end{center}
\end{table}}



{\renewcommand{\arraystretch}{1.2}
\begin{table}       
\begin{center}
\begin{tabular}{||l|l|l|l|l||}\hline
 $H+$jets & $jj$, central & 
           $jjj$, central
& $jj$, fwd
& $jjj$,fwd
                 \\  \hline
LHC, \mH=120 (pb)   &1.27(1)  &0.458(6)& 0.504(3) &0.089(1) 
\\ \hline	       	       	  
LHC, \mH=140 (pb)   &0.96(1)  &0.320(4) & 0.458(2) &0.078(2) 
\\ \hline	       	       	  
LHC, \mH=200 (pb)   &0.459(4) &0.132(3) & 0.346(2) &0.0528(6) 
\\ \hline	       	       	  
FNAL, \mH=120 (fb)  & 121(1)  &26.7(2) & 0.769(2) &0.0730(5) 
\\ \hline	       	       	  
FNAL, \mH=140 (fb)  & 75.0(4) &16.0(1) & 0.573(1) &0.0513(3) 
\\ \hline	       	       	  
FNAL, \mH=200 (fb)  &21.8(2)  &4.28(3) & 0.246(1) &0.0193(1) 
\\ \hline
\end{tabular}            
\ccaption{}{\label{tab:hjxs} Associated production of $H$ and jets,
at the Tevatron and at the LHC. In all cases, $\et_j>20~\gev$ and
$\Delta R_{jj}>0.7$. The {\em central} configurations correspond to
all jets with $\vert \eta_j \vert<2.5$. The {\em fwd} configurations
have two jets in opposite rapidity hemispheres 
with $2.5 < \vert \eta_j \vert<5$, plus a central jet in the $jjj$ case.}
\end{center}
\end{table}}


{\renewcommand{\arraystretch}{1.2}
\begin{table}
\begin{center}
\begin{tabular}{||l|l|l|l|l||}\hline
        & $WWW$ & $WWZ$ & $WZZ$ & $ZZZ$ 
   \\  \hline
LHC, \mH=120  & 130(1)&98(2)&31.0(4) &10.9(1)
   \\  \hline 	       	      
LHC, \mH=200  & 305(5)&199(4)&95(1) & 45.2(5)
   \\  \hline	       	      
FNAL, \mH=120 & 6.39(2)&5.13(4)&1.21(1) &0.519(2)
   \\  \hline	       	      
FNAL, \mH=200 & 18.1(2)&13.5(2)&5.47(6) &3.36(3)
  \\  \hline
\end{tabular}            
\ccaption{}{\label{tab:wwwxs} Triboson production at the 
Tevatron and 
at the LHC (fb).}
\end{center}
\end{table}}

{\renewcommand{\arraystretch}{1.2}
\begin{table}
\begin{center}
\begin{tabular}{||l|l|l|l|l|l||}\hline
        & $WWWW$ & $WWWZ$ & $WWZZ$ & $WWWWW$ & $WWWWZ$ 
                 \\  \hline
LHC, \mH=120 & 0.606(6) & 0.72(1) & 0.48(1) & $5.8(1)\cdot 10^{-3}$
                                            & $10.8(3)\cdot 10^{-3}$
                 \\  \hline
\end{tabular}            
\ccaption{}{\label{tab:4wxs} 4- and 5-boson production at the 
%Tevatron ($\ppbar$ at $\sqrt{S}=2$~TeV) and 
LHC (fb). }
\end{center}
\end{table}}


{\renewcommand{\arraystretch}{1.2}
\begin{table}
\begin{center}
\begin{tabular}{||l|l|l|l||l|l|l|l||}\hline
$WH$ & \mH=120 & \mH=140 & \mH=200 &
$ZH$ & \mH=120 & \mH=140 & \mH=200 
\\ \hline
LHC (pb)   & 1.364(8) & 0.833(4) & 0.251(2) & & 0.727(4) & 0.449(2) & 0.137(1)
\\ \hline
FNAL (fb)  & 122.4(6)& 70.0(2) & 16.55(4) & & 75.0(3) & 44.2(2) & 11.24(3)
\\ \hline
\end{tabular}            
\ccaption{}{\label{tab:whxs} H+boson production 
at the Tevatron and LHC.}
\end{center}
\end{table}}

\subsection{$\mathbf{ Q\Qbar+}$ jets}
\label{sec:2Q}
Production of heavy quark pairs plus jets. Here $Q$ can be either $c$,
$b$, or $t$. In all cases, additional production of $Q\bar{Q}$ pairs
of $Q$ in the initial state are excluded.
The subprocesses considered include all configurations with up to two
light quark pairs, where ``light'' quarks are those with mass lower
than $Q$. They are listed in Table~\ref{tab:2Q}, following 
the notations employed in the code. The list covers all possible 
processes involving up to 4 light-parton jets with 2 light quark pairs.
Subprocesses involving 3 light-quark pairs, which would only appear in
the case of 4 jets in addition to the heavy quarks, are not included,
as they are expected to contribute a negligible rate.

As a default, the code generates kinematical configurations defined by
cuts applied to the following variables (the cuts related to the heavy
quarks are only applied in the case of $b$, while top quarks are
always generated without cuts):
\begin{itemize}
\item $\pt^{\rm jet}$, $\eta^{\rm jet}$, $\Delta R_{jj},\Delta R_{jb}$
\item $\pt^{ b}$, $\eta^{ b}$, $\Delta R_{b\bbar} \; .$
\end{itemize}
The respective threshold values can be provided by the user at run
time. Additional cuts can be supplied by the user in an appropriate
routine. In the case $Q=t$ the decay of $t \bar t$ pairs in six fermions 
(in the on-shell approximation) is enforced.
With this option the decay of the top quarks takes into account 
all spin correlations among the decay products by means of exact 
matrix elements. More information on the selection of decay products can 
be found in Appendix~\ref{app:topdec}.

In the code initialization phase, the user can select 
between 3 choices for the parameterization of the 
factorization and renormalization scale $Q$. A real input
parameter ({\tt qfac}) allows to vary the overall scale of $Q$,
$Q={\tt qfac}\times Q_0$, while the preferred functional form for
$Q_0$ is selected through an integer input parameter ({\tt
iqopt}=0,1,2).  In more detail:
{\renewcommand{\arraystretch}{1.2}
\begin{center}
\begin{tabular}{l||l|l|l}
{\tt iqopt} & 0 & 1 & 2\\  \hline
$Q_0^2$ & 1 & $\sum \mTsq$ & $\hat{s}$ \\
\end{tabular}
\end{center}}
where $\mT$ is the transverse mass defined as $\mTsq=m^2+\ptsq$,
$m$ is the heavy quark mass, and the sum $\sum \mTsq $ extends to all final
state partons (including the heavy quarks).

\begin{table}
\begin{center}
\begin{tabular}{ll|ll|ll}
{\tt jproc} & subprocess & {\tt jproc} & subprocess & {\tt jproc} &
subprocess \\ 
1 &  $g g \to  Q\Qbar$ 
&2 &  $q \qbar \to Q\Qbar$ 
&3 &  $g q \to  Q\Qbar q$ 
\\
4 &  $ q g \to Q\Qbar q$ 
&5 &  $g g  \to  Q \Qbar q \qbar$ 
&6 &  $q \qbar  \to Q \Qbar q \qbar$ 
\\
7 &  $q \qbar \to Q \Qbar q' \qbar'$ 
&8 &  $q q' \to Q \Qbar q q'$ 
&9 &  $q\qbar' \to  Q \Qbar q \qbar'$ 
\\
10 &  $q q\to Q \Qbar q q$ 
&11 &  $g q \to Q \Qbar q q'\qbar'$ 
&12 &  $q g  \to  Q \Qbar q q' \qbar'$ 
\\
13 &  $g q \to  Q \Qbar q q \qbar$ 
&14 &  $q g \to Q \Qbar q q \qbar $ 
&15 &  $g g \to  Q \Qbar q \qbar q \qbar$ 
\\
16 &  $g g \to  Q \Qbar q \qbar q' \qbar'$ 
& &  
& &
\end{tabular}
\ccaption{}{\label{tab:2Q} Subprocesses included in the $Q\Qbar+$jets
  code. Additional final-state gluons are not explicitly 
  shown here but are included in the code if the requested light-jet
  multiplicity ($N\le 6$) exceeds the number of indicated final-state partons.
  For example, the subprocess {\tt jproc=1},
  in the case of 2 extra jets, will correspond to the final state  
  $gg\to Q\Qbar g g$.
  The details can be found in the subroutine {\tt selflav} of
  the file {\tt 2Qlib/2Q.f}. For each process, the charge-conjugate 
  ones are always understood.}
\end{center}
\end{table}


Some benchmark results are given in Table~\ref{tab:QQxs}, obtained
for the following set of cuts and scale choice:
\ba \label{eq:2Qcut1}
t\tbar: && \mt=175~\gev, Q^2=\mtsq
\\
b\bbar: && \mb=4.75~\gev, Q^2=(\ptbsq+\ptbbsq+\sum \ptsq_j)/(2+N)
\\
        && \pt^{\rm jet}>20~\gev, \quad \vert \eta_j\vert < 2.5, \quad \Delta
        R_{jj} >0.7
\\
        && \pt^{b}>20~\gev, \quad \vert \eta_b \vert < 2.5, \quad \Delta
        R_{b\bbar} >0.7, \Delta R_{bj} >0.7 \; .
\label{eq:2Qcut2}
\ea
In the case of 0 and 1 jet, we find agreement with the results obtained
using the \oacube\ code of ref.~\cite{Mangano:jk}.
{\renewcommand{\arraystretch}{1.2}
\begin{table}
\begin{center}
\begin{tabular}{||l|l|l|l|l|l|l|l||}\hline
%& & & \\
$Q \Qbar + N~{\rm jets}$  & $N= 0$  & 
$N = 1$ & $N = 2$ & $N = 3$ & $N=4$ & $N=5$ & $N=6$ \\ 
\hline
%& & & \\ 
$Q=t$, LHC (pb)  & 530.0(8) & 462.6(6) & 255(1) & 111.5(5) & 42.4(4)  
& 14.07(16) & 4.36(8) \\ 
\hline
%& & & \\ 
$Q=t$, FNAL (fb) &  6,364(8) &  1,592(3) & 282(1) & 40.6(3) & 4.83(4) & 
0.483(6) & 0.0419(9) \\ 
\hline
%& & & \\ 
$Q=b$, LHC (nb)   &  1,533(4)
& 422(1) & 130.2(6)  & 30.9(4) & 7.5(4) 
& 1.53(6) & 0.337(9) \\ 
\hline
%& & & \\ 
$Q=b$, FNAL (pb)   & 72,1(1) &  12,15(2) & 2,51(1) & 365(7) & 47.3(9) 
& 5.6(2)& 0.58(2)\\ 
\hline
\end{tabular}            
\ccaption{}{\label{tab:QQxs} $\sigma(Q \Qbar + N~{\rm jets})$
at the Tevatron and 
at the LHC, with parameters and cuts given in
eqs.~(\ref{eq:2Qcut1}-\ref{eq:2Qcut2}).}
\end{center}
\end{table}}






\subsection{$\mathbf{ Q\Qbar Q'\Qbar'+}$ jets}
\label{sec:4Q}
The list of available processes is given in Table~\ref{tab:4Q}. This
covers all possible processes up to 2 light jets and at most 1 light
quark pair.  The cases with one additional heavy quark pair (e.g.
$t\tbar b\bbar b\bbar$) are also included. Subprocesses with two
light-quark pairs give a negligible contribution to the case of 2
extra jets, and are not calculated.  For the parameterization of the
factorization and renormalization scale the user has two choices, {\tt
  iqopt}=0,1,2, as described in the following Table:
{\renewcommand{\arraystretch}{1.2}
\begin{center}
\begin{tabular}{l||l|l|l}
{\tt iqopt} & 0 & 1 & 2 \\ \hline
$Q_0^2$ & 1 & $\sum \mTsq$ & $\hat{s}$ \\
\end{tabular}
\end{center}}
where $\mT$ is the transverse mass defined as $\mTsq=m^2+\ptsq$,
and the sum $\sum \mTsq $ extends to all final
state partons (including the heavy quarks).

Again, as a default, the code generates kinematical configurations defined by
cuts applied to the following variables (the cuts related to the heavy
quarks are only applied in the case of $b$, while top quarks are
always generated without cuts):
\begin{itemize}
\item $\pt^{\rm jet}$, $\eta^{\rm jet}$, $\Delta R_{jj}, \Delta R_{jb}$
\item $\pt^{ b}$, $\eta^{ b}$, $\Delta R_{b\bbar} \; .$
\end{itemize}
The respective threshold values can be provided by the user at run
time. Additional cuts can be supplied by the user in an appropriate
routine. 

\begin{table}
\begin{center}
\begin{tabular}{ll|ll|ll}
{\tt jproc} & subprocess & {\tt jproc} & subprocess & {\tt jproc} &
subprocess \\ 
1 &  $g g \to  Q\Qbar Q'\Qbar'$ 
&2 &  $q \qbar \to Q\Qbar Q'\Qbar' $ 
&3 &  $g q \to  Q\Qbar Q'\Qbar' q$ 
\\
4 &  $ q g \to Q\Qbar Q'\Qbar' q$ 
&5 &  $g g  \to  Q \Qbar Q'\Qbar' q \qbar$ 
&6 &  $g g  \to Q \Qbar Q'\Qbar' b {\bar{b}}$ 
\end{tabular}
\ccaption{}{\label{tab:4Q} Subprocesses included in the $Q\Qbar Q'\Qbar'+$
  jets code. Additional final-state gluons are not explicitly 
  shown here but are included in the code if the requested light-jet
  multiplicity ($N\le 4$) exceeds the number of indicated final-state partons.
 For example, the subprocess {\tt jproc=1}, in the case of 2 extra 
  jets, will correspond to the final state  $gg\to Q\Qbar Q'\Qbar' g g$.
  The details can be found in the subroutine {\tt selflav} of
  the file {\tt 4Qlib/4Q.f}. For each process, the charge-conjugates 
  ones are always understood. }
\end{center}
\end{table}
Some benchmark results are given in Table~\ref{tab:4Qxs}, obtained
for the following set of cuts and scale choice:
\mt=175~\gev, \mb=4.75~\gev, 
\ba \label{eq:4Qcut1}
t\tbar t\tbar {\rm ~and~} t\tbar b\bbar : && Q^2=\mtsq
\\
b\bbar b\bbar: && Q^2=(\ptbsq+\ptbbsq+\sum \ptsq_j)/(2+N)
\\
        && \pt^{\rm jet}>20~\gev, \quad \vert \eta_j\vert < 2.5, \quad \Delta
        R_{jj} >0.7
\\
        && \pt^{b}>20~\gev, \quad \vert \eta_b \vert < 2.5, \quad \Delta
        R_{b\bbar} >0.7, \Delta R_{bj} >0.7 \; .
\label{eq:4Qcut2}
\ea

{\renewcommand{\arraystretch}{1.2}
\begin{table}
\begin{center}
\begin{tabular}{||l|l|l|l|l|l||}\hline
%& & & \\
$Q \Qbar Q' \Qbar' + N~{\rm jets}$  & $N = 0$  & 
$N = 1$ & $N = 2$ & $N = 3$ & $N = 4$\\ 
%& & & \\ 
\hline
%& & & \\ 
$t \tbar t \tbar$, LHC (fb) &  12.73(8) & 17.4(2) & 13.5(1) & 7.55(6) 
& 3.48(5)\\ 
%& & &\\ 
\hline
%& & & \\ 
$t \tbar b \bbar$, LHC (pb)   &  1.35(1) & 1.47(2) &  
0.94(2) & 0.457(8) & 0.189(4)\\ 
%& & &\\ 
\hline
%& & & \\ 
$t \tbar b \bbar$, FNAL (fb)  & 3.44(3) & 0.95(1) & 0.154(1) & 
0.0187(2) & 0.00187(5) \\ 
%& & &\\ 
\hline
%& & & \\ 
$b \bbar b \bbar$, LHC (pb) & 477(2)  & 259(5) & 95(1) & 28.6(6) &
25.0(3) \\ 
%& & &\\ 
\hline
%& & & \\ 
$b \bbar b \bbar$, FNAL (pb) & 6.64(5) & 2.25(3) & 0.470(5) & 0.076(1) 
& 0.0025(5) \\ 
%& & &\\ 
\hline
\end{tabular}            
\ccaption{}{\label{tab:4Qxs} $\sigma(Q \Qbar Q' \Qbar' + N~{\rm jets})$
at the Tevatron and 
at the LHC, with parameters and cuts given in
eqs.~(\ref{eq:4Qcut1}-\ref{eq:4Qcut2}).}
\end{center}
\end{table}}

\subsection{$\mathbf{Q \Qbar H +}$ jets}
\label{sec:QQH}
The list of processes is given in 
Table~\ref{tab:QQH}. The Higgs is produced only via Yukawa couplings
to the heavy quarks, no other EW process is included.
All cases with up to 2 light-quark pairs are
included, covering in full the possible final states with up to 3
jets in addition to the $QQH$ system. The code will otherwise deal
with up to 4 extra jets.
\begin{table}
\begin{center}
\begin{tabular}{ll|ll|ll}
{\tt jproc} & subprocess & {\tt jproc} & subprocess & {\tt jproc} & 
subprocess \\  
1  &  $g g \to  Q\Qbar H$   &
2 &  $q \qbar \to Q\Qbar H$ &
3 &  $g q \to Q\Qbar q H$ \\
4 &  $q g \to Q\Qbar q H$ &
5 &  $g g \to Q\Qbar q \qbar H$ &
6 &  $g g \to Q\Qbar b \bbar H$ \\
7 &  $q q \to Q\Qbar q q H$ &
8 &  $q q' \to Q\Qbar q q' H$ &
9 &  $q \qbar \to Q\Qbar q \qbar H$ \\
10 &  $q \qbar' \to Q\Qbar q \qbar' H$ &
11 &  $q \qbar \to Q\Qbar q' \qbar' H$ &
12 &  $q \qbar \to Q\Qbar b \bbar H$ \\
13 &  $g q \to Q\Qbar q \qbar q H$ &
14 &  $q g \to Q\Qbar q \qbar q H$ &
15 &  $g q \to Q\Qbar q' \qbar' q H$  \\
16 &  $q g \to Q\Qbar q' \qbar' q H$ &
17 &  $g q \to Q\Qbar b \bbar q H$ &
18 &  $q g \to Q\Qbar b \bbar q H$  \\
\end{tabular}
\ccaption{}{\label{tab:QQH} Subprocesses included in the $Q\Qbar H$ code. 
  The details can be found in the subroutine {\tt selflav} of
  the file {\tt QQhlib/QQh.f}. For each process, the charge-conjugate
  subprocesses are always understood.}
\end{center}
\end{table}

As a default, the code generates kinematical configurations defined by
cuts applied to the following variables:
\begin{itemize}
\item $\pt^{\rm jet}$, $\eta^{\rm jet}$, $\Delta R_{jj}, \Delta R_{jb}$
\item $\pt^{ b}$, $\eta^{ b}$, $\Delta R_{b\bar{b}} \; .$
\end{itemize}
The respective threshold values can be provided by the user at run
time. Additional cuts can be supplied by the user in an appropriate
routine. In the case $Q=t$ the decay of $t \bar t$ pairs in six fermions 
(in the on-shell approximation). 
With this option the decay of the top quarks takes into account 
all spin correlations among the decay products by means of exact 
matrix elements. More information on how to use 
the selection of decay products can 
be found in Appendix~\ref{app:topdec}. The Higgs boson decay will be soon available.

For the parameterization of the factorization  and 
renormalization scale  the user has three 
choices, {\tt iqopt}=0,1,2  as described in the following Table:
{\renewcommand{\arraystretch}{1.2}
\begin{center}
\begin{tabular}{l||l|l|l}
{\tt iqopt} & 0 & 1 & 2  \\  \hline
$Q_0^2$ & 1 & $\mHsq + \sum \mTsq$  & $\hat{s}$  \\
\end{tabular}
\end{center}}
where $\mT$ is the transverse mass defined as $\mTsq=m^2+\ptsq$,
and the sum $\sum \mTsq $ extends to all final
state partons (including the heavy quarks).

Some benchmark results are given in Table~\ref{tab:QQhxs}, obtained
for the following set of cuts and scale choice:
\ba \label{eq:QQh1}
t\tbar : && Q^2=(2 \mt + \mH)^2
\\
b\bbar : && Q^2=\mHsq + (\ptbsq+\ptbbsq)/2 \; .
\label{eq:QQh2}
\ea
 As a default, the Yukawa coupling of the Higgs to the bottom
quarks is evaluated at the b-quark pole mass. Since the rate is
directly proportional to $y_b^2$, the result corresponding to the 
choice $y_b\propto m_b(\mH)$ can be obtained via a trivial rescaling.
The numbers we obtain in the case of 0 extra jets agree with what
 found in the literature~\cite{Carena:2000yx,spira}, after possibly
 correcting for the difference between pole and running $b$-mass.
{\renewcommand{\arraystretch}{1.2}
\begin{table}
\begin{center}
\begin{tabular}{||l|l|l|l||}\hline
%& & & \\
$Q \Qbar H$  & $\mH = 120~\gev$  & $\mH = 150~\gev$ & $\mH = 200~\gev$ \\ 
%& & & \\ 
\hline
%& & & \\ 
$Q=t$, LHC (fb) & 401(2) & 212(1) & 89.1(4) \\ 
%& & &\\ 
\hline
%& & & \\ 
$Q=t$, FNAL (fb) & 4.22(2) & 2.00(1) & 0.637(4) \\ 
%& & &\\ 
\hline
%& & & \\ 
$Q=b$, LHC (fb) & 599(3) & 279(3) & 99(2) \\ 
%& & &\\ 
\hline
%& & & \\ 
$Q=b$, FNAL (fb) & 3.73(3) & 1.20(1) & 0.240(2) \\ 
%& & &\\ 
\hline
\end{tabular}            
\ccaption{}{\label{tab:QQhxs} $\sigma(Q \Qbar H)$,
at the Tevatron and 
at the LHC, with parameters given in
eqs.~(\ref{eq:QQh1}-\ref{eq:QQh2}). No cuts applied.}
\end{center}
\end{table}}

\subsection{$\mathbf{ N}$ jets}
\label{sec:Njets}
The subprocesses considered include all configurations with up to two
light quark pairs. They are listed in Table~\ref{tab:Njets}, following
the notations employed in the code. The list covers all possible
processes involving up to 6 light-parton jets with 2 light quark
pairs.  Subprocesses involving 3 light-quark pairs, which only appear
for 4 or more jets, are not included, but are expected to contribute a
negligible rate (they are fully included in the {\tt \small NJETS}
code by Berends et al~\cite{Berends:1989ie}).  In both initial and
final states we only assume as quark types $u$, $d$, $s$ and $c$. For
the generation of events with heavier quarks ($b$ and $t$), we suggest
using the {\tt 2Q} element of the package.

As a default, the code generates kinematical configurations defined by
cuts applied to the following variables:
\begin{itemize}
\item $\pt^{\rm jet}$, $\eta^{\rm jet}$, $\Delta R_{jj} \; .$
\end{itemize}
The respective threshold values can be provided by the user at run
time. Additional cuts can be supplied by the user in an appropriate
routine. 
In the code initialization phase, the user can select 
between 3 choices for the parameterization of the 
factorization and renormalization scale $Q$. A real input
parameter ({\tt qfac}) allows to vary the overall scale of $Q$,
$Q={\tt qfac}\times Q_0$, while the preferred functional form for
$Q_0$ is selected through an integer input parameter ({\tt
iqopt}=0,1,2).  In more detail:
{\renewcommand{\arraystretch}{1.2}
\begin{center}
\begin{tabular}{l||l|l|l}
{\tt iqopt} & 0 & 1 & 9 \\  \hline
$Q_0^2$ & 1 & $\sum_{jets}\pt^2$ & $\hat{s}$ \\
\end{tabular}
\end{center}}
where the sum extends over the $\pt^2$ of all the final state jets.

\begin{table}
\begin{center}
\begin{tabular}{ll|ll|ll}
{\tt jproc} & subprocess & {\tt jproc} & subprocess & {\tt jproc} &
subprocess \\ 
1 &  $g g \to  gg$ 
&2 &  $q \qbar \to gg$ 
&3 &  $g q \to  qg $ 
\\
4 & $ qg \to qg$
&5 &  $g g  \to   q \qbar$ 
&6 &  $q q   \to qq $ 
\\
7 &  $q q' \to q q'$ 
&8 &  $q\qbar' \to  q \qbar'$ 
&9 &  $q\qbar \to  q \qbar$ 
\\
10 &  $q\qbar \to  q' \qbar'$ 
 & & & &
\end{tabular}
\ccaption{}{\label{tab:Njets} Subprocesses included in the $N$jets
  code. $N-2$ additional final-state gluons are not explicitly 
  shown here but are included in the code, with $N$ up to 6.
  For example, the subprocess {\tt jproc=4},
  in the case of $N=4$ corresponds to the final state  
  $qg \to qggg$.
  The details can be found in the subroutine {\tt selflav} of
  the file {\tt Njetlib/Njet.f}. For each process, the charge-conjugates 
  ones are always understood.}
\end{center}
\end{table}

Some benchmark results are given in Table~\ref{tab:NJxs}, obtained
for the following set of cuts and scale choice:

\ba
\label{eq:NJscale}
&&Q^2=\langle \pt \rangle\\
&&\pt^{\rm jet}>20~\gev, \quad \vert \eta_j\vert < 2.5, \quad \Delta
        R_{jj} >0.7 \; .
\label{eq:NJcut}
\ea



\begin{table}
\begin{center}
\begin{tabular}{||l|l|l|l|l|l||}\hline
%& & & \\
N jets & $N = 2$ & $N = 3$ & $N=4$ & $N=5$ & $N=6$ \\ 
\hline
%& & & \\ 
LHC (nb)  & 375(1)$\cdot 10^3$ & 24.5(2)$\cdot 10^3$ 
& 4,174(8) & 709(1) & 126.3(4)\\ 
\hline
%& & & \\ 
FNAL (nb) &  23,706(60) &  857(5) & 90.9(1) & 8.66(1) & 826(1)$\cdot 10^{-3}$ \\ 
\hline
\end{tabular}            
\ccaption{}{\label{tab:NJxs} $\sigma(N~{\rm jets})$
at the Tevatron and 
at the LHC, with parameters and cuts given in
eqs.~(\ref{eq:NJscale})-(\ref{eq:NJcut}). }
\end{center}
\end{table}

\subsection{$\mathbf{m \gamma +}$~jets}
\label{sec:gammajets}
A final state with $N\ge 0$ jets and $m \ge 1$ real 
photons with $N+m\leq 8$ is generated.
The EW parameters are fixed by default using the option {\tt iewopt=3}
(see eq.~(\ref{eq:iew2})).
All subprocesses with up to 2 light quark pairs are included. 
This means that the cross-sections with up to 3 final-state partons are 
exact. The emission of additional hard gluons can however be
calculated.
The subprocesses considered are listed in Table~\ref{tab:gammajets}.

\begin{table}[h]
\begin{center}
\vskip .3cm
\begin{tabular}{ll|ll|ll}
{\tt jproc} & subprocess & {\tt jproc} & subprocess & {\tt jproc} &
subprocess \\ 
1  &  $\qu \qubar \to   (m\gamma)$ 
&2 &  $\qd \qdbar \to   (m\gamma)$ 
&3 &  $ g \qu \to \qu (m\gamma) $ 
\\
4  &  $g \qd  \to \qd (m\gamma)$ 
&5 &  $\qu g  \to \qu (m\gamma) $ 
&6 &  $\qd g  \to \qd (m\gamma) $ 
\\
7  &  $g g  \to \qu \qubar (m\gamma) $ 
&8 &  $g g  \to \qd \qdbar (m\gamma)$ 
&9 &  $\qu \qubar \to \qu \qubar (m\gamma)$ 
\\
10  &  $\qd \qdbar \to \qd \qdbar (m\gamma)  $ 
&11 &  $ \qu \qu   \to \qu \qu (m\gamma)$ 
&12 &  $ \qd \qd   \to \qd \qd (m\gamma)$ 
\\               
13 &   $\qu \qubar \to \qu' \qubar' (m\gamma) $ 
&14 &  $\qd \qdbar \to \qd' \qdbar' (m\gamma) $ 
&15 &  $\qu \qu' \to \qu \qu' (m\gamma)$ 
\\
16  &  $\qu \qubar'  \to \qu \qubar' (m\gamma) $ 
&17 &  $\qd \qd'     \to \qd \qd' (m\gamma) $ 
&18 &  $ \qd \qdbar' \to \qd \qdbar' (m\gamma) $ 
\\
19 &   $\qu \qubar \to \qd \qdbar (m\gamma) $ 
&20 &  $\qd \qdbar \to \qu \qubar (m\gamma)$ 
&21 &  $\qu \qd    \to \qu \qd (m\gamma)$ 
\\               
22 & $\qd \qu    \to \qu \qd (m\gamma) $ 
& 23 & $\qu \qdbar \to \qu \qdbar (m\gamma)$ 
& 24 & $\qd \qubar \to \qd \qubar (m\gamma)$
\\ 
25 & $g \qu     \to \qu \qu  \qubar (m\gamma)   $
& 26 & $\qu g   \to \qu \qu  \qubar (m\gamma)   $
& 27 & $g \qu   \to \qu \qu' \qubar' (m\gamma)   $
\\
28 & $\qu g     \to \qu \qu' \qubar' (m\gamma)   $
& 29 & $g \qu   \to \qu \qd  \qdbar (m\gamma)   $
& 30 & $\qu g   \to \qu \qd  \qdbar (m\gamma)   $
\\
31 & $g \qd     \to \qd \qd  \qdbar (m\gamma)   $
& 32 & $\qd g   \to \qd \qd  \qdbar (m\gamma)  $
& 33 & $g \qd   \to \qd \qd' \qdbar' (m\gamma)   $
\\
34 & $\qd g     \to \qd \qd' \qdbar' (m\gamma)   $
& 35 & $g \qd   \to \qd \qu  \qubar (m\gamma)   $
& 36 & $\qd g   \to \qd \qu  \qubar (m\gamma)  $
\end{tabular}
\ccaption{}{\label{tab:gammajets} Subprocesses included in the
  $\gamma+$jets code.  It is always understood that quarks $u$ and
  $d$ represent generic quarks of type up or down. The $m$ in the
  table stands for the number of photons.  
  The complex conjugate processes are also understood.
  Additional final-state gluons are not explicitly shown here but are
  included in the code if the requested jet multiplicity ($N\le 8-m$)
  exceeds the number of indicated final-state partons.  For example,
  the subprocess {\tt jproc=1} in the case of 2  jets and 1 photon will
  correspond to the final state $u\ubar \to \gamma  g g$.  The
  details can be found in the subroutine {\tt selflav} of the file
  {\tt phjetlib/phjet.f}.}
\end{center}
\end{table}

As a default, the code generates kinematical configurations defined by
cuts applied to the following variables:
\begin{itemize}
\item $\pt^{\rm jet}$, $\eta^{\rm jet}$, $\Delta R_{jj}$
\item $\pt^{\gamma}$,  $\eta^{\gamma}$, $\Delta R_{\gamma j}$  . 
\end{itemize}
If the final state contains more than one photon then the identification
$\Delta R_{\gamma j}=\Delta R_{\gamma \gamma}$ is made.

Additional cuts can be supplied by the user in an appropriate routine.
In the code initialization phase, 
the user can select between 3 continuous choices for the parametrization
of the factorization and renormalization scale $Q$: a real input
parameter ({\tt qfac}) allows to vary the overall scale of $Q$,
$Q={\tt qfac}\times Q_0$, while the preferred functional form for
$Q_0$ is selected through the integer input parameter {\tt
  iqopt}:
{\renewcommand{\arraystretch}{1.2}
\begin{center}
\begin{tabular}{l||l|l|l}
{\tt iqopt} & 0 & 1                                  & 2\\  \hline
$Q_0^2$     & 1 & $\sum\pt_{j}^2+\sum\pt_{\gamma}^2$ & $\sum \pt_{j}^2$
\end{tabular}
\end{center}
}

Some benchmark results are given in Table~\ref{tab:gammajxs}, obtained
for the following set of cuts and scale choice:
\ba \label{eq:gammaj1}
&& Q^2 = \mZsq,
\\
        && \pt^{\rm jet}>20~\gev, \quad \vert \eta_j\vert < 2.5, \quad \Delta
        R_{jj} >0.4 \; ,
\\
        && \pt^{\rm \gamma}>20~\gev, \quad \vert \eta_\gamma\vert < 2.5, \quad \Delta R_{\gamma j} >0.4 \; .
\label{eq:gammaj2}
\ea

{\renewcommand{\arraystretch}{1.2}
\begin{table}
\begin{center}
\begin{tabular}{||l|l|l|l|l||}\hline
 & $N = 1$  & 
$N = 2$ & $N = 3$ & $N = 4$ \\ 
\hline
$1\gamma$ (pb)  & 89.8(2)  & 19.7(2)  & 7.45(8) & 2.59(4) \\ 
\hline
 & $N = 0$  & 
$N = 1$ & $N = 2$ & $N = 3$ \\ 
\hline
$2\gamma$ (pb)  & 45.5(1)  & 25.3(1)  & 18.7(2) & 9.6(2) \\ 
\hline
%FNAL (pb)  & 179.4(2) & 21.44(2) & 3.36(1) & 0.489(2) & 
%0.0630(3) & 0.00700(4)& 0.000690(6) \\ 
\hline
\end{tabular}            
\ccaption{}{\label{tab:gammajxs} $\sigma(1,2\gamma + N~{\rm jets})$
%at the Tevatron and 
at LHC. Parameters and cuts are given
in eqs.~(\ref{eq:gammaj1}-\ref{eq:gammaj2}).}
\end{center}
\end{table} }

\subsection{$\mathbf{n H+}$~jets}
\label{hjet}
The code computes all processes where, due to an {\em effective}
gluon-gluon-Higgs coupling, one on-shell Higgs 
is produced together with (up to 3) additional light jets.
Producing more jets is also possible, but, in this case,
we do not calculate processes with 3 light-quark pairs, meaning that
only gluons produce the extra jets beyond the third one.
The EW parameters are fixed by default using the option {\tt iewopt=3}
(see eq.~(\ref{eq:iew3})).

The effective gluon-gluon-Higgs coupling we consider 
is derived from the following effective Lagrangian
\begin{eqnarray}
{\cal L}_{eff} &=& - \frac{1}{4}\,c\, 
H G^a_{\mu \nu} G^{a\,\, \mu \nu}\,,~~
 c  ~=~ \frac{\alpha_s}{3 \pi v}\,,~~v~\simeq~246~{\rm GeV}\,,
\end{eqnarray}
obtained taking the limit $m_t \to \infty$ in the top quark loop 
connecting the Higgs boson and the gluons, and 
neglecting all electro-weak couplings.

The subprocesses considered are listed in Table~\ref{tab:hjets}.
\begin{table}
\begin{center}
\begin{tabular}{ll|ll|ll}
{\tt jproc} & subprocess & {\tt jproc} & subprocess & {\tt jproc} &
subprocess \\ 
1  &  $g g    \to   H$ 
&2 &  $q\qbar \to   H$ 
&3 &  $g q    \to q H$ 
\\
4  &  $q g    \to q H$ 
&5 &  $gg     \to q \qbar H$ 
&6 &  $qq     \to q q     H$ 
\\
 7 &  $q q'  \to  q q'      H$ 
&8 &  $q q'' \to  q q''     H$ 
&9 &  $q \qbar \to   q \qbar     H$ 
\\
 10 &  $q \qbar \to   q' \qbar'   H$ 
&11 &  $q \qbar \to   q'' \qbar'' H$ 
&12 &   $q \qbar'  \to   q \qbar' {\rm~or~} \qbar q' H$ 
\\
 13 &  $q \qbar'' \to   q \qbar''                   H$ 
&14 &  $g q  \to q q \qbar H$ 
&15 &  $q g  \to q q \qbar H$ 
\\
 16 &   $g q  \to q q' \qbar'   H$ 
&17 &  $q g  \to q q' \qbar'   H$ 
&18 &  $g q  \to q q'' \qbar'' H$ 
\\
19 & $q g  \to q q'' \qbar'' H$ 
& &
& &
\end{tabular}
\ccaption{}{\label{tab:hjets} Subprocesses included in the $H+$jets
  code. It is always understood that quarks $q$ and $q'$ belong to the same
iso-doublet, while $q$ and $q''$ belong to different iso-doublets.
  Additional final-state
  gluons are not explicitly indicated but are included in the
  code. 
  The details can be found in the subroutine {\tt selflav} of
  the file {\tt hjetlib/hjet.f}.}
\end{center}
\end{table}

As a default, the code generates kinematical configurations defined by
cuts applied to the following variables:
\begin{itemize}
\item $\pt^{\rm jet}$, $\eta^{\rm jet}$, $\Delta R_{jj} \; .$
\end{itemize}
Additional cuts can be supplied by the user in an appropriate
routine. 

In the code initialization phase, the user can select among 3
continuous choices for the parametrization of the factorization and
renormalization scale $Q$: a real input parameter ({\tt qfac}) allows
to vary the overall scale of $Q$, $Q={\tt qfac}\times Q_0$, while the
preferred functional form for $Q_0$ is selected through an integer
input parameter ({\tt iqopt}=0,1,2).  In more detail:
{\renewcommand{\arraystretch}{1.2}
\begin{center}
\begin{tabular}{l||l|l|l|l|}
{\tt iqopt} & 0 & 1 & 2 \\  \hline
$Q_0^2$ & 1 & $n\,m_H^2+\sum_{jets}\pt^2$ & ${\hat{s}}$ 
\end{tabular}
\end{center}
}
where the sum extends over the $\pt^2$ of all the final state jets.

Some benchmark results are given in
Table~\ref{tab:hjetxs}, using the following
input options
\ba \label{eq:hjet1}
        && \pt^{\rm jet}>20~\gev, \quad \vert \eta_j\vert < 2.5, \quad \Delta
        R_{jj} >0.7,
\\
  && m_H= 120~\gev \; .
\label{eq:hjet2}
\ea
{\renewcommand{\arraystretch}{1.2}
\begin{table}
\begin{center}
\begin{tabular}{||l|l|l|l|l||}\hline
  &  $N = 0$  & 
$N = 1$ & $N = 2$ & $N = 3$ \\ 
\hline
 LHC (pb)   & 21.16(4) & 12.03(1) & 4.97(1) & 1.74(1)\\ 
 FNAL (pb)  & 0.306(1) &0.124(1)  & 0.0295(1) & 0.0052(1)  \\
\hline
\end{tabular}            
\ccaption{}{\label{tab:hjetxs} $\sigma( H + N~{\rm jets})$
at the Tevatron and 
at the LHC. Parameters and cuts are given
in eqs.~(\ref{eq:hjet1}-\ref{eq:hjet2}).}
\end{center}
\end{table} }
% pippo

\subsection{Single-top}
\label{singletop}
Production of single-top processes.
The effects of finite $b$-quark masses are always taken into account, in the final 
state as well as in the initial state. 
The user can select among four main processes: 
\begin{itemize}
\item 1) $t(\bar t) \, q$, 
\item 2) $t(\bar t) {\bar b}(b)$,
\item 3) $t(\bar t)+W$,
\item 4) $t(\bar t) {\bar b}(b)+W$.
\end{itemize}
All four channels include $t\to b f\bar{f}'$, 
  $W\to f\bar{f}'$ decays and relative spin correlations. 
The subprocesses considered include all configurations with up to two 
quark currents (including the top contribution). 
The subprocesses related to each main process are listed in 
Table~\ref{tab:singlet}, following 
the notations employed in the code. 

As a default, the code generates kinematical configurations defined by
cuts applied to the following variables:
\begin{itemize}
\item $\pt^{\rm jet}$, $\eta^{\rm jet}$, $\Delta R_{jj},\Delta R_{jb}$
\item $\pt^{ b}$, $\eta^{ b}$, $\Delta R_{b\bbar} \; .$
\end{itemize}
The respective threshold values can be provided by the user at run
time. For process n. 4 ($t b W$ final states) the contribution from 
$t \tbar$ production and decay is suppressed by imposing a phase space cut 
on the $W$-$b$ invariant mass ($\vert M(Wb) - m_t| \geq 5$~GeV) 
at generation level.
Additional cuts can be supplied by the user in an appropriate
routine. 
The decay of top quarks in three fermions 
(in the narrow width approximation) is enforced.
With this option the decay of the top quarks takes into account 
all spin correlations among the decay products by means of exact 
matrix elements. More information on the selection of decay products can 
be found in Appendix~\ref{app:topdec}.

In the code initialization phase, the user can select 
among 4 choices for the parameterization of the 
factorization and renormalization scale $Q$. A real input
parameter ({\tt qfac}) allows to vary the overall scale of $Q$,
$Q={\tt qfac}\times Q_0$, while the preferred functional form for
$Q_0$ is selected through an integer input parameter ({\tt
iqopt}=0,1,2).  In more detail:
{\renewcommand{\arraystretch}{1.2}
\begin{center}
\begin{tabular}{l||l|l|l}
{\tt iqopt} & 0 & 1 & 2\\  \hline
$Q_0^2$ & 1 & $\sum \mTsq (+ \mW^2)$ & $\hat {s} $ \\
\end{tabular}
\end{center}}
where $\mT$ is the transverse mass defined as $\mTsq=m^2+\ptsq$,
the sum $\sum \mTsq $ extends to all final
state partons, and the $\mW$ term is present only for the processes 3
and 4.

\begin{table}
\begin{center}
\begin{tabular}{ll|ll|ll|ll}

 &      &         &           &             & \\

$t(\bar t)$ &+ jets   & $t(\bar t) \bbar(b)$ &+ jets        &    $t(\bar t) W $ &+ jets          &  $t(\bar t) \bbar(b) W$ &+ jets         \\
{\tt jproc} & subproc & {\tt jproc} & subproc & {\tt jproc} &
subproc &{\tt jproc} & subproc \\ 
1 & $b \qbar \to  t \qbar'$ & 1 & $q \qbar' \to  t \bbar$  & 1 & $g b \to t W^-$ & 1 & $g g \to t \bbar W^-$ \\ 
2 & $\bbar q \to \tbar q'$ & 2 & $\qbar q' \to \tbar b$  & 2 & $b g \to t W^-$ & 2 & $g g \to \tbar b W^+$\\ 
3 & $g b \to  t q \qbar'$ & 3 & $g q \to  t \bbar q'$ & 3 & $g \bbar \to \tbar W^+$ & 3 & $q \qbar \to t \bbar W^-$ \\
4 & $b g \to  t q \qbar'$ & 4 & $q g \to  t \bbar q'$ & 4 & $\bbar g \to \tbar W^+$ & 4 & $\qbar q \to t \bbar W^-$ \\
  &                     &   &                       & 5 & $q b \to t W^- q $ & 5 & $q \qbar \to \tbar b W^+$\\
  &                     &   &                       & 6 & $b q \to t W^- q $ & 6 & $\qbar q \to \tbar b W^+$ \\
  &                     &   &                       & 7 & $\qbar b \to t W^- \qbar $ &  7 & $g q \to t \bbar W^- q$ \\
  &                     &   &                       & 8 & $b \qbar \to t W^- \qbar $ &  8 & $q g \to t \bbar W^- q $ \\
  &                     &   &                       & 9 & $q \bbar \to \tbar W^+ q $ &  9 & $g \qbar \to t \bbar W^- \qbar $\\
  &                     &   &                       & 10 & $\bbar q \to \tbar W^+ q $ & 10 & $\qbar g \to t \bbar W^- \qbar $\\
  &                     &   &                       & 11 & $\qbar \bbar \to \tbar W^+ \qbar $ &  11 & $g q \to \tbar b W^+ q $\\
  &                     &   &                       & 12 & $\bbar \qbar \to \tbar W^+ \qbar $ &  12 & $q g \to \tbar b W^+ q $\\
  &                     &   &                       &    &                             
  &   13 & $g \qbar \to \tbar b W^+ \qbar $\\
  &                     &   &                       &    &                             
  &   14 & $\qbar g \to \tbar b W^+ \qbar $\\
  &                     &   &                       &    &                             
  &   15 & $g g \to t \bbar W^- q \qbar $\\
  &                     &   &                       &    &                             
  &   16 & $g g \to \tbar b W^+ q \qbar $\\

\end{tabular}
\ccaption{}{\label{tab:singlet} Subprocesses included in the single top + jets
  code. Additional final-state gluons are not explicitly 
  shown here but are included in the code if the requested light-jet
  multiplicity exceeds the number of indicated final-state partons.
  The details can be found in the subroutines {\tt selflav\# (\#=1,2,3,4)} of
  the file {\tt toplib/top.f}.}
\end{center}
\end{table}


Some benchmark results are given in Table~\ref{tab:singletxs}, obtained
for the following set of cuts and scale choice:
\ba \label{eq:singletcut1}
        && \mt=175~\gev, \mb=4.75~\gev, Q^2=\mtsq
\\
        && \pt^{\rm jet}>20~\gev, \quad \vert \eta_j\vert < 2.5, \quad \Delta
        R_{jj} >0.7
\\
        && \pt^{b}>20~\gev, \quad \vert \eta_b \vert < 2.5, \quad \Delta
        R_{b\bbar} >0.7, \Delta R_{bj} >0.7 \; .
\label{eq:singletcut2}
\ea
{\renewcommand{\arraystretch}{1.2}
\begin{table}
\begin{center}
\begin{tabular}{||l|l|l|l|l||}\hline
%& & & \\
Single top  & $t + q$  & 
$t + b$ & $t + W$ & $t + \bbar + W$  \\ 
\hline
%& & & \\ 
LHC (pb)  & 92.4(4) & 5.46(2) & 54.3(1) & 60.0(3)  \\ 
\hline
%& & & \\ 
FNAL (fb) & 1.317(6) & 0.571(2) & 0.0917(2) & 0.635(3)  \\ 
\hline
\end{tabular}            
\ccaption{}{\label{tab:singletxs} Single top cross sections 
at the Tevatron and 
at the LHC, with parameters and cuts given in
eqs.~(\ref{eq:singletcut1}-\ref{eq:singletcut2}).}
\end{center}
\end{table}}

\subsection{$\mathbf{W+}$$\mathbf{\gamma +}$~jets}
\label{sec:wphjets}
A final state with $N \ge 0$ jets, $0 \le m \le 2$ photons and 1 $W$
is generated. As in the previous cases, we use the notation $W$ 
as a short hand; what is
actually calculated is the matrix element for a lepton+neutrino final
state. All spin correlations and finite width effects are therefore
accounted for. The quoted cross sections refer to a single lepton
family; in the flavour assignment, the code selects by default an
electron.
The EW parameters are fixed by default using the option {\tt iewopt=3}
(see eq.~(\ref{eq:iew1})).
The real final state photons are allowed to couple to all particles, including
the $W$ and its decay products,
but only strong interacting particles are allowed as virtual states.
In addition, we work in the approximation of vanishing
Cabibbo angle.

 The subprocesses considered include all configurations
with up to 2 light-quark pairs and up to 2 photons; they are listed in
Table~\ref{tab:wphjets}, following the notation employed in the
code.
\begin{table}
\begin{center}
\begin{tabular}{ll|ll|ll}
{\tt jproc} & subprocess & {\tt jproc} & subprocess & {\tt jproc} &
subprocess \\ 
1 &  $q\qbar' \to W(m\gamma)  $ 
&2 &  $q g \to q' W(m\gamma)  $ 
&3 &  $g q \to q' W(m\gamma)  $ 
\\
4 &  $gg \to q \qbar' W(m\gamma)  $ 
&5 &  $q\qbar' \to W(m\gamma)  q'' \qbar'' $ 
&6 &  $qq'' \to W(m\gamma)  q' q'' $ 
\\
7 &  $q'' q \to W(m\gamma)  q' q'' $ 
&8 &  $q\qbar \to W(m\gamma)  q' \qbar'' $ 
&9 &  $q\qbar' \to W(m\gamma)  q \qbar $ 
\\
10 &  $\qbar' q\to W(m\gamma)  q \qbar $ 
&11 &  $q\qbar \to W(m\gamma)  q \qbar' $ 
&12 &  $q\qbar \to W(m\gamma)  q' \qbar $ 
\\
13 &  $q q \to W(m\gamma)  q q' $ 
&14 &  $q q' \to W(m\gamma)  q q $ 
&15 &  $q q' \to W(m\gamma)  q' q' $ 
\\
16 &  $q g \to W(m\gamma)  q' q''\qbar'' $ 
&17 &  $g q \to W(m\gamma)  q' q''\qbar'' $ 
&18 &  $q g \to W(m\gamma)  q q \qbar' $ 
\\
19 &  $q g \to W(m\gamma)  q' q \qbar $ 
&20 &  $g q \to W(m\gamma)  q q \qbar' $ 
&21 &  $g q \to W(m\gamma)  q' q \qbar $ 
\\
22 &  $q g \to W(m\gamma) q' q' \qbar' $ 
&23 &  $g q \to W(m\gamma) q' q' \qbar' $ 
&24 &  $g g \to W(m\gamma) q \qbar' q'' \qbar'' $ 
\\
25 &  $g g \to W(m\gamma) q \qbar q \qbar' $ 
& &
& &
\end{tabular}
\ccaption{}{\label{tab:wphjets} Subprocesses included in the
  $\mathbf{W+}$$\gamma+$~jets
  code.  The $m$ in the table stands for the number of photons.  
  Additional final-state gluons are not explicitly 
  shown here but are included in the code if the requested light-jet
  multiplicity ($N\le 6-m$) exceeds the number of 
  indicated final-state partons.
  For example, the subprocess {\tt jproc=1} in the case of 2 jet
  will correspond to the final state  $q\qbar' \to W(m\gamma) g g$.
  The details can be found in the subroutine {\tt selflav} of
  the file {\tt wphjetlib/wphjet.f}.}
\end{center}
\end{table}
For each process, the charge-conjugate ones are always understood.
The above list fully covers all the possible processes with up to 3
light jets. In the case of 4 extra
jets, we do not calculate processes with 3 light-quark pairs. Within
the uncertainties of the LO approximation, these can be safely
neglected~\cite{Berends:1991ax}.

As a default, the code generates kinematical configurations defined by
cuts applied to the following variables:
\begin{itemize}
\item $\pt^{\rm jet}$, $\eta^{\rm jet}$, $\Delta R_{jj}$
\item $\pt^{\rm \gamma}$, $\eta^{\rm \gamma}$, $\Delta R_{\gamma j}$ 
\item $\pt^{\rm \ell}$, $\eta^{\rm \ell}$, $\pt^{\rm \nu}$, $\Delta
  R_{\ell j}$, $\Delta R_{\ell \rm \gamma}  \; . $
\end{itemize}
The respective threshold values can be provided by the user at run
time. Additional cuts can be supplied by the user in an appropriate
routine. 
The choice of scale is as follows
{\renewcommand{\arraystretch}{1.2}
\begin{center}
\begin{tabular}{l||l|l|l|l|l|}
{\tt iqopt} & 0 & 1 & 2 & 3 & 4 \\  \hline
$Q_0^2$ & 1 & $m_W^2+ \sum \mTsq$ & $m_W^2$ & $m_W^2+\pt_W^2$ & $\sum \mTsq$ 
\end{tabular}
\end{center}
}
where $\mT$ is the transverse mass defined as $\mTsq=m^2+\ptsq$,
and the sum $\sum \mTsq $ extends to all final
state partons and photons (excluding the $W$ decay products).
The option  {\tt iqopt}=0 allows the user to freeze the scale to an
arbitrary value, which can be selected specifying the numerical value
of {\tt qfac}.   

Some benchmark results are given in Table~\ref{tab:wphjxs}, obtained
for the following set of cuts and scale choice:
\ba \label{eq:wphj1}
&& Q^2 = \mWsq + \ptWsq,
\\
        && \pt^{\rm jet}>20~\gev, \quad \vert \eta_j\vert < 2.5, \quad \Delta
        R_{jj} >0.7, 
\\
        && \pt^{\rm \gamma}>20~\gev, \quad \vert \eta_\gamma\vert 
< 2.5, \quad \Delta
        R_{\gamma j} >0.7, \quad \Delta
        R_{\ell \gamma} >0.7 \; .
\label{eq:wphj2}
\ea

{\renewcommand{\arraystretch}{1.2}
\begin{table}
\begin{center}
\begin{tabular}{||l|l|l|l|l|l||}\hline
  & & $N = 1$  & 
$N = 2$ & $N = 3$ & $N = 4$ \\ 
\hline
$1\gamma$ (pb)  & LHC       & 7.66(2)  & 3.28(1)  & 1.197(4) & 0.390(3) \\ 
                & FNAL  & 0.526(1) & 0.103(1) & 0.0173(1)& 0.00242(3) \\
\hline
  & & $N = 0$  & 
$N = 1$ & $N = 2$ & $N = 3$ \\ 
\hline
$2\gamma$ (fb)  &  LHC      & 8.71(4)  & 11.34(4)  & 7.27(3) & 3.26(3) \\ 
                &  FNAL     & 1.60(1)  & 0.526(2)  & 0.149(1)& 0.0296(3)\\
\hline
%FNAL (pb)  & 179.4(2) & 21.44(2) & 3.36(1) & 0.489(2) & 
%0.0630(3) & 0.00700(4)& 0.000690(6) \\ 
\hline
\end{tabular}            
\ccaption{}{\label{tab:wphjxs} $\sigma( W +\,1,2\gamma + N~{\rm jets})$
at the Tevatron and 
at the LHC. Parameters and cuts are given
in eqs.~(\ref{eq:wphj1}-\ref{eq:wphj2}).}
\end{center}
\end{table} }

\subsection{$\mathbf{W+}$$\gamma+$$\mathbf{Q\Qbar+}$~jets}
\label{sec:wphqq}
Code for the associated production of 1 $W$, $0 \le m \le 2$ photons, 
heavy quark ($Q=c,b$ or $t$) pairs, and jets.  
We use the notation $W$ as a short hand; what is
actually calculated is the matrix element for a fermion-antifermion final
state. All spin correlations and finite width effects are therefore
accounted for.  
The quoted cross sections refer to a single lepton
family; in the flavour assignment, the code selects by default an
electron. Different flavours can be selected during the unweighting
phase, covering all possible leptonic decays, as well as inclusive
quark decays (for more details see the Appendix~\ref{app:topdec}).
In the case $Q=t$, the top quark is left undecayed.
The EW parameters are fixed by default using the option {\tt iewopt=3}
(see eq.~(\ref{eq:iew1})).
The real final state photons are allowed to couple to all particles, including
the $W$ and its decay products,
but only strong interacting particles are allowed as virtual states.
In addition, we work in the approximation of vanishing
Cabibbo angle.

 The subprocesses considered include all configurations
with up to 2 light-quark pairs and 2 photons; they are listed in
Table~\ref{tab:wphqq}, following the notation employed in the
code. 
\begin{table}
\begin{center}
\begin{tabular}{ll|ll|ll}
{\tt jproc} & subprocess & {\tt jproc} & subprocess & {\tt jproc} &
subprocess \\ 
1 &  $q\qbar' \to W(m\gamma) Q\Qbar$ 
&2 &  $q g \to q' W(m\gamma) Q\Qbar$ 
&3 &  $g q \to q' W(m\gamma) Q\Qbar$ 
\\
4 &  $gg \to q \qbar' W(m\gamma) Q\Qbar$ 
&5 &  $q\qbar' \to W(m\gamma) Q \Qbar q'' \qbar'' $ 
&6 &  $qq'' \to W(m\gamma) Q \Qbar q' q'' $ 
\\
7 &  $q'' q \to W(m\gamma) Q \Qbar q' q'' $ 
&8 &  $q\qbar \to W(m\gamma) Q \Qbar q' \qbar'' $ 
&9 &  $q\qbar' \to W(m\gamma) Q \Qbar q \qbar $ 
\\
10 &  $\qbar' q\to W(m\gamma) Q \Qbar q \qbar $ 
&11 &  $q\qbar \to W(m\gamma) Q \Qbar q \qbar' $ 
&12 &  $q\qbar \to W(m\gamma) Q \Qbar q' \qbar $ 
\\
13 &  $q q \to W(m\gamma) Q \Qbar q q' $ 
&14 &  $q q' \to W(m\gamma) Q \Qbar q q $ 
&15 &  $q q' \to W(m\gamma) Q \Qbar q' q' $ 
\\
16 &  $q g \to W(m\gamma) Q \Qbar q' q''\qbar'' $ 
&17 &  $g q \to W(m\gamma) Q \Qbar q' q''\qbar'' $ 
&18 &  $q g \to W(m\gamma) Q \Qbar q q \qbar' $ 
\\
19 &  $q g \to W(m\gamma) Q \Qbar q' q \qbar $ 
&20 &  $g q \to W(m\gamma) Q \Qbar q q \qbar' $ 
&21 &  $g q \to W(m\gamma) Q \Qbar q' q \qbar $ 
\\
22 &  $q g \to W(m\gamma) Q \Qbar q' q' \qbar' $ 
&23 &  $g q \to W(m\gamma) Q \Qbar q' q' \qbar' $ 
&24 &  $g g \to W(m\gamma) Q \Qbar q \qbar' q'' \qbar'' $ 
\\
25 &  $g g \to W(m\gamma) Q \Qbar q \qbar q \qbar' $ 
& &
& &
\end{tabular}
\ccaption{}{\label{tab:wphqq} Subprocesses included in the 
  $W\gamma Q\Qbar+$jets
  code. The $m$ in the table stands for the number of photons.
  Additional final-state gluons are not explicitly 
  shown here but are included in the code if the requested light-jet
  multiplicity ($N\le 4-m$) exceeds the number of 
  indicated final-state partons.
  For example, the subprocess {\tt jproc=1} in the case of 2 light jets
  will correspond to the final state  $q\qbar' \to W(m\gamma) Q\Qbar g g$.
  The details can be found in the subroutine {\tt selflav} of
  the file {\tt wphqqlib/wphqq.f}.}
\end{center}
\end{table}
For all processes, the charge-conjugate ones are always understood.
The above list fully covers all the possible processes with up to 3
light jets in addition to the heavy quarks. In the case of 4 extra
jets, we do not calculate processes with 3 light-quark pairs. Within
the uncertainties of the LO approximation, these can be safely
neglected~\cite{Berends:1991ax}.

The selection of the flavour takes place at run time.
As a default, the code generates kinematical configurations defined by
cuts applied to the following variables (the cuts related to the heavy
quarks are only applied in the case of $b$, while top quarks are
always generated without cuts):
\begin{itemize}
\item $\pt^{\rm jet}$, $\eta^{\rm jet}$, $\Delta R_{jj}$
\item $\pt^{ b}$, $\eta^{ b}$, $\Delta R_{b\bbar}$ 
\item $\pt^{\rm \gamma}$, $\eta^{\rm \gamma}$, $\Delta R_{\gamma j}$
\item $\pt^{\rm \ell}$, $\eta^{\rm \ell}$, $\pt^{\rm \nu}$, $\Delta
  R_{\ell j}$ , $\Delta R_{\ell \rm \gamma} \; .$  
\end{itemize}
The  cut values can be provided by the user at run
time. Additional cuts can be supplied by the user in the 
routine {\tt usrcut} contained in the user file {\tt wphqqwork/wphqqusr.f}.

In the code initialization phase, 
the user can select among 4 continuous choices for the parametrization
of the factorization and renormalization scale $Q$: a real input
parameter ({\tt qfac}) allows to vary the overall scale of $Q$,
$Q={\tt qfac}\times Q_0$, while the preferred functional form for
$Q_0$ is selected through the integer input parameter {\tt
  iqopt}:
{\renewcommand{\arraystretch}{1.2}
\begin{center}
\begin{tabular}{l||l|l|l|l|l|}
{\tt iqopt} & 0 & 1 & 2 & 3 & 4 \\  \hline
$Q_0^2$ & 1 & $m_W^2+ \sum \mTsq$ & $m_W^2$ & $m_W^2+\pt_W^2$ & $\sum \mTsq$ 
\end{tabular}
\end{center}
}
where $\mT$ is the transverse mass defined as $\mTsq=m^2+\ptsq$,
and the sum $\sum \mTsq $ extends to all final
state partons (including the heavy quarks and the photons, 
excluding the $W$ decay products).
The option  {\tt iqopt}=0 allows the user to freeze the scale to an
arbitrary value, which can be selected specifying the numerical value
of {\tt qfac}.   
 
Some numerical benchmark results are given in
Table~\ref{tab:wphbbxs}. 
The following scale and cuts are used:
\ba \label{eq:wphqq1}
&& Q^2 = \mWsq + \ptWsq,
\\
        && \pt^{\rm jet}>20~\gev, \quad \vert \eta_j\vert < 2.5, \quad \Delta
        R_{jj} >0.7,
\\
        && \pt^{b}>20~\gev, \quad \vert \eta_b \vert < 2.5, \quad \Delta
        R_{b\bbar} >0.7, \quad \Delta R_{bj} >0.7,
\\
        && \pt^{\rm \gamma}>20~\gev, \quad \vert \eta_\gamma\vert 
< 2.5, \quad \Delta
        R_{\gamma j} >0.7, \quad \Delta
        R_{\ell \gamma} >0.7 \; .
\label{eq:wphqq2}
\ea
{\renewcommand{\arraystretch}{1.2}
\begin{table}
\begin{center}
\begin{tabular}{||l|l|l|l|l||}\hline
 & & $N = 0$  & 
$N = 1$ & $N = 2$ \\ 
\hline
$1\gamma$ (fb)  & LHC  & 7.53(2)  & 10.47(4)  & 7.00(5) \\ 
                & FNAL & 1.06(1)  & 0.287(1)  & 0.0613(5)\\
\hline
 & & $N = 0$  & 
$N = 1$ & $N = 2$ \\ 
\hline
$2\gamma$ (fb)  & LHC & 0.0186(1)  & 0.0272(2)  & 0.0202(2) \\ 
                & FNAL& 0.00172(1) & 0.000491(4)& 0.000109(1) \\
\hline
\hline
\end{tabular}
\ccaption{}{\label{tab:wphbbxs} $\sigma(1,2\gamma 
+ b \bbar \ell \nu + N~{\rm jets})$ at the Tevatron and
at the LHC. Parameters and cuts are given
in eqs.~(\ref{eq:wphqq1}-\ref{eq:wphqq2}). }
\end{center}
\end{table} }

\subsection{$\mathbf{ Q\Qbar + m \gamma + N}$ jets}
\label{sec:2Qph}
Production of heavy quark pairs plus photons plus jets ($m + N \leq 6$). 
Here $Q$ can be either $c$, $b$, or $t$. The tree-level matrix elements 
are calculated at ${\cal O}(\alpha_s^N \alpha^m)$. 
In all cases, 
additional production of $Q\bar{Q}$ pairs
of $Q$ in the initial state are excluded. For the case of $Q=t$ no 
radiation from decay products is allowed. 
The subprocesses considered include all configurations with up to two
light quark pairs, where ``light'' quarks are those with mass lower
than $Q$. They are listed in Table~\ref{tab:2Qph}, following 
the notations employed in the code. The list covers all possible 
processes involving up to 4 light-parton jets with 2 light quark pairs.
Subprocesses involving 3 light-quark pairs, which would only appear in
the case of 4 jets in addition to the heavy quarks, are not included,
as they are expected to contribute a negligible rate.

As a default, the code generates kinematical configurations defined by
cuts applied to the following variables (the cuts related to the heavy
quarks are only applied in the case of $b$, while top quarks are
always generated without cuts):
\begin{itemize}
\item $\pt^{\rm jet}$, $\eta^{\rm jet}$, $\Delta R_{jj},\Delta R_{jb}$
\item $\pt^{ b}$, $\eta^{ b}$, $\Delta R_{b\bbar} \; .$
\item $\pt^{\gamma}$, $\eta^\gamma$, $\Delta R_{\gamma \gamma}$, 
      $\Delta R_{\gamma {\rm jet}}$, $\Delta R_{\gamma b}$
\end{itemize}
The respective threshold values can be provided by the user at run
time. Additional cuts can be supplied by the user in an appropriate
routine. In the case $Q=t$ the decay of $t \bar t$ pairs in six fermions 
(in the on-shell approximation) is enforced.
With this option the decay of the top quarks takes into account 
all spin correlations among the decay products by means of exact 
matrix elements. More information on the selection of decay products can 
be found in Appendix~\ref{app:topdec}.

In the code initialization phase, the user can select 
between 3 choices for the parameterization of the 
factorization and renormalization scale $Q$. A real input
parameter ({\tt qfac}) allows to vary the overall scale of $Q$,
$Q={\tt qfac}\times Q_0$, while the preferred functional form for
$Q_0$ is selected through an integer input parameter ({\tt
iqopt}=0,1,2).  In more detail:
{\renewcommand{\arraystretch}{1.2}
\begin{center}
\begin{tabular}{l||l|l|l}
{\tt iqopt} & 0 & 1 & 2\\  \hline
$Q_0^2$ & 1 & $\sum \mTsq$ & $\hat{s}$ \\
\end{tabular}
\end{center}}
where $\mT$ is the transverse mass defined as $\mTsq=m^2+\ptsq$,
$m$ is the heavy quark mass, and the sum $\sum \mTsq $ extends to all final
state QCD partons (including the heavy quarks).

\begin{table}
\begin{center}
\begin{tabular}{ll|ll|ll}
{\tt jproc} & subprocess & {\tt jproc} & subprocess & {\tt jproc} &
subprocess \\ 
1 &  $g g \to  Q\Qbar (m\gamma)$ 
&2 &  $q \qbar \to Q\Qbar (m\gamma)$ 
&3 &  $g q \to  Q\Qbar (m\gamma) q$ 
\\
4 &  $ q g \to Q\Qbar (m\gamma) q$ 
&5 &  $g g  \to  Q \Qbar (m\gamma) q \qbar$ 
&6 &  $q \qbar  \to Q \Qbar (m\gamma) q \qbar$ 
\\
7 &  $q \qbar \to Q \Qbar (m\gamma) q' \qbar'$ 
&8 &  $q q' \to Q \Qbar (m\gamma) q q'$ 
&9 &  $q\qbar' \to  Q \Qbar (m\gamma) q \qbar'$ 
\\
10 &  $q q\to Q \Qbar (m\gamma) q q$ 
&11 &  $g q \to Q \Qbar (m\gamma) q q'\qbar'$ 
&12 &  $q g  \to  Q \Qbar (m\gamma) q q' \qbar'$ 
\\
13 &  $g q \to  Q \Qbar (m\gamma) q q \qbar$ 
&14 &  $q g \to Q \Qbar (m\gamma) q q \qbar $ 
&15 &  $g g \to  Q \Qbar (m\gamma) q \qbar q \qbar$ 
\\
16 &  $g g \to  Q \Qbar (m\gamma) q \qbar q' \qbar'$ 
& &  
& &
\end{tabular}
\ccaption{}{\label{tab:2Qph} Subprocesses included in the $Q\Qbar+m\gamma+$jets
  code. Additional final-state gluons are not explicitly 
  shown here but are included in the code if the requested light-jet
  multiplicity ($m+N\le 6$) exceeds the number of indicated final-state 
  partons. For example, the subprocess {\tt jproc=1},
  in the case of 2 extra jets, will correspond to the final state  
  $gg\to Q\Qbar m\gamma g g$.
  The details can be found in the subroutine {\tt selflav} of
  the file {\tt 2Qphlib/2Qph.f}. For each process, the charge-conjugate 
  ones are always understood.}
\end{center}
\end{table}


Some benchmark results are given in Tables~\ref{tab:QQphxs} and 
\ref{tab:QQ2phxs}, 
obtained
for the following set of cuts and scale choice:
\ba \label{eq:2Qphcut1}
t\tbar: && \mt=174.3~\gev, Q^2=2\mt^2 + \pttsq+\pttbsq + \sum \ptsq_j
\\
b\bbar: && \mb=4.7~\gev, Q^2=2\mb^2 + \ptbsq+\ptbbsq+\sum \ptsq_j
\\
        && \pt^{\rm jet}>20~\gev, \quad \vert \eta_j\vert < 2.5, \quad \Delta
        R_{jj} >0.7
\\
        && \pt^{\gamma}>20~\gev, \quad \vert \eta_\gamma\vert < 2.5, 
           \quad \Delta R_{\gamma j} > 0.7, \, \, 
                 \Delta R_{\gamma \gamma} > 0.7, \, \, 
                 \Delta R_{\gamma b} >0.7
\\
        && \pt^{b}>20~\gev, \quad \vert \eta_b \vert < 2.5, \quad \Delta
        R_{b\bbar} >0.7, \, \, \Delta R_{bj} >0.7 \; .
\label{eq:2Qphcut2}
\ea
{\renewcommand{\arraystretch}{1.2}
\begin{table}
\begin{center}
\begin{tabular}{||l|l|l|l|l|l||}\hline
%& & & \\
$Q \Qbar +\gamma + N~{\rm jets}$  & $N= 0$  & 
$N = 1$ & $N = 2$ & $N = 3$ & $N=4$ \\ 
\hline
$Q=t$, LHC (pb)  & 1.312(8) & 1.071(6) & 0.534(4) & 0.201(2)  & 0.621(9)\\ 
\hline
$Q=b$, LHC (pb) &  170.0(3) & 102.1(3) & 32.0(1) & 8.17(4) & 1.82(2)  \\ 
\hline
\end{tabular}            
\ccaption{}{\label{tab:QQphxs} $\sigma(Q \Qbar + \gamma + N~{\rm jets})$
at the LHC, with parameters and cuts given in 
Eqs.~(\ref{eq:2Qphcut1}-\ref{eq:2Qphcut2}).}
\end{center}
\end{table}}

{\renewcommand{\arraystretch}{1.2}
\begin{table}
\begin{center}
\begin{tabular}{||l|l|l|l|l|l||}\hline
%& & & \\
$Q \Qbar + 2 \gamma + N~{\rm jets}$  & $N= 0$  & 
$N = 1$ & $N = 2$ & $N = 3$ & $N=4$ \\ 
\hline
$Q=t$, LHC (fb)  & 8.36(6) & 6.55(5) & 3.30(4) & 1.25(1)  & 0.39(2) \\ 
\hline
$Q=b$, LHC (fb) &  121.8(3) & 107.4(5) & 46.4(3) & 14.7(1) & 3.69(8)  \\ 
\hline
\end{tabular}            
\ccaption{}{\label{tab:QQ2phxs} $\sigma(Q \Qbar + 2 \gamma + N~{\rm jets})$
at the LHC, with parameters and cuts given in 
Eqs.~(\ref{eq:2Qphcut1}-\ref{eq:2Qphcut2}).}
\end{center}
\end{table}}



\section{Conclusions}
\label{sec:concl}
We presented in this paper a new MC tool for the generation of
complex, high-multiplicity hard final states in hadronic collisions.
To the best of our knowledge, a large fraction of the processes we
discussed have never been calculated before in the literature to the
level of jet multiplicities considered here, due to the complexity of
the matrix elements involved. In addition to the evaluation of the
matrix elements, and the possibility of performing complete
parton-level simulations, the code we developed offers the possibility
to carry out the shower evolution and hadronization of the partonic
final states. In the current version we implemented the Les Houches
format for the event representation, and developed the relative
interface with \herwig.  In the future other hard processes (for
example including emission of real, hard photons) will be added to the
list of available reactions and Higgs decay to two or four fermions
will be included.

Our code will allow complete and accurate studies of the SM
backgrounds to a large fraction of the most interesting new physics
phenomena accessible at the Tevatron, at the LHC, and at future
high-energy hadron colliders.

\section*{Acknowledgements}
We thank P.~Richardson and B.~Webber for invaluable help in the
development of the Les Houches compliant \herwig\ interface for
\herwig, and A.~Messina, T.~Sj\"ostrand, S.~Mrenna, F.~Ambroglini and
S.~Cucciarelli for their contribution to the \pythia\ interface.
FP thanks G. Montagna and O. Nicrosini for their collaboration during
the early stage of the development of the $Q\Qbar Q'\Qbar'$ processes,
and the Pavia Gruppo IV of INFN for access to the local computing resources.

\begin{appendix}
\section{The contents of the code package}
The code is written in Fortran77, with the part relative to the matrix
  element evaluation available as well in Fortran 90 (see
  Appendix~\ref{sec:f90}).
The code package, contained in the compressed file {\tt alpgen.tar.gz}, 
can be obtained from the URL
{\tt  http://home.cern.ch/mlm/alpgen}.

Unpacking the zipped tarred file {\tt alpgen.tar.gz} with the command:

\begin{verbatim}
> tar -zxvf alpgen.tar.gz
\end{verbatim}

will create the following directory structure:

\begin{verbatim}
2Qlib/    DOCS/     alplib/      hjetwork/  topwork/    wjetlib/    wqqlib/
2Qphlib/  Makefile  compare      phjetlib/  validation/ wjetwork/   wqqwork/
2Qphwork/ Njetlib/  compile.mk   phjetwork/ vbjetlib/   wphjetlib/  zjetlib/
2Qwork/   Njetwork/ ft90V.tar.gz prc.list   vbjetwork/  wphjetwork/ zjetwork/
4Qlib/    QQhlib/   herlib/      pylib/     wcjetlib/   wphqqlib/   zqqlib/
4Qwork/   QQhwork/  hjetlib/     toplib/    wcjetwork/  wphqqwork/  zqqwork/
\end{verbatim}                                                  



More in detail:
\begin{itemize}
\item The directory {\tt alplib/} contains the parts of code which are
  generic to the evaluation of matrix elements using the \ALPHA\ 
  algorithm.  The user should treat this directory as a black box.
  When new processes will be added in the future, this part of the
  code should not change.  More in detail:
  \begin{itemize}
  \item {\tt alplib/alpgen.f}: contains the general structure of the
    code, preparing the input for the matrix element calculation, the
    bookkeeping of the cross-section determination, the event
    generation, etc.
  \item {\tt alplib/alpgen.inc}: include file, with the necessary common
    blocks.
  \item {\tt alplib/Aint.f, Asu3.f, Acp.f}: the set of programmes
    necessary for the calculations of the matrix element, done by the
    \ALPHA\  algorithm.
  \item {\tt alplib/alppdf.f}: contains a collection of structure
    function parameterizations; some of them require at run time input
    tables, which are provided as part of the package, and stored in
    the subdirectory {\tt alplib/pdfdat/}. The command file {\tt
      alplib/pdfdat/hvqpdf} contains the necessary logical links to
    all PDF data tables. As a default, we already provide a logical
    link to this file in all {\tt /*work} directories. It is
    sufficient to issue the {\tt pdflnk} command within the desired
    working subdirectory to create the necessary logical links, and
    allow the use of all available PDFs.
  \item {\tt alplib/alputi.f}: This program unit contains a
    histogramming package which allows to generate
    {\tt topdrawer}~\cite{topdrawer}  files with
    the required distributions. Examples of the use of this package
    are provided in the default user files {\tt *work/*usr.f}.  
    Users who prefer other histogramming packages, such as HBOOK, do
    not need to link to this file.
  \end{itemize}
  
\item The directories {\tt *lib/} ({\tt *=wqq, zqq, wcjet, wjet, zjet, hjet,
    vbjet, 2Q, 4Q, QQh, Njet, phjet, top, wphjet, wphqq, 2Qph}) 
      contain the parts of the code
    specific to the generation of $WQ\Qbar+$~jets, $ZQ\Qbar+$~jets,
    $W+$~jets, $W+c+$~jets, $Z+$~jets, $nH+$~jets, $nW+mZ+lH+j\gamma+$~jets,
    $Q\Qbar+$~jets, $Q\Qbar Q'\Qbar'+$~jets $Q\Qbar H+$~jets, $N$~jets,
    $\gamma+$~jets, single-$t+$~jets,  $W+\gamma+$~jets,  
    $W+\gamma+Q\Qbar+$~jets and $Q\Qbar+\gamma+$~jets events.  
    The respective {\tt include} files with
    the necessary process-dependent common blocks are included in
    these directories.  The user should treat these directories as
    black boxes.
  
\item The directories {\tt *work/} ({\tt *=wqq, zqq,} etc.)  contain
    the parts of the code that the user is supposed to interact with,
    in order to implement his own analysis cuts, etc.  They contain
    the files {\tt *usr.f} ({\tt *=wqq, zqq,} etc.), where sample
    analysis routines are provided. These files host the routines in
    which the user can select generation mode, generation parameters
    (e.g. beam energy, PDF sets, heavy quark mass, etc.) as well as
    generation cuts (minimum \pt\ thresholds, etc.). Here the user
    initialises the histograms, writes the analysis routine, and
    prints out the required program output.  This is the only part of
    the code in which the user is supposed to operate, editing the
    analyses files, and producing and running the executable (see next
    Section). The versions provided as a default contain already
    complete running examples, with the respective command files ({\tt
    input}) containing sets of default settings (see next Section for
    more details on running the code).
    
\item The directory {\tt herlib/} contains the parts of code relevant
  for the shower evolution using \herwig. In addition to the \herwig\ 
  source and include files for version 6.510, this directory includes
  the file {\tt atoher.f}, which is the interface between the
  parton-level matrix elements and \herwig, and the file {\tt
    hwuser.f}, which includes the main driver for the running of \herwig,
  and the part of the code where the user can input the analyses
  routines. More in detail:
  \begin{itemize}
  \item {\tt hwuser.f}: user initialization of the analysis. Includes
    the standard \herwig\ initialization, histogram intialization,
    analysis routines, etc.  The calls to new routines {\tt hwigup}
    and {\tt hwupro} create the interface with the generated hard
    events. 
  \item {\tt atoher.f}: this file contains all routines necessary to
    read in the unweighted events produced by the hard matrix element
    generator.  The routine {\tt hwigup} downloads the initialization
    parameters of the hard process (process type, number of partons,
    beam energy and beam type, etc.), and allows the main \herwig\
    initialization. The routine {\tt hwupro} is called for each event:
    it reads the event kinematics, flavour and colour information from
    the file of unweighted events, and translates the event data to
    allow the \herwig\ processing of the shower.  This file should be
    treated by the user as a black box.
  \item {\tt herwig6510.f}: the \herwig\ source code. 
  \item {\tt HERWIG65.INC, herwig6510.inc}: \herwig\  common blocks
  \item {\tt pdfdummy.f}: dummy PDF routine, required by \herwig\ 
    unless the CERN library PDF sets are used. As a default, the
    current version runs with the default \herwig\ PDF set, regardless
    of the PDF set which was used to generate the hard process. We
    verified that this does not affect the features of the showered
    final state.  Nevertheless we plan in the next version to enforce
    the consistency between PDF set used in the hard generation and in
    the shower evolution.
  \end{itemize} 
\item A similar directory and file structure is provided for \pythia.
\item The  directory {\tt DOCS/} contains this document and its
source. 
\item The file {\tt Makefile} allows to work with the directory
structures, including for example commands to allow the packing of the
full code into a single tarred file, and the validation of the installation
(see later for details).
\end{itemize} 

\section{Running the code}
To compile the code for the $WQ\Qbar$ process\footnote{Analogous
  procedures allow compilation and run of other processes.}, change
directory to:

{\tt $>$ cd wqqwork}

A {\tt Makefile} is provided for compilation. Issue the
comand\footnote{This command applies starting from version 2.1. Before
  one needed {\tt $>$ make *gen}, with *={\tt wqq,} etc.}

{\tt $>$ make gen}

and the executable {\tt wqqgen} will be prepared.  If the user
  wants to rename the analysis file {\tt wqqusr.f} to, say, {\tt
    myanal.f}, the executable can be obtained adding the following
  optional statement to the {\tt make} command:

{\tt $>$ make usrfile=myanal gen}

The executable will then be called simply  {\tt myanal}. 
The executables can be run
interactively, inputting from the keyboard the run parameters
requested by the code, or using
the default command file {\tt input}, issuing the command 

{\tt $>$ wqqgen < input}

Editing the file {\tt input} allows to change the initialization
defaults (e.g. the number of jets, the heavy quark masses, the PDF
sets, etc.). The general structure of the input is discussed later in
section~\ref{sec:v20}. 

The first input parameter requested is the running mode {\tt
  imode}. The three available running modes are discussed in detail in
the following subsections. 



\subsection{{\tt imode=0}}
The simplest option is {\tt imode=0}, where events are generated
according to the selected cuts, a total cross section is evaluated,
and the user can use the routine {\tt evtana} to analyse the event and
fill histograms with the desired distributions. To facilitate the job
of the user, we provide a (redundant) array of kinematical variables
relative to the event. The array is initialised in the routine {\tt
  usrfll} contained in the file {\tt wqqlib/wqq.f}, and is 
stored in the common block 
{\tt usrevt} contained in the include file {\tt wqq.inc}. Examples of
variables provided include:
\begin{itemize}
\item {\tt pin(4,2):} momenta of the incoming partons
\item {\tt pout(4,8):} momenta of the outgoing particles (maximum 8
  outgoing partons)
\item {\tt pjet(4,8):} momenta of the final-state partons
  (i.e. quarks and gluons)
\item {\tt ptj(8), etaj(8):} transverse momentum and pseudorapidity
  of the final-state partons
\item {\tt pbott(4) (pbbar(4)), ptb (ptbb):} momentum and transverse
  momentum of the heavy (anti)quark 
\item {\tt plep(4) (ptlep), pnu(4) (ptmiss):} (transverse) momentum 
 of the charged lepton and neutrino
\item {\tt drjj(8,8):}  $\Delta R$ separation in $\eta-\phi$ space
  among the final-state partons
\item {\tt drbj(8) (drbbj(8)) :}  $\Delta R$ separation in $\eta-\phi$ space
  between the heavy (anti)quark, and the final-state partons
\item etc\dots
\end{itemize}
Similar sets of variables are provided for the other processes.  As an
output the user will find the following files: ({\tt `file'} is the
label assigned by the user at run-start time):
\begin{itemize}
\item {\tt file.stat}: the header of this file contains information on
  the run: value of the input parameters (EW and strong couplings,
  beam types and energies, PDF set), hard process selected and
  generation cuts. Furthermore, this file reports the results of each
  individual integration cycle, with total cross sections, as well as
  individual contributions from the allowed subprocesses. It gives the
  maximum weights of the various iterations, and the corresponding
  unweighting efficiencies, and the value of the cross-section
  accumulated over the various iterations, weighted by the respective
  statistical errors.
\item {\tt file.top}: includes the topdrawer plots of the
  distributions, if requested. The default normalization of the
  histograms is in pb/bin.
\item {\tt cnfg.dat}: file required by ALPHA, generated at run time;
         it is not needed for the
         analysis of the output, and will be recreated anew any time
         the code runs, so the user should not bother about it, and
         it can be safely deleted at any time.
\item {\tt file.mon}: produced/updated after each 100K events;
       it  contains information on the status of the
         run, dumped  every 100K events. It is useful to monitor the
         progress  of the run. In addition to this monitoring tool,
         the user can choose to perform other tests during the run, in
         order to save partial information, or monitor the evolution
         of the plotted distributions. Every 100K events the program
         calls the routine {\tt monitor}, contained in the user file {\tt
           wqqwork/wqqusr.f}, where the user can select which
         operations to perform. As a default, the provided routine
         prints out each 1M events the topdrawer file with the
         distributions being histogrammed. In case of crash, the
         results relative to the statistics accumulated up to that
         point  are therefore retrievable.
\item {\tt file.grid*}: The phase-space is discretised and
         parameterised by a multi-dimensional grid. During the
         phase-space integration, a record is kept of the rate
         accumulated within each bin of each integration variable. At
         the end of an integration cycle (``iteration''), the
         total bin-by-bin rates are used to improve the grid
         sampling efficiency. This is achieved by assigning sampling
         probabilities proportional to the bin integrals (we ensure
         however that 20\% of the sampling is uniformly distributed
         among all bins, to avoid artificial biases introduced by runs
         with limited statistics). A subsequent iteration can then
         benefit from a better sampling. The state of the grid at the
         end of each iteration is saved in the file {\tt file.grid1}.
         Since the first few
         iterations give rise to distributions which are likely to be
         biased by large statistical fluctuations, we separate a phase
         of grid warm-up from a phase in which events will be
         generated and distributions calculated. The user should then
         specify in the {\tt input} file the number of warm-up
         iterations, the number of events to be calculated for each
         iteration, and then the number of events that will be used
         for the final event generation and for the analysis. At the
         end of the event generation, the grid will be saved to the
         file {\tt file.grid2}. We keep this separate from {\tt
           file.grid1} to allow the user to choose whether to start a
         new generation cycle
         using the grid status at the end of the previous warm-up
         phase, or at the end of the previous generation phase.
         These choices are made by the user in the {\tt input} file,
         selecting the variable {\tt igrid} to be 0 (to reset the
         grids and start a new grid optimization), 1 (to start the new
         run using the grid obtained during the previous warm-up
         phase) or 2 (to start the new run using the grid optimised at
         the end of the latest event generation). Both grid files are
         saved in {\tt file.grid*-old} (*=1,2) at the beginning
         of each run, to allow recovery of the grid information in
         case of run crashes or mistakes.
\end{itemize}
\subsection{{\tt imode=1}}
Running the code with {\tt imode=1} offers the same functionality than
{\tt imode=0}, but will in addition write the weighted events to a
file. To limit the size of the file, only events which passed a
pre-unweighting are saved. The pre-unweighting is based on a maximum
weight $w_{tmp}$ which is equal to 1\% of the actual maximum weight at
the moment of the generation of the event: $w_{tmp}=w_{max}/100$.  An
event with weight $w$ passing the pre-unweighting is then assigned a
weight $w'=w_{tmp}$ if $w<w_{tmp}$, or $w'=w$ if $w>w_{tmp}$. The
weight $w'$ is then saved to a file, together with the random number
seed which initiated the generation of this event, and with the value
of $x_1$ (useful to check the sanity of the file when it will be read
again for the unweighting).  Some statistical information on the run,
including the total number of generated events, the integral, and the
overall maximum weight, are saved as well in a separate file. The file
with weighted events is to be used for a later unweighting. One can
easily verify that the pre-unweighting procedure does not introduce
any bias in the final unweighting. The random number seed will then be
sufficient to regenerate the full kinematical, flavour and colour
information on the event. The size of each event is 57~bytes. Make sure
you have enough disk space to write out the number of events you
require.  In addition to the files listed above for {\tt imode=0}, as
an output the user will find the following files:
\begin{itemize}
\item {\tt file.par}: includes run parameters (e.g. beam energies and types,
  generation cuts, etc), phase-space grids,  cross-section and maximum-weight
  information;
\item {\tt file.wgt}: for each event we store the two seeds of the
  random number generation, the event weight, and the value of $x_1$
  for the event (as a sanity benchmark after the kinematics has been
  reconstructed from the random seeds)
\end{itemize}
As a default, the bookkeeping of the weight distribution is kept in
the routine {\tt *usr.f}, and the relative data are printed in the
topdrawer file {\tt file.top}. The study of the weight distribution
can guide the user to a more efficient choice of maximum weight before
starting the event unweighting.
\subsection{{\tt imode=2}}
After a run with {\tt imode=1}, a run with {\tt imode=2} will perform
the unweighting of the already generated events, and will prepare the
input file for the processing of the events with \herwig~or~\pythia.  
The code
reads first the phase-space grids used for the weighted-event
generation and
 the maximum weight from {\tt file.par}. The user has the possibility
 to edit {\tt file.par} and replace the maximum weight with a
 different value, if he convinces himself that a more efficient
 unweighting can be obtained, without biasing the sample, by selecting a
 smaller maximum weight\footnote{The user must however avoid
  using a maximum weight smaller than 1\% of
  the true maximum weight, because of the threshold used in 
the pre-unweighting phase.}. We are working on techniques to perform
this optimised unweigthing in an automatic way. The code will then scan the
file {\tt file.wgt}, containing the weighted events. A comparison of
the event weights with the maximum weight is made, and the unweighting
is performed. The kinematics of each unweighted event is reconstructed
from the relative random number seed. The colour flow for the event is
then calculated, and the full event information is written to a new
file. This file will be the starting point for the generation of the
full shower, to be performed using \herwig\ or \pythia.  As an output the user will
find the following files:
\begin{itemize}
\item {\tt file\_unw.stat}: includes cross sections, max weight, etc;
\item {\tt file\_unw.top}: While no new events are generated, the analysis
  routines used when running in {\tt imode=1} are applied to the
  unweighted events, and the relative distributions are evaluated. In
  this way the user can compare distributions before and after unweighting;
\item {\tt file.unw}: list of unweighted events, including event
  kinematics, flavour and colour structure, and event weight.
\end{itemize}

\subsection{Decay of top quarks and vector bosons with spin correlations}
\label{app:topdec}
The on-shell top quarks generated by {\tt 2Q}, {\tt QQh}, {\tt top}
and {\tt 2Qph} undergo 
a fully exclusive decay in three fermions weighted with exact matrix 
element.
When running in {\tt imode=0,1} the information on top decay product 
momenta is stored in the matrix {\tt idec(4,3,2)}, available in 
the routines {\tt usrfll} and {\tt usrcut}. The meaning of the entries 
is as follows: {\tt idec(1:4,i,j)} is the four momentum ($p_x, p_y,
p_z, E$) 
of the i-th decay product of the j-th particle ({\tt jtl} and {\tt jtbl} 
are the labels for top and antitop quark respectively); 
{\tt i=1} is the label for the $b$ ($\bar{b}$) quark; {\tt i=2,3} 
are the labels for the fermion and antifermion coming from the $W$ decay 
respectively. When running in {\tt imode=2} the user is required
(interactively) to 
select one top decay mode among seven options, in the case of {\tt 2Q}, {\tt QQh} and {\tt top} (only with final states containing an extra $W$): 
\begin{eqnarray}
{\tt 1} &=& e \nu_e b \bar b + 2\, {\rm jets},  \nonumber \\ 
{\tt 2} &=& \mu \nu_\mu b \bar b + 2\, {\rm jets}, \nonumber \\
{\tt 3} &=& \tau \nu_\tau b \bar b + 2\, {\rm jets},  \nonumber \\
{\tt 4} &=& l \nu_l b \bar b + 2\, {\rm jets},\, \, 
(l = e, \mu, \tau) \nonumber  \\
{\tt 5} &=& l \nu_l l' \nu_{l'} b \bar b, (l = e, \mu, \tau) \nonumber  \\
{\tt 6} &=& b \bar b + 4\,{\rm jets}, \nonumber \\
{\tt 7} &=& {\rm fully} \,\, {\rm inclusive}, \nonumber 
\end{eqnarray}
and among six options, in the case of {\tt top}: 
\begin{eqnarray}
{\tt 1} &=& e \nu_e b,  \nonumber \\ 
{\tt 2} &=& \mu \nu_\mu b, \nonumber \\
{\tt 3} &=& \tau \nu_\tau b,  \nonumber \\
{\tt 4} &=& l \nu_l b \, \, (l = e, \mu, \tau), \nonumber  \\
{\tt 5} &=& b + 2\,{\rm jets}, \nonumber \\
{\tt 6} &=& {\rm fully} \,\, {\rm inclusive}. \nonumber 
\end{eqnarray}
Non-zero  masses of the $W$-decay products are introduced by rescaling 
their momenta, and keeping invariant the $W$ 4-momentum.

The on-shell $W$ and $Z$ gauge bosons generated in {\tt vbjets} undergo 
a fully exclusive decay. 
While the decay products of the $W$ can be changed when running in 
{\tt imode=2} (see below), given the universal electroweak couplings of 
the $W$ to 
fermions, the $Z$ decay options must be specified at the very beginning. 
When running in {\tt imode=0,1} the information on the $W$, $Z$ decay product 
momenta is stored in the matrix {\tt idec(4,4,maxpar-2)}, available in 
the routines {\tt usrfll} and {\tt usrcut}. The meaning of the entries 
is as follows: {\tt idec(1:4,i,j)} is the four momentum ($p_x, p_y,
p_z, E$) 
of the i-th decay product of the j-th particle; {\tt i=1,2} are the labels 
for fermion and antifermion respectively. The ordering in {\tt j}
correspond to the ordering in which $Z$ and $W$'s are generated.
 The flavour of the $Z$ decay products is stored 
in the variable {\tt zfl(maxpar-2)} according to PDG conventions. 
For every $Z$ boson in the final state the user have to select its decay 
mode in the input file by entering an integer string with the decay modes 
of the individual $Z$'s according to the following table
\begin{eqnarray}
{\tt 1} &=& \nu {\bar \nu}, ({\rm summed}\, \, {\rm over }\, \, {\rm all}\, \, 
{\rm flavours}), \nonumber \\
{\tt 2} &=& l^+ l^-, ({\rm summed}\, \, {\rm over }\, \, {\rm all}\, \, 
{\rm flavours}), \nonumber \\
{\tt 3} &=& q {\bar q}, ({\rm summed}\, \, {\rm over }\, \, {\rm all}\, \,
{\rm flavours}), \nonumber \\
{\tt 4} &=& b {\bar b}, \nonumber \\
{\tt 5} &=& {\rm fully} \,\, {\rm inclusive}. \nonumber 
\end{eqnarray}
Concerning the decay modes of the $W$'s, they have to be specified 
when running in {\tt imode=2} as follows (the code asks for this
automatically and interactively):
\begin{eqnarray}
{\tt 1} &=& e {\bar \nu_e}, \nonumber \\
{\tt 2} &=& \mu {\bar \nu_\mu}, \nonumber \\
{\tt 3} &=& \tau {\bar \nu_\tau}, \nonumber \\
{\tt 4} &=& l {\bar \nu}_l (l = e, \mu, \tau),  \nonumber  \\
{\tt 5} &=& q {\bar q}',  \nonumber  \\
{\tt 6} &=& {\rm fully} \,\, {\rm inclusive}. \nonumber 
\end{eqnarray}
The same final state options for $W$ bosons are available also 
for {\tt wjets} and {\tt wqq} with {\tt imode=2}.
Finite fermionic masses of the decay products are introduced by rescaling 
the momenta and preserving the vector boson 4-momentum.



\subsection{Running \herwig\ or \pythia\ on the unweighted events}
A \herwig\ executable can be obtained starting
from the default driver {\tt herlib/hwuser.f}. To compile and link,
issue the command:

{\tt $>$ make hwuser}

from the {\tt herlib} directory. The resulting executable, {\tt
  hwuser}, should then be run in the directory containing the
unweighted event file. The default code simply generates the shower
evolution for the given events, and prints to the screen the particle
content of the first few events (number of printed events
set by the variable {\tt maxpr}). Analysis code should be provided by
the user, by filling the standard \herwig\ routines {\tt hwabeg,
  hwanal, hwaend}. A log file {\tt file-her.log} documenting the
inputs and outputs of the run is produced.  An analogous interface for
running \pythia\ is available.

\section{Portability and  Fortran 90}
\label{sec:f90}
The code was tested on several platforms, including Linux based PC's,
Digital Alpha Unix, HP series 9000/700, Sun work stations and MAC-OSX
with a {\tt g77} compiler.
The user should however check that the compilation
options provided by default in the first few lines of the {\tt
  Makefile} files, including the choice of Fortran compiler, are
consistent with what he has available. We shall be happy to receive
comments related to the portability of the code, and will update the
code to improve its usability.

Toghether with the old Fortran77 ({\tt F77}) version of the \ALPHA\ code
\cite{Caravaglios:1995cd,Caravaglios:1999yr}, we provide a new 
  Fortran90 ({\tt F90}) version.  The evaluation of the \ALPHA\ matrix elements
with the {\tt F90} version is a factor of five to twenty times faster
than the {\tt F77} version, depending on the selected process (the
more complex the process, the better the improvement). When the
overhead of the rest of the code (phase-space, parton densities, etc)
is added in, the overall performance of the code improves by a factor
of two to five (we stress that only the \ALPHA\ part of the code is
available in {\tt F90}; users unfamiliar with {\tt F90} should not be
discouraged from using this version, since this component is a black
box, and its use is compatible with the {\tt F77} part of the code
which the user has access to. Furthermore, the {\tt F90} executables
will run using the same {\tt input} files as the {\tt F77} versions of
the code, and produce the same results, to machine precision).

To link the F90 version of \ALPHA\ it is sufficient to input the choice
of {\tt F90} compiler in the {\tt Makefile}, and issue the comand
{\tt make gen90}, which will produce the executable {\tt wqqgen90}.

For user who do not have access to a {\tt F90} compiler, we provide
one suitable for running on PC's with Linux operating
systems. To set it up, proceed as follows: 
\begin{itemize}
\item go to the {\tt ALPGEN} home directory;
\item issue the command {\tt make ft90V}. This will unpack the file
  {\tt ft90V.tar.gz} and install the {\em Vast/Verydian {\tt F90}}
    compiler into the directory {\tt F90V}. This software was
  distributed freeware for personal use only by Pacific-Sierra
  Research. Before use, you are therefore supposed to agree with the license
  term contained in the directory {\tt F90V/}. In the same directory
  the user can find some documentation on the compiler, including the
  list of supported platforms.
\item Move to the desired work directory (e.g. {\tt wqqwork});
\item issue the command {\tt make gen90V} which will produce the executable
{\tt wqqgen90V}. 
\end{itemize}

\section{New features of version V2.0}
\label{sec:v20}
This Appendix provides an  introduction to the features
of V2. What is
contained in this Appendix supercedes properties or features of eariler
versions possibly listed earlier in this document. 
\subsection{Running modes}
New documentation running modes are available:
\begin{itemize}
\item imode=3: prints to the screen the list of parameters relative to the
  chosen hard process, their default values, and the explanation of
  their meaning
\item imode=4: same as imode=3, but prints to the file par.list
\item imode=5: prints to file prc.list the list of ALL hard processes
  included in alpgen, with the global set of parameters and default
  values
\end{itemize} 
 
The files par.list are already present in each subdirectory /proclib, for each
hard process "proc" (proc=wqq,zqq,etc). 
The file prc.list is already present in the main directory

 \subsection{Input structure:} 
The input files for all processes have now the same structure. There
are 5 input items which are mandatory:
\begin{itemize}
\item the running mode
\item the name of the files to be generated
\item the choice of input grid (new generation, or use or existing grids)
\item the number of events per warmup iteration, and the number of warmup
iterations
\item the number of events to be evaluated during the run after warmup
\end{itemize} 
 
All other parameters required are already set to reasonable default
values. The user can change them by entering a sequence of strings
(representing a given parameter) and the corresponding value. For
example:
 
{\tt njets 3}
 
{\tt mb 4.75}
 
{\tt ptbmin 25}
 
would change the value of njets, b-mass and minimum pt of the b jets.
The code will keep reading until an end of file is met (in interactive
mode the end-of-file command can be issued with ctrl-D). The list of
the strings correpsonding to the relevant parameters is obtained form
the previously mentioned par.list or prc.list files, or can be
generated on the fly by simply entering the string "print 1" (print
tothe screen) or "print 2" (print to the par.list file).

The same structure operates when running in imode=2. Parameters wich
have already been set during the run in imode=1 do not need to be
reinitialised, since they are automatically read in by the executable
from the files written during the imode=1 run. Any attempt to alter
them will be rejected. The list of parameters to be entered in imode=2
can agian be obtained with a "print 1" statement. these correspond
typically to parameters controlling the decay modes of W, Z and top.

In all case, whether imode=0,1 or 2, the code checks whether the
parameter changes requierd by the user are compatible with the
requirements of the code, so there is no risk to make mistakes.


\subsection{Jet-parton matching:} 
A large set of processes allows for the matching of matrix-element
hard partons and shower-generated jets, following the so-called MLM
prescription\footnote{See e.g. {\tt
    http://cern.ch/\~mlm/talks/lund-alpgen.pdf}}.  
Jet-parton matching allows to generate inclusive samples
of arbitrary jet multiplicity. This will be described in detail in a
subsequent publication. A short summary of the use of this option is
given here.
\begin{enumerate}
\item 
  Create weighted event samples running in imode=1, assigning to the
  parameter ickkw the value 1 (default ickkw=0). With this setting,
  the code implements the CKKW scale prescription for the scale of
  alphas. 
\item 
  Generate the unweighted event sample running as usual in
  imode=2. The executable automatically recognizes the ickkw=1
  setting, and performs the alphas-reweighting.
\item 
  The above steps should be carried out for the full set of
  multiplicites one is interested in. For example, for an analysis
  which will extend up to W+4 jet final states, one should generate
  samples of W+0, W+1, W+2, W+3 and W+4 partons.
\item 
  Each of the above samples of unweighted events should be processed
  through a shower code (Herwig or Pythia), using the interface
  provided in the package. The code will enforce the jet matching
  prescription, rejecting events which fail the matching. The
  parameters for the matching (e.g. the definition of the jet to be
  used in the matching) are selected automatically. The code requires
  as input a parameter specifying whether events passing the matching
  criterion and having extra jets due to the parton shower evolution
  can be kept (inclusive mode) or are rejected (exclusive mode). The
  inclsuive mode must be used only for the sample with the highest
  parton multiplcity (e.g. the W+4 jet sample in the example above).
\item 
  The set of showered events which survived the matching should be
  combined to obtain a fully inclusive result. In other words:
  the distributions derived by applying the user analysis to each
  individual sample (W+0, +1, etc) should at the end be summed
  up. Each of the individual distributions will have its own absolute
  normalization, inherited from the cross-section information.
  Since the definition of jet used by the matching prescription
  will most likely not coincide with the jet definition used by the
  user analysis, events from a given partonic multiplicity will
  result in events with a spectrun of jet multiplicities. For example,
  the events obtained form the W+3 jet partonic sample after matching
  may result in events with jet multiplcity not necessarily identical
  to 3. This is the reason why even if we are interested in studying
  just W+3jet final states it is required to include samples obtained
  from lower, as well as higher, parton multiplicity.
\item 
  Processes for which matching is available: 
  \begin{itemize}
  \item all hard processes, with the exception of 4Q, QQh, single top
  \item in the case of Wbb/Wcc, Zbb/Zcc, bb+jets/cc+jets , it is assumed
    that the bottom of charm quarks are treated as jets, and as such
    are subject to the matching requirement. This will evolve in future
    updates, where heavy quarks will be protected from matching,
    thereby allowing te generation of configuration where the pair of
    heavy quarks reconstructs a single jet, or where one of the two is
    either too soft or at too large rapidity to reconstruct an
    independent jet.
  \end{itemize} 
\end{enumerate}

\subsection{New hard processes}

V2.0 includes the following new processes:
\begin{itemize}
\item {\tt hjet}: Higgs production by gluon-gluon fusion, with possible extra
  jets: H+ N jet, with N up to 4. The shower evolution for these
  processes is not enabled as yet.
\item {\tt top}: single top production, with a choice of the following
  channels: 
  \begin{itemize}
  \item top+light quark
  \item top+b
  \item top+W
  \item top+b+W
  \end{itemize}
The different channels can be selected at run time specifying the
parameters {\tt itopprc} (=1,2,3,4 for the 4 channels above, respectively)
No extra jets are allowed for these processes.
\end{itemize} 

\subsection{Code validation utility:}
After unpacking the source code, the user can validate its
installation by issuing the following command in the main directory:

{\tt $>$ make validate}

This script will compile all executables for all hard processes, will
run a test job for each of them and for each jet multiplicity, and
will compare the results against templates already included in the
package. The file val.summary in the /validation directory collects
the differences encountered during this comparison. If the
installation is fine (e.g. is there are no problems with the
difference of operating system and architecture), no difference should
be detected. Else please contact the authors.  the script can be used
to validate the herwig and pythia packages as well. Just run

{\tt $>$ cd validation}

{\tt $>$ ./validate her}

or
 
{\tt $>$ cd validation}
 
{\tt $>$ ./validate pyt}
 
In these cases, the validation might take some time due to the long
compilation time for the herwig and pythia sources. 

When you are done with the validation, you may want to clean up all
the files created in the various subdirectories by the validatin
itself. Just issue the command

{\tt $>$ cd validation}
 
{\tt $>$ ./validate clean}
 
\subsection{Herwig/Pythia versions}
We provide with this package the latest releases of \herwig\  and \pythia,
namely Herwig6507 and Pythia6.320. There is backward compatibility
with earlier verisons of herwig down to 6500, and with
pythia$>$6227. Previous versions of pythia would not work, because of
the lack of the infrastructure (the call to the routine UPVETO)
necessary for the matching. 

\subsection{Other notable features:}
\begin{itemize}
\item The set of available scale options (iqopt) has been rationalised,
  with new scales added and some removed. The various options for each
  process are listed in the par.list files, or can be displayed at run
  time with the "print 1" input. the default is always iqopt=1
\item The definition of "njets" follows the following criterion:
  \begin{itemize}
  \item aside form the expections listed below, njets=0 corresponds to the
    most elementary final state in a given hard process. For example,
    for wbb njets=0 corresponds to the final state W+b+bbar and no extra
    jets. In the case of wjet, njets=0 means production of a single W. 
  \end{itemize} 
\item exceptions:
  \begin{itemize}
  \item Njet: njets=2 is the default, corresponding to 2 final state
    jets
  \item phjet: njets=1 is the default, corresponding to a single photon
    recoiling against a jet
  \end{itemize} 
\end{itemize} 

\section{New features of version V2.1}
\label{sec:v21}
This Appendix provides an introduction to the features of V2.1. What
is contained in this Appendix supercedes properties or features of
eariler versions possibly listed earlier in this document.

\subsection{New hard processes}
V2.1 includes the following new processes:
\begin{itemize}
 \item{\tt vbjet} update: it includes now also photons in the final states 
  \item {\tt wphjet}: $W + \gamma$ production with possible extra jets.
  \item {\tt wphqqjet}: $W b {\bar b} + \gamma$ 
        production with possible extra jets.
  \item {\tt 2Qphjet}: ${Q \bar Q} + \gamma$ production with possible 
        extra jets.
\end{itemize} 

\subsection{Makefiles, compilation}
 To help the user select the compilers and compilation options most
suitable to his platform, these are now collected in a single file
located in the main directory: {\tt compile.mk}

 The Makefile for the various processes refer to compile.mk for the
 compiler options, and to alplib/alpgen.mk for the list of files to be
 compiled. The executable can now be generated within each working
 directory issuing the command:

For the F77 version: 

{\tt $>$ make gen}

For the F90 version: 

{\tt $>$ make gen90}

The two commands generate, respectively, the two files {\tt *usr} and
{\tt *usr90}, where {\tt *=wjet, zjet,} etc.

Thanks to the help of Rafael Yaari, V2.1 compiles under gfortran and
g95.          

\subsection{Matching}
Few changes to the matching are implemented in V2.1:

\begin{enumerate}
\item The matching and vetoing for events with heavy quarks (charm, b and
top) has been improved. For these processes the matching now proceeds
in two steps. First the matching of the light partons with the jets is
performed. As in previous versions, radiation off the heavy quarks is
not used to cluster the jets used in the matching. The radiation
emitted by the heavy quarks is then subsequently included in the
recostruction of possible additional jets, and is used to veto against
extra-jet emissions when generating exclusive samples. 
\item The treatment
of the highest-multiplicity jet sample, which is to be generated in
"inclusive mode (iexc=0) has been slightly modified. Events with extra
jets are kept only if the jets which do not match partons have pt
smaller than the minimum t of the matched jets.  
\item The default
parameters of the matching algorithm (the minimum et of the clusters,
their rapidity range) have been modified. The proposed defaults appear
to be better than the previous ones.  
\item The scale choice has been slightly modified.  In the case of
processes that at LO start with several powers of \as\ (like jet
production, {\tt 2Q} 2Q, or {\tt wqq}), we separate the
powers of \as\ that accompany the LO process from the powers of
\as\ that accompany the additional light partons. It is only to
these extra powers that the CKKW scale is applied. By construction,
this means that for a LO process now ickkw=0 or =1 should give the
same result as ickkw=0 before. 
\item The matching algorithm for some processes (most
notably vbjet with vector boson fusion) is still being optimized, and
will likely be updated in forthcoming releases.
\end{enumerate}

\subsection{Herwig/Pythia versions}
We provide with this package the latest release of \herwig,
namely Herwig6510, which is compatible with {\tt gfortran}.
The version of \pythia\ with this release is 6325, but its use with
the matching is limited to the old \pythia\ shower. The compatibility
of the matching algorithm and the new $k_T$ shower will be addressed
in future releases.

\subsection{Other notable features}
\begin{itemize}
\item A photon-photon minimal separation parameter (drphmin) has been
introduced. So far the minimal separation between photon pairs was
taken to be equal to the minimal dR separaiotn between light
jets. This, among other changes, called for an addition to the
alpgen.inc common blocks, which are therefore now incompatible with
those of V2.0X.  
\item The default value of the charm mass has been set to
1.5 gev (was 0) for all processes with an explicit final-state charm
(e.g. Wc cbar+jets, Wc+jets, cc+jets, etc). 
\item In the process Wc, it is
now possible to place separate generation cuts for the charm (ptcmin
and etacmax) and for the light jets (ptjmin, etajmax).
\item As always, the most recent bugs and bug fixes are documented on
  the alpgen webpage.
\end{itemize}


\end{appendix}
\begin{thebibliography}{99}                                            
%\def    \nuke   #1#2#3{{Nucl. Phys.} {\bf B#1}  (19#2) #3}
\def    \sjnp    #1#2#3{{Sov. J. Nucl. Phys.} {\bf #1} (19#2) #3}
\def    \np     #1#2#3{{Nucl. Phys.} {\bf B#1} (19#2) #3}
\def    \prep   #1#2#3{{Phys. Rep.} {\bf #1}  (19#2) #3}   
\def    \pl     #1#2#3{{Phys. Lett.} {\bf B#1} (19#2) #3}
%\def    \plold  #1#2#3{{Phys. Lett.} {\bf #1B} (19#2) #3}
\def    \prl    #1#2#3{{Phys. Rev. Lett.} {\bf #1}  (19#2) #3}
\def    \pr     #1#2#3{{Phys. Rev.} {\bf D#1}  (19#2) #3}
\def    \prd    #1#2#3{{Phys. Rev.} {\bf D#1}  (19#2) #3}
\def    \zeit   #1#2#3{{Z. Phys.} {\bf C#1}  (19#2) #3}
\def    \epj #1#2#3{{Eur. Phys. J.} {\bf C#1}  (19#2) #3}
\def    \cmp    #1#2#3{{Comm. Math. Phys.} {\bf #1}  (19#2) #3}
\def    \journal #1#2#3#4{{#1} {\bf #2}  (19#3) #4}
\def    \ibid   #1#2#3{{\it ibid.} {\bf #1} (19#2) #3}    
\def    \hepph  #1 {{\tt hep-ph/#1}}
\def    \hepex  #1 {{\tt hep-ex/#1}}
\parskip 0pt
\itemsep=0pt
\small

\bibitem{Mangano:1991by} 
For a review of multi-parton processes in
  QCD, see M.~L.~Mangano and S.~J.~Parke,
%``Multiparton amplitudes in gauge theories,''
Phys.\ Rept.\  {\bf 200} (1991) 301.
%%CITATION = PRPLC,200,301;%%
\bibitem{Gianotti:2002xx}
F.~Gianotti {\it et al.},
%``Physics potential and experimental challenges of the LHC luminosity
%upgrade,'' 
hep-ph/0204087.
%%CITATION = HEP-PH 0204087;%%
\bibitem{Hinchliffe:1993de}
I.~Hinchliffe,
%``The Papageno partonic Monte Carlo program,''
LBL-34372
{\it Submitted to Workshop on Physics at Current Accelerators and the Supercollider, Argonne, IL, 2-5 Jun 1993}.

\bibitem{Berends:1991ax}
F.~A.~Berends, H.~Kuijf, B.~Tausk and W.~T.~Giele,
%``On the production of a W and jets at hadron colliders,''
Nucl.\ Phys.\ B {\bf 357} (1991) 32.
%%CITATION = NUPHA,B357,32;%%

\bibitem{Berends:1989ie}
F.~A.~Berends, W.~T.~Giele and H.~Kuijf,
%``On Six Jet Production At Hadron Colliders,''
Phys.\ Lett.\ B {\bf 232} (1989) 266.
%%CITATION = PHLTA,B232,266;%%

\bibitem{Draggiotis:2002hm}
P.~D.~Draggiotis, R.~H.~Kleiss and C.~G.~Papadopoulos,
%``Multi-jet production in hadron collisions,''
hep-ph/0202201.
%%CITATION = HEP-PH 0202201;%%

\bibitem{Draggiotis:2000sh}
P.~D.~Draggiotis and R.~Kleiss,
%``Parton counting: Physical and computational complexity of multi-jet  production at hadron colliders,''
Eur.\ Phys.\ J.\ C {\bf 17} (2000) 437
[hep-ph/0006133].
%%CITATION = HEP-PH 0006133;%%

\bibitem{Stelzer:1994ta}
T.~Stelzer and W.~F.~Long,
%``Automatic generation of tree level helicity amplitudes,''
Comput.\ Phys.\ Commun.\  {\bf 81} (1994) 357
[hep-ph/9401258].
%%CITATION = HEP-PH 9401258;%%

\bibitem{Pukhov:1999gg}
A.~Pukhov {\it et al.},
%``CompHEP: A package for evaluation of Feynman diagrams and integration  over multi-particle phase space. User's manual for version 33,''
hep-ph/9908288.
%%CITATION = HEP-PH 9908288;%%

\bibitem{Ishikawa:1993qr}
T.~Ishikawa et al.,
[MINAMI-TATEYA group Coll.],
%``GRACE manual: Automatic generation of tree amplitudes in Standard Models: Version 1.0,''
KEK-92-19.

%\cite{Krauss:2001iv}
\bibitem{Krauss:2001iv}
F.~Krauss, R.~Kuhn and G.~Soff,
%``AMEGIC++ 1.0: A matrix element generator in C++,''
JHEP {\bf 0202} (2002) 044
[arXiv:hep-ph/0109036].
%%CITATION = HEP-PH 0109036;%%

\bibitem{Marchesini:1988cf}
G.~Marchesini and B.~R.~Webber,
%``Monte Carlo Simulation Of General Hard Processes With Coherent QCD Radiation,''
Nucl.\ Phys.\ B {\bf 310} (1988) 461.
%%CITATION = NUPHA,B310,461;%%
%\cite{Marchesini:1992ch}
%\bibitem{Marchesini:1992ch}
G.~Marchesini, et al, 
%``HERWIG: A Monte Carlo event generator for simulating hadron emission reactions with interfering gluons. Version 5.1 - April 1991,''
Comput.\ Phys.\ Commun.\  {\bf 67} (1992) 465.
%%CITATION = CPHCB,67,465;%%
%\cite{Corcella:2001bw}
%\bibitem{Corcella:2001bw}
G.~Corcella {\it et al.},
%``HERWIG 6: An event generator for hadron emission reactions with  interfering gluons (including supersymmetric processes),''
JHEP {\bf 0101} (2001) 010
[hep-ph/0011363].
%%CITATION = HEP-PH 0011363;%%

\bibitem{Sjostrand:1994yb}
T.~Sjostrand,
%``High-energy physics event generation with PYTHIA 5.7 and JETSET 7.4,''
Comput.\ Phys.\ Commun.\  {\bf 82} (1994) 74.
%%CITATION = CPHCB,82,74;%%
%\bibitem{Sjostrand:2001wi}
T.~Sjostrand, et al., 
%``High-energy-physics event generation with PYTHIA 6.1,''
Comput.\ Phys.\ Commun.\  {\bf 135} (2001) 238
[hep-ph/0010017].
%%CITATION = HEP-PH 0010017;%%

\bibitem{Paige:1998xm}
F.~E.~Paige, S.~D.~Protopopescu, H.~Baer and X.~Tata,
%``ISAJET 7.40: A Monte Carlo event generator for p p, anti-p p, and  e+ e- reactions,''
hep-ph/9810440.
%%CITATION = HEP-PH 9810440;%%

\bibitem{Seymour:1995df}
M.~H.~Seymour,
%``Matrix element corrections to parton shower algorithms,''
Comput.\ Phys.\ Commun.\  {\bf 90} (1995) 95
[hep-ph/9410414].
%%CITATION = HEP-PH 9410414;%%
%\cite{Corcella:2000gs}
%\bibitem{Corcella:2000gs}
G.~Corcella and M.~H.~Seymour,
%``Initial state radiation in simulations of vector boson production at  hadron colliders,''
Nucl.\ Phys.\ B {\bf 565} (2000) 227
[hep-ph/9908388].
%%CITATION = HEP-PH 9908388;%%
%\cite{Miu:1999ju}
%\bibitem{Miu:1999ju}
G.~Miu and T.~Sjostrand,
%``W production in an improved parton shower approach,''
Phys.\ Lett.\ B {\bf 449} (1999) 313
[hep-ph/9812455].
%%CITATION = HEP-PH 9812455;%%

\bibitem{Corcella:1998rs}
G.~Corcella and M.~H.~Seymour,
%``Matrix element corrections to parton shower simulations of heavy quark  decay,''
Phys.\ Lett.\ B {\bf 442} (1998) 417
[hep-ph/9809451].
%%CITATION = HEP-PH 9809451;%%
%\cite{Corcella:2000wq}
%\bibitem{Corcella:2000wq}
G.~Corcella, M.~L.~Mangano and M.~H.~Seymour,
%``Jet activity in t anti-t events and top mass reconstruction at hadron  colliders,''
JHEP {\bf 0007} (2000) 004
[hep-ph/0004179].
%%CITATION = HEP-PH 0004179;%%

\bibitem{Dobbs:2001gb}
M.~Dobbs,
%``Incorporating next-to-leading order matrix elements for hadronic  diboson production in showering event generators,''
Phys.\ Rev.\ D {\bf 64} (2001) 034016
[hep-ph/0103174].
%%CITATION = HEP-PH 0103174;%%

\bibitem{Grace:2003npb}
Y. Kurihara et al., Nucl. \ Phys. \ B {\bf 654} (2003) 301
[hep-ph/0212216].

\bibitem{Frixione:2002ik}
S.~Frixione and B.~R.~Webber,
%``Matching NLO QCD computations and parton shower simulations,''
[hep-ph/0204244].
%%CITATION = HEP-PH 0204244;%% 

\bibitem{Frixione:2003ei}
S.~Frixione, P.~Nason and B.~R.~Webber,
%``Matching NLO QCD and parton showers in heavy flavour production,''
JHEP {\bf 0308} (2003) 007
[arXiv:hep-ph/0305252].
%%CITATION = HEP-PH 0305252;%%

\bibitem{Collins:2000qd}
J.~C.~Collins,
%``Subtraction method for NLO corrections in Monte Carlo event generators  for leptoproduction,''
JHEP {\bf 0005} (2000) 004
[hep-ph/0001040].
%%CITATION = HEP-PH 0001040;%%
\bibitem{Catani:2001cc}
S.~Catani, F.~Krauss, R.~Kuhn and B.~R.~Webber,
%``QCD matrix elements + parton showers,''
JHEP {\bf 0111} (2001) 063
[arXiv:hep-ph/0109231].
%%CITATION = HEP-PH 0109231;%%
\bibitem{Krauss:2002up}
F.~Krauss,
%``Matrix elements and parton showers in hadronic interactions,''
JHEP {\bf 0208} (2002) 015
[arXiv:hep-ph/0205283].
%%CITATION = HEP-PH 0205283;%%
\bibitem{kraussetal}
See talks by F. Krauss, S. Mrenna and P. Richardson at the Fermilab
Workshop on MC tunings, April 29-30 2003, {\tt
  http://cepa.fnal.gov/CPD/MCTuning/} .
\bibitem{Caravaglios:1999yr}
F.~Caravaglios, M.~L.~Mangano, M.~Moretti and R.~Pittau,
%``A new approach to multijet calculations in hadron collisions,''
Nucl.\ Phys.\ B {\bf 539} (1999) 215,
[hep-ph/9807570].
%%CITATION = HEP-PH 9807570;%%

\bibitem{Mangano:2001xp}
M.~L.~Mangano, M.~Moretti and R.~Pittau,
%``Multijet matrix elements and shower evolution in hadronic collisions:  W b anti-b + n jets as a case study,''
Nucl.\ Phys.\ B {\bf 632} (2002) 343, [hep-ph/0108069].
%%CITATION = HEP-PH 0108069;%%

\bibitem{Belyaev:2000wn}
A.~S.~Belyaev {\it et al.},
%``CompHEP-PYTHIA interface: Integrated package for the collision events  generation based on exact matrix elements,''
hep-ph/0101232.
%%CITATION = HEP-PH 0101232;%%
\bibitem{Sato:2001ae}
K.~Sato et al., 
%``Integration of GRACE and PYTHIA,''
hep-ph/0104237.
%%CITATION = HEP-PH 0104237;%%
\bibitem{Kersevan:2002dd}
B.~P.~Kersevan and E.~Richter-Was,
%``The Monte Carlo event generator AcerMC version 1.0 with interfaces to  PYTHIA 6.2 and HERWIG 6.3,''
hep-ph/0201302.
%%CITATION = HEP-PH 0201302;%%
\bibitem{Tsuno:2002ae}
S.~Tsuno {\it et al.}, 
%``CITATION of GR@PPA_4b''
hep-ph/0204222.
%%CITATION = HEP-PH 0104237;%%
\bibitem{Maltoni:2002qb}
F.~Maltoni and T.~Stelzer,
%``MadEvent: Automatic event generation with MadGraph,''
JHEP {\bf 0302} (2003) 027
[arXiv:hep-ph/0208156].
%%CITATION = HEP-PH 0208156;%%
\bibitem{Boos:2001cv}
E.~Boos {\it et al.},
%``Generic user process interface for event generators,''
hep-ph/0109068.
%%CITATION = HEP-PH 0109068;%%
%\bibitem{Giele:2002hx}
W.~Giele {\it et al.},
%``The QCD/SM working group: Summary report,''
hep-ph/0204316.
%%CITATION = HEP-PH 0204316;%%

\bibitem{Caravaglios:1995cd}
F.~Caravaglios and M.~Moretti,
%``An algorithm to compute Born scattering amplitudes without Feynman graphs,''
Phys.\ Lett.\ B {\bf 358} (1995) 332
[hep-ph/9507237].
%%CITATION = HEP-PH 9507237;%%

\bibitem{Draggiotis:1998gr}
P.~Draggiotis, R.~H.~Kleiss and C.~G.~Papadopoulos,
%``On the computation of multigluon amplitudes,''
Phys.\ Lett.\ B {\bf 439} (1998) 157
[hep-ph/9807207].
%%CITATION = HEP-PH 9807207;%%

\bibitem{Lai:1996mg}
H.~L.~Lai {\it et al.},
%``Improved parton distributions from global analysis of recent deep  inelastic scattering and inclusive jet data,''
Phys.\ Rev.\ D {\bf 55} (1997) 1280
[hep-ph/9606399].
%%CITATION = HEP-PH 9606399;%%
\bibitem{Lai:2000wy}
H.~L.~Lai {\it et al.}  [CTEQ Coll.],
%``Global {QCD} analysis of parton structure of the nucleon: CTEQ5 parton  distributions,''
Eur.\ Phys.\ J.\ C {\bf 12} (2000) 375
[hep-ph/9903282].
%%CITATION = HEP-PH 9903282;%%

\bibitem{Pumplin:2002vw}
J.~Pumplin et al., 
%``New generation of parton distributions with uncertainties from global  QCD analysis,''
hep-ph/0201195.
%%CITATION = HEP-PH 0201195;%%

\bibitem{Stump:2003yu}
D.~Stump, J.~Huston, J.~Pumplin, W.~K.~Tung, H.~L.~Lai, S.~Kuhlmann and J.~F.~Owens,
%``Inclusive jet production, parton distributions, and the search for new  physics,''
arXiv:hep-ph/0303013.
%%CITATION = HEP-PH 0303013;%%

\bibitem{Martin:1999ww}
A.~D.~Martin, R.~G.~Roberts, W.~J.~Stirling and R.~S.~Thorne,
%``Parton distributions and the LHC: W and Z production,''
Eur.\ Phys.\ J.\ C {\bf 14} (2000) 133
[hep-ph/9907231].
%%CITATION = HEP-PH 9907231;%%
\bibitem{Martin:2001es}
A.~D.~Martin, R.~G.~Roberts, W.~J.~Stirling and R.~S.~Thorne,
%``MRST2001: Partons and alpha(s) from precise deep inelastic scattering  and Tevatron jet data,''
Eur.\ Phys.\ J.\ C {\bf 23} (2002) 73
[hep-ph/0110215].
%%CITATION = HEP-PH 0110215;%%
\bibitem{Martin:2002dr}
A.~D.~Martin, R.~G.~Roberts, W.~J.~Stirling and R.~S.~Thorne,
%``NNLO global parton analysis,''
hep-ph/0201127.
%%CITATION = HEP-PH 0201127;%%
\bibitem{Kunszt:1984ri}
Z.~Kunszt,
%``Associated Production Of Heavy Higgs Boson With Top Quarks,''
Nucl.\ Phys.\ B {\bf 247} (1984) 339.
%%CITATION = NUPHA,B247,339;%%
%\cite{Mangano:1993kp}
\bibitem{Mangano:1993kp}
M.~L.~Mangano,
%``Production of W plus heavy quark pairs in hadronic collisions,''
Nucl.\ Phys.\ B {\bf 405} (1993) 536.
%%CITATION = NUPHA,B405,536;%%
%\bibitem{Ellis:1999fv}
R.~K.~Ellis and S.~Veseli,
%``Strong radiative corrections to W b anti-b production in p anti-p  collisions,''
Phys.\ Rev.\ D {\bf 60} (1999) 011501
[hep-ph/9810489].
%%CITATION = HEP-PH 9810489;%%

%\cite{Haywood:1999qg}
\bibitem{Haywood:1999qg}
S.~Haywood {\it et al.},
%``Electroweak physics,''
hep-ph/0003275.
%%CITATION = HEP-PH 0003275;%%

\bibitem{Barger:1989cp}
V.~D.~Barger, T.~Han and H.~Pi,
%``Four Weak Boson Production At E+ E- And P P Supercolliders,''
Phys.\ Rev.\ D {\bf 41} (1990) 824.
%%CITATION = PHRVA,D41,824;%%

\bibitem{Mangano:jk}
M.~L.~Mangano, P.~Nason and G.~Ridolfi,
%``Heavy Quark Correlations In Hadron Collisions At Next-To-Leading Order,''
Nucl.\ Phys.\ B {\bf 373} (1992) 295.
%%CITATION = NUPHA,B373,295;%%

%\cite{Carena:2000yx}
\bibitem{Carena:2000yx}
M.~Carena {\it et al.},
%``Report of the Tevatron Higgs working group,''
hep-ph/0010338.
%%CITATION = HEP-PH 0010338;%%
\bibitem{spira}
M.~Spira, Higgs production code available from the URL: \newline
{\tt http://home.cern.ch/mspira/hqq/}. The 
code is based on matrix elements from~\cite{Kunszt:1984ri,gunion}.
\bibitem{gunion}
J.F. Gunion, Phys. Lett. B{\bf 253} (1991) 269;
W.J. Marciano and F.E. Paige, Phys. Rev. Lett. {\bf 66} (1991) 2433;
D.A. Dicus and S. Willenbrock, Phys. Rev. D {\bf39} (1989) 751;
M. Spira, Fortschr. Phys. {\bf 46} (1998) 203.

\bibitem{topdrawer}
    Topdrawer, Roger B. Chaffee, Computation Research Group, SLAC, 
   CGTM No.  178. 
\end{thebibliography}
\end{document}
----------------------------------------------------------------------

%%%%%%%%%%%%%%%%%%%%%%%%%%%%%%%%%%%%%%%%%%%%%%%%%%%%%%%%%%%%%%%%%
\bibitem{Bassetto:1983ik}
A.~Bassetto, M.~Ciafaloni and G.~Marchesini,
%``Jet Structure And Infrared Sensitive Quantities In Perturbative QCD,''
Phys.\ Rept.\  {\bf 100} (1983) 201.
%%CITATION = PRPLC,100,201;%%
%\cite{Gribov:1983tu}
%\bibitem{Gribov:1983tu}
L.~V.~Gribov, E.~M.~Levin and M.~G.~Ryskin,
%``Semihard Processes In QCD,''
Phys.\ Rept.\  {\bf 100} (1983) 1.
%%CITATION = PRPLC,100,1;%%


\bibitem{Abe:1994nj}
F.~Abe {\it et al.}  [CDF Coll.],
%``Evidence for color coherence in p anti-p collisions at s**(1/2) = 1.8-TeV,''
Phys.\ Rev.\ D {\bf 50} (1994) 5562.
%%CITATION = PHRVA,D50,5562;%%
%\cite{Abbott:1999cu}
%\bibitem{Abbott:1999cu}
B.~Abbott {\it et al.}  [D0 Coll.],
%``Evidence of color coherence effects in W + jets events from p anti-p  collisions at s**(1/2) = 1.8-TeV,''
Phys.\ Lett.\ B {\bf 464} (1999) 145
[hep-ex/9908017].
%%CITATION = HEP-EX 9908017;%%
%\cite{Abbott:1997bk}
%\bibitem{Abbott:1997bk}
B.~Abbott {\it et al.}  [D0 Coll.],
%``Color coherent radiation in multijet events from p anti-p collisions  at s**(1/2) = 1.8-TeV,''
Phys.\ Lett.\ B {\bf 414} (1997) 419
[hep-ex/9706012].
%%CITATION = HEP-EX 9706012;%%

\bibitem{Mangano:1988xk}
%\cite{Berends:1987cv}
%\bibitem{Berends:1987cv}
F.~A.~Berends and W.~Giele,
%``The Six Gluon Process As An Example Of Weyl-Van Der Waerden Spinor Calculus,''
Nucl.\ Phys.\ B {\bf 294} (1987) 700.
%%CITATION = NUPHA,B294,700;%%
M.~Mangano, S.~Parke and Z.~Xu,
%``Duality And Multi - Gluon Scattering,''
Nucl.\ Phys.\ B {\bf 298} (1988) 653.
%%CITATION = NUPHA,B298,653;%%


\bibitem{Mangano:1988kp}
M.~Mangano and S.~J.~Parke,
%``Quark - Gluon Amplitudes In The Dual Expansion,''
Nucl.\ Phys.\ B {\bf 299} (1988) 673.
%%CITATION = NUPHA,B299,673;%%
%\cite{Berends:1988me}
%\bibitem{Berends:1988me}
F.~A.~Berends and W.~T.~Giele,
%``Recursive Calculations For Processes With N Gluons,''
Nucl.\ Phys.\ B {\bf 306} (1988) 759.
%%CITATION = NUPHA,B306,759;%%


\bibitem{Odagiri:1998ep}
K.~Odagiri,
%``Color connection structure of (supersymmetric) {QCD} (2 $\to$ 2)  processes,''
JHEP {\bf 9810} (1998) 006
[hep-ph/9806531].
%%CITATION = HEP-PH 9806531;%%

\bibitem{Draggiotis:2000gm}
P.~D.~Draggiotis, A.~van Hameren and R.~Kleiss,
%``SARGE: An algorithm for generating QCD antennas,''
Phys.\ Lett.\ B {\bf 483} (2000) 124
[hep-ph/0004047].
%%CITATION = HEP-PH 000404
%\bibitem{vanHameren:2000aj}
A.~van Hameren and R.~Kleiss,
%``Generating QCD antennas,''
Eur.\ Phys.\ J.\ C {\bf 17}, 611 (2000)
[hep-ph/0008068].




\bibitem{Belyaev:1999dn}
A.~S.~Belyaev, E.~E.~Boos and L.~V.~Dudko,
%``Single top quark at future hadron colliders: Complete signal and  background study,''
Phys.\ Rev.\ D {\bf 59} (1999) 075001
[hep-ph/9806332].
%%CITATION = HEP-PH 9806332;%%


\bibitem{Abe:1994st}
F.~Abe {\it et al.}  [CDF Coll.],
%``Evidence for top quark production in anti-p p collisions at s**(1/2) = 1.8-TeV,''
Phys.\ Rev.\ D {\bf 50} (1994) 2966.
%%CITATION = PHRVA,D50,2966;%%
%\cite{Abe:1995hr}
%\bibitem{Abe:1995hr}
F.~Abe {\it et al.}  [CDF Coll.],
%``Observation of top quark production in anti-p p collisions,''
Phys.\ Rev.\ Lett.\  {\bf 74} (1995) 2626
[hep-ex/9503002].
%%CITATION = HEP-EX 9503002;%%
%\cite{Abachi:1995iq}
%\bibitem{Abachi:1995iq}
S.~Abachi {\it et al.}  [D0 Coll.],
%``Observation of the top quark,''
Phys.\ Rev.\ Lett.\  {\bf 74} (1995) 2632
[hep-ex/9503003].
%%CITATION = HEP-EX 9503003;%%



\bibitem{Giele:1990vh}
W.~T.~Giele, T.~Matsuura, M.~H.~Seymour and B.~R.~Webber,
%``W Boson Plus Multijets At Hadron Colliders: Herwig Parton Showers Versus Exact Matrix Elements,''
FERMILAB-CONF-90-228-T
{\it Contribution to Proc. of 1990 Summer Study on High Energy
  Physics: Research Directions for the Decade, Snowmass, CO, Jun 25 -
  Jul 13, 1990}.
Published in Snowmass Summer Study 1990:0137-147.


\bibitem{Benlloch:1992fk}
J.M.~Benlloch, A.~Caner, M.L.~Mangano and T.~Rodrigo,
CDF/DOC/MONTECARLO/1823, Oct.1992.
%\bibitem{Benlloch:1992fk}
J.~M.~Benlloch  [CDF Coll.],
%``Comparison of W + jets CDF data and ME MC's interfaced with fragmentation models,''.
Published in the Proceedings of DPF 92, vol. 2, 1091-1093.

\bibitem{getjet}
GETJET, F. Paige and M.~Seymour, private communication.


\bibitem{Catani:1992zp}
S.~Catani, Y.~L.~Dokshitzer and B.~R.~Webber,
%``The K-perpendicular clustering algorithm for jets in deep inelastic scattering and hadron collisions,''
Phys.\ Lett.\ B {\bf 285} (1992) 291.
%%CITATION = PHLTA,B285,291;%%
%\cite{Ellis:1993tq}
%\bibitem{Ellis:1993tq}
S.~D.~Ellis and D.~E.~Soper,
%``Successive combination jet algorithm for hadron collisions,''
Phys.\ Rev.\ D {\bf 48} (1993) 3160
[hep-ph/9305266].
%%CITATION = HEP-PH 9305266;%%
%\cite{Seymour:1998kj}
%\bibitem{Seymour:1998kj}
M.~H.~Seymour,
%``Jet shapes in hadron collisions: Higher orders, resummation and  hadronization,''
Nucl.\ Phys.\ B {\bf 513} (1998) 269
[hep-ph/9707338].
%%CITATION = HEP-PH 9707338;%%

%\cite{Abe:1993si}
\bibitem{Abe:1993si}
F.~Abe {\it et al.}  [CDF Coll.],
%``Measurement of jet multiplicity in W events produced in p anti-p collisions at s**(1/2) = 1.8-Tev,''
Phys.\ Rev.\ Lett.\  {\bf 70} (1993) 4042.
%%CITATION = PRLTA,70,4042;%%
%\cite{Abe:1996fg}
\bibitem{Abe:1996fg}
F.~Abe {\it et al.}  [CDF Coll.],
%``Properties of jets in Z boson events from 1.8-TeV anti-p p collisions,''
Phys.\ Rev.\ Lett.\  {\bf 77} (1996) 448
[hep-ex/9603003].
%%CITATION = HEP-EX 9603003;%%
----------------------------------------------------------------------


INPUT PARAMETERS
================

Parameters controlling the run and I/O are passed to the codes through
the routine SETPAR. The code contains comment lines which should be
sufficiently self-explanatory. The kinematics of the events is defined
in the rest frame of the hadron-hadron collision.

When running hard processes involving hard quarks, the masses of the
heavy quarks are kept different from 0 throughout the calculation. As
a result no cuts on the minimum momentum of the heavy quark nor on the
minimum angular separation between heavy quark and heavy antiquark are
necessary. One can then explore for example the contribution to the
cross-section of configurations where both b and bbar are close enough
to belong to the same jet (`gluon-splitting' contributions)

Due to the high dimensionality of the final phase-space, and due to
the large range of possible partonic CM energies, the event weights
span a very large range. While several techniques have been used to
optimise the generation of phase-space points, a large number of
events is necessary before distributions become smooth. When the code
is run in imode=0 or 1, the option exists to generate a number of
events with the sole purpose of producing an optimal integration grid
which will reduce the event-by-event fluctuations. The information on
this grid can be saved, using the proper run-time option, to a file, so that
later runs will not require this initialisation phase.  As a rule of
thumb, for the generation of W+4 jet events we suggest one or two
initialisation iterations, with no less than 5 million events
each. For different jet multiplicities, we suggest a factor of 2
increase (reduction) for each parton extra (less).


TESTS
=====
Some benchmark results are provided with the package. They are
contained in the subdirectory tests/
They include the output of a series of runs generated for 3-jet final
states. The user can recreate them by following the instructions given
here:

In the directory wqqwork/, issue the comand:

{\tt $>$ wqqgen < input   }

This will start a run with imode=1, 1000000 events in 2 iterations for
optimization, generated 10000000 weighted events written to file
(w3j.wgt). The kinematical cuts, PDFs etc. are those appearing as
defaults. Some distributions are contained in the topdrawer file
w3j.top. This is a readable ASCII file, containing all the
distributions. The user can turn it into a postscript file, if she has
topdrawer installed. For this example, running topdrawer on the file
should give the output we produce in test/w3j.ps 

To unweight the events generated in the previous run, 
rerun the code:

{\tt $>$ wqqgen   }

inputting imode=2 and w3j as file name.  This leads to some number of
unweighted events. The same distributions as above, now limited to the
set of unweighted events, are contained in the topdrawer file
w3j\_unw.top, and in a postscript file (w3j\_unw.ps)

We can now process the unweighted events through Herwig:

{\tt $>$ hwgwqq}

This code develops the unweighted events selected above through the
HERWIG parton showers. For simplicity, in this example only the
perturbative radiation stage was included (no hadron formation), and
jets are reconstructed using the GETJET routines.
To generate the full hadronic shower, including hadron formation and
underlying event, use herlib/hwuser.f as a driver.

Some jet distributions for the unweighted events are contained in the
topdrawer file w3j\_her.top. Here quantities before and after shower
evolution are compared (e.g. spectra of jets, correlations in eta-phi
between the n-th parton and the n-th jets, etc).  The plots show for
example that some extra jet, beyond those present at parton level, is
generated by the shower evolution.  The plots can also be seen viewing
directly the postscript file in test/w3j\_her.ps.


General comments, Q&A, etc.
---------------------------
Q: why is the unweighting not done on the fly?
A: processes with many partons in the final state have a very large
   range of weights. A new maximum weight can be found after a very
   long run, and the unweighting of the previous events may have been
   biased. Doing the unweighting at the end guarantees that the
   maximum weight for this data set has been found. There is one more
   reason for doing the unweighting in a second step: once the full
   weight distribution for the generated sample is known, the
   unweighting efficiency can be optimised by selecting a reference
   maximum weight which is lower than the absolute maximum
   weight. This is possible, provided it does not bias the
   unweighting. The user can steer the unweighting procedure by
   editing the function UNWGT, contained at the end of the WQQUSR.F
   module. 

Q: why is the HERWIG shower generated as a third step, instead of
   taking place during the  unweighting phase?
A: keeping the HERWIG shower generation as a separate step allows to
   keep the two groups of codes separated. This allows a simpler
   maintenance of the codes when either of the two evolves and is
   updated. Furthermore, this allows to perform the shower evolution
   using different HERWIG settings (in order, for example, to study
   systematic effects due to a choice of shower cutoffs, underlying
   event modeling, etc) without having to repeat the unweighting. It
   is sufficient to rerun HERWIG on the same file of unweighted
   events. Finally, given the larger disk-space requirements of the
   weighted-event files, one can consider adding new statistics to the
   unweighted-event file in subsequent steps.

Q: what is the overall normalization of the plots?  
A: all x-sections are given in pb. Plots filled when running with
   imode=0,1 should be rescaled at the end of the run by multiplying
   with avgwgt/totwgt (i.e. the inverse of the total number of events
   generated, including those rejected at the generation level because
   of phase-space cuts etc.).  Plots filled with imode=2 should be
   rescaled by the user dividing by the total number of events which
   were unweighted (unwev). In all cases (imode=0,1,2) these rescalings are
   already done automatically in the provided version of the
   histogram-finalisation routine FINHIS.  
   In the provided version, the plots normalization is pb/bin. For
   absolute normalizations such as pb/gev, a bin-size correction
   should be done by the user. In the version provided this can
   be achieved directly on the topdrawer plots, by including the
   statement SET ORDER X Y XNORM just before the string of histogram
   points, with XNORM=1/bin-size.


References
[1] F.Caravaglios and M.Moretti, Phys.Lett.B358:332-338,1995, 
[2] F. Caravaglios, M.L. Mangano, M. Moretti and R. Pittau,
    Nucl.Phys.B539:215-232,1999.
[3] G. Marchesini and B.R. Webber, Nucl.Phys.B310:461,1988 
    G. Corcella, I.G. Knowles, G. Marchesini, S. Moretti, K. Odagiri,
    P. Richardson, M.H. Seymour and B.R. Webber, 
    JHEP 0101:010,2001 (hep-ph/0011363)
    G. Marchesini, B.R. Webber, G. Abbiendi, I.G. Knowles, 
    M.H. Seymour, L. Stanco, Comput.Phys.Commun.67:465-508,1992 
[4] \bibitem{Boos:2001cv}
    E.~Boos {\it et al.},
    %``Generic user process interface for event generators,''
    hep-ph/0109068.
    %%CITATION = HEP-PH 0109068;%%

Michelangelo Mangano  
michelangelo.mangano@cern.ch

Mauro Moretti
moretti@fe.infn.it

Fulvio Piccinini
fulvio.piccinini@cern.ch

Roberto Pittau
pittau@to.infn.it

Antonello Polosa
antonio.polosa@cern.ch

\end{document}
