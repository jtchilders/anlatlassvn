\documentclass[paper]{JHEP3}
\usepackage{amssymb,enumerate,url,amsmath}
\bibliographystyle{JHEP}


%%%%%%%%%% Start TeXmacs macros
\newcommand{\tmtextit}[1]{{\itshape{#1}}}
\newcommand{\tmtexttt}[1]{{\ttfamily{#1}}}
\newenvironment{enumeratenumeric}{\begin{enumerate}[1.] }{\end{enumerate}}
\newcommand\sss{\mathchoice%
{\displaystyle}%
{\scriptstyle}%
{\scriptscriptstyle}%
{\scriptscriptstyle}%
}

\newcommand\as{\alpha_{\sss\rm S}}
\newcommand\POWHEG{{\tt POWHEG}}
\newcommand\HERWIG{{\tt HERWIG}}
\newcommand\PYTHIA{{\tt PYTHIA}}
\newcommand\MCatNLO{{\tt MC@NLO}}

\newcommand\kt{k_{\sss\rm T}}

\newcommand\pt{p_{\sss\rm T}}
\newcommand\LambdaQCD{\Lambda_{\scriptscriptstyle QCD}}
%%%%%%%%%% End TeXmacs macros


\title{The \mbox{POWHEG BOX} user manual:\\
  Single-top t-channel in the 4-flavour scheme} \vfill
\author{Emanuele Re\\
  Institute for Particle Physics Phenomenology, Department of Physics\\
  University of Durham, Durham, DH1 3LE, UK\\
  E-mail: \email{emanuele.re@durham.ac.uk}}

\vskip -0.5truecm

\keywords{POWHEG, Shower Monte Carlo, NLO}

\abstract{This note documents the use of the package \tmtexttt{POWHEG
    BOX} for the single-top $t$-channel production process computed in
  the $4$-flavour scheme.  Results can be easily interfaced to shower
  Monte Carlo programs, in such a way that both NLO and shower
  accuracy are maintained.  Please read carefully the manual before
  using this code.}  \preprint{\today\\
  \tmtexttt{POWHEG BOX} 1.0}

\begin{document}


\section{Introduction}

The \tmtexttt{POWHEG BOX} program is a framework for implementing NLO
calculations in Shower Monte Carlo programs according to the
\POWHEG{} method. An explanation of the method and a discussion of
how the code is organized can be found in
refs.~\cite{Nason:2004rx,Frixione:2007vw,Alioli:2010xd}.  The code is
distributed according to the ``MCNET GUIDELINES for Event Generator
Authors and Users'' and can be found at the web page \\
%
\begin{center}
 \url{http://powhegbox.mib.infn.it}.
\end{center}
%
~\\
%
In the following we will focus on the implementation of single-top
$t$-channel production in the $4$-flavour scheme, whose source files
can be found in the \tmtexttt{POWHEG-BOX/ST\_tch\_4f} subdirectory.

This program is an implementation of the NLO cross section calculated
in~\cite{Campbell:2009ss} in the \POWHEG{} formalism of
refs.~\cite{Nason:2004rx,Frixione:2007vw}. Tree level Born and real
matrix-elements were obtained from scratch, whereas the one-loop
contribution was taken from the MCFM package.

A detailed description of the implementation, together with a
comparison with the results for the same process as implemented with
the \MCatNLO{} method, can be found in ref.~\cite{Frederix:2012dh}. In
the current version, spin correlations of the top-quark decay products
are not included, which means that angular correlation between the
production and the decay stage are not present.
%

\section{Installation}
Before describing the installation procedure, we want to recall that a
NLO+PS simulation of the single-top $t$-channel process, in the
``$5$-flavour'' scheme, is already present in the \tmtexttt{POWHEG
  BOX} package (\tmtexttt{POWHEG-BOX/ST\_tch} directory,
ref.~\cite{Alioli:2009je}). However, the process of single-top
$t$-channel production can be described also in a different scheme,
namely the ``$4$-flavour'' scheme. Although the $4$- and $5$-flavour
prescriptions are formally equivalent if all the perturbative
expansion were taken into account, they can show differences when the
perturbative series is truncated, as unavoidably happens in a real
computation. Both schemes have advantages and disadvantages, depending
on the observables studied. The purpose of this implementation is to
allow a simulation in the $4$-flavour scheme too, in order to allow
for an assessment of these differences with NLO+PS accuracy. More
commments and a more detailed explanation of differences and
similarities between the 2 approaches can be found in
ref.~\cite{Frederix:2012dh}, and in the references therein.

Please notice that the PDF choice is \textbf{not} arbitrary for this
implementation: a $4$-flavour PDF set (as for example {\tt
  MSTW2008nlo68cl\_nf4}) must be used for consistency of the
calculation. To this aim, the installation of
LHAPDF~\cite{Whalley:2005nh} is strongly suggested. Notice also that,
in the current version, the program doesn't check explicitly that such
a set is actually selected in the input file.\footnote{It is possible
  to add in the code some compensating terms, to make the computation
  consistent with a $5$-flavour PDF set. By default these terms are
  off. An explicit switch to turn them on from the input card could be
  made available upon request, if needed.}

In order to run the \tmtexttt{POWHEG BOX} program, we recommend the
reader to start from the \tmtexttt{POWHEG BOX} user manual, which
contains all the information and settings that are common between all
subprocesses. In this note we focus on the settings and parameters
specific to the single-top $t$-channel $4$-flavour implementation.

\section{Generation of events and showering}

The executable is built with the following commands\\~\\
\tmtexttt{
%
  \$ cd POWHEG-BOX/ST\_tch\_4f\\
% 
  \$ make pwhg\_main\\
%
}
\\
In the \tmtexttt{testrun} folder, there are examples of input files.
For example, you can start a run doing\\~\\
\tmtexttt{
%
  \$ cd testrun\\
% 
  \$ ../pwhg\_main\\
%
}
\\
and then insert the prefix of an input file when prompted.  The input
file read in this case will be \tmtexttt{<prefix>-powheg.input} and at
the end a file named \tmtexttt{<prefix>-events.lhe} will contain
events, in the Les Houches format.
To shower them with \PYTHIA{} do\\~\\
\tmtexttt{
%
\$ cd POWHEG-BOX/ST\_tch\_4f\\
%
\$ make main-PYTHIA-lhef \\
%
\$ cd testrun \\
%
\$ ../main-PYTHIA-lhef\\
%
}
\\
Similar commands will run the \HERWIG{} shower.
~\\~\\
It is possible to speed up the generation of events by splitting this
stage in several short runs, producing several {\tt .lhe} files.  The
full description of this procedure can be found in {\tt
  POWHEG-BOX/Docs/Manyseeds.pdf}, and a summary of it, suited for this
implementation, is reported in the following.\\
%
To parallelize the event generation, the code has first to be run setting in the input card\\~\\
\tmtexttt{
%
numevts 0\\
%
}
\\
When this run is finished all the grids for subsequent runs will be
ready, but no event file will be present yet. At this point, the user
should uncomment the option {\tt manyseeds} in the input card:\\~\\
\tmtexttt{
%
manyseeds 1\\
%
}
\\
and decide the number of events for each short run, setting
{\tt numevts} as usual.
In order to start the event generation, a file named {\tt <prefix>-seeds.dat}
has to be created, with the following format:\\~\\
\tmtexttt{
%
  randomseed1\\
  randomseed2\\
  randomseed3\\
  ..\\
  ..\\
%
}\\
where each line has to be the integer that will be used to initialize
the random numbers generator.  At this point, by running the code with
the usual command {\tt pwhg\_main}, the user will be asked not only
the prefix of the input file, but also an integer. This integer
corresponds to the line in {\tt <prefix>-seeds.dat} that the program
will read to initialize the random numbers generator.  The program
will use all the grids generated, and the event file produced will
contain in the name also the integer typed, so there is no risk of
overwriting files.



\section{Process specific input parameters}
\tmtexttt{
  facscfact 1 ! factorization scale factor: mufact=muref*facscfact\\
  renscfact 1 ! renormalization scale factor: muren=muref*renscfact\\
}
\\
Factorization and renormalization scale factors appearing here have to
do with the computation of the inclusive cross section (i.e.~the
$\bar{B}$ function~\cite{Nason:2004rx,Frixione:2007vw,Alioli:2010xd}),
and can be varied by a factor of order 1 to study scale dependence.  A
physically motivated choice for this process is of the order of the
transverse mass of the $b$-quark~\cite{Frederix:2012dh}. The default
scale choice is therefore
\begin{equation}
\mu = 4 \sqrt{(m_b^2+p_{T,b}^2)}\,,\nonumber
\end{equation}
although other scale choices are possible. The experienced user can
change this setting modifying the \tmtexttt{set\_fac\_ren\_scales}
routine.
\\~\\
It follows a description of parameters which are relevant for this
production process:
\begin{itemize}
\item When generating events with this implementation, negative
  weights can appear. As usual the fraction of negative weights in
  {\tt POWHEG} can be largely reduced increasing \tmtexttt{foldcsi},
  \tmtexttt{foldy}, \tmtexttt{foldphi}.  Allowed values are 1, 2, 5,
  10, 25, 50. The speed of the program is inversely proportional to
  the product of these numbers, so that a reasonable compromise should
  be found. When using the folding parameters provided in the
  examples, the fraction of negative-weighted generated events is
  quite small (few \%), even at LHC energies.  However, it is
  recommended to keep these events in the generated sample, by keeping
  the \tmtexttt{withnegweights} flag enabled, as in the example input
  files.
\item Enabling the \tmtexttt{withdamp} option can slightly improve the
  speed of event-generation. Notice that, for this process, no special
  damping function is needed (cfr. the Higgs via gluon-fusion
  implementaion and the large $\bar{B}/B$ factor). The only effect
  here is to improve the efficency of event generation for events
  where the ratio $R/B$ in the \POWHEG{} Sudakov exponent becomes too
  large with respect to its corresponding collinear or soft
  approximation~\cite{Alioli:2010xd}.
\end{itemize}

The other relevant parameters are:\\
~\\
\tmtexttt{
ttype   1 \phantom{aaaaaaaaaaaa} ! 1 for t, -1 for tbar\\
topmass 175.0 \phantom{aaaaaa}   ! top mass\\
bmass 4.75 \phantom{aaaaaaaaa}   ! b mass\\
wmass 80.398 \phantom{aaaaaaa}   ! w mass \\
sthw2 0.2226459 \phantom{aaaa}   ! (sin(theta\_W))**2 \\
alphaem\_inv  132.3384 \phantom{}! 1/alphaem \\
~\\
charmthr     1.5\\
bottomthr    4.75\\
~\\
CKM\_Vud 0.9740 \phantom{aaaaaaa} ! CKM matrix entries ...\\
CKM\_Vus 0.2225 \\
CKM\_Vub 0.000001 \\
CKM\_Vcd 0.2225 \\
CKM\_Vcs 0.9740 \\
CKM\_Vcb 0.000001 \\
CKM\_Vtd 0.000001 \\
CKM\_Vts 0.000001 \\
CKM\_Vtb 1.0\\
}
%
\\
The value of \tmtexttt{ttype} is mandatory and it is used to decide if
top or antitop quarks will be produced. The meaning of all the other
parameters is self-explanatory. We remind that it is not allowed to
set any entry of the CKM matrix exactly equal to zero.



% \section{Generation of a sample with $t$ and $\bar{t}$
%   events}\label{sec:merging}

% The user can be interested in the generation of a sample where both
% top and antitop events appear. To this purpose, a script and a
% dedicated executable have been included. The script is named
% \tmtexttt{merge\_ttb.sh} and can be found in the directory
% \tmtexttt{testrun}.  It can be run in any subfolder of
% \tmtexttt{ST\_wtch\_DR}. Three inputs are mandatory: the first two are
% the prefixes of the input files used to generate $t$ and $\bar{t}$
% events. The third input has to be an integer and correspond to the
% total number of events that the final \emph{merged} sample will
% contain. The script has to be run twice, using a positive integer
% value at the first call and its opposite afterward.  Therefore, for
% example, to produce a sample of 10000 events at the LHC, starting
% from the input files \tmtexttt{lhc\_wt\_t-powheg.input} and \tmtexttt{lhc\_wt\_tb-powheg.input}, the invocation lines should be as follows:~\\~\\
% \tmtexttt{\$ sh merge\_ttb.sh lhc\_wt\_t lhc\_wt\_tb 10000}\\~\\
% and then~\\~\\
% \tmtexttt{\$ sh merge\_ttb.sh lhc\_wt\_t lhc\_wt\_tb -10000}\\~

% Few remarks are needed:
% \begin{itemize}
% \item it is responsibility of the user to check that the 2 input files
%   are equal. The \tmtexttt{ttype} tokens have to be different,
%   obviously.
% \item the two values of \tmtexttt{numevts} are not really used: the
%   program re-calculate the needed values as a function of the $t$ and
%   $\bar{t}$ cross sections and of the total number of events to be
%   generated.
% \item the final event file is always named
%   \tmtexttt{t\_tb\_sample-events.lhe}. In the header section it also
%   contains a copy of the two input files used to generate it, for
%   cross-checking purposes
% \end{itemize}


\begin{thebibliography}{10}

\bibitem{Nason:2004rx}
  P.~Nason,
  ``A new method for combining NLO QCD with shower Monte Carlo algorithms,''
  JHEP {\bf 0411} (2004) 040
  [arXiv:hep-ph/0409146].
  %%CITATION = JHEPA,0411,040;%%

%\cite{Frixione:2007vw}
\bibitem{Frixione:2007vw}
  S.~Frixione, P.~Nason and C.~Oleari,
``Matching NLO QCD computations with Parton Shower simulations: the POWHEG
method,''
  JHEP {\bf 0711} (2007) 070
  [arXiv:0709.2092 [hep-ph]].
  %%CITATION = JHEPA,0711,070;%%

%\cite{Alioli:2010xd}
\bibitem{Alioli:2010xd}
  S.~Alioli, P.~Nason, C.~Oleari and E.~Re,
``A general framework for implementing NLO calculations in shower Monte Carlo
programs: the POWHEG BOX,''
  JHEP {\bf 1006}, 043 (2010)
  [arXiv:1002.2581 [hep-ph]].
  %%CITATION = JHEPA,1006,043;%%

%\cite{Campbell:2009ss}
\bibitem{Campbell:2009ss} 
  J.~M.~Campbell, R.~Frederix, F.~Maltoni and F.~Tramontano,
``Next-to-Leading-Order Predictions for t-Channel Single-Top Production at Hadron Colliders,''
  Phys.\ Rev.\ Lett.\  {\bf 102}, 182003 (2009)
  [arXiv:0903.0005 [hep-ph]].
  %%CITATION = ARXIV:0903.0005;%%

%\cite{Frederix:2012dh}
\bibitem{Frederix:2012dh} 
  R.~Frederix, E.~Re and P.~Torrielli,
``Single-top t-channel hadroproduction in the four-flavour scheme with POWHEG and aMC@NLO,''
  arXiv:1207.5391 [hep-ph].
  %%CITATION = ARXIV:1207.5391;%%

%\cite{Alioli:2009je}
\bibitem{Alioli:2009je}
  S.~Alioli, P.~Nason, C.~Oleari and E.~Re,
  ``NLO single-top production matched with shower in POWHEG: s- and t-channel
  contributions,''
  JHEP {\bf 0909}, 111 (2009)
  [arXiv:0907.4076 [hep-ph]].
  %%CITATION = JHEPA,0909,111;%%

%\cite{Whalley:2005nh}
\bibitem{Whalley:2005nh}
  M.~R.~Whalley, D.~Bourilkov and R.~C.~Group,
``The Les Houches accord PDFs (LHAPDF) and LHAGLUE,''
  [arXiv:hep-ph/0508110].
  %%CITATION = HEP-PH/0508110;%%

\end{thebibliography}

\end{document}





