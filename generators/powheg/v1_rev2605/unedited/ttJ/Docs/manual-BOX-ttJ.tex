\documentclass[paper]{JHEP3}
\usepackage{amsmath,amssymb,enumerate,url}
\bibliographystyle{JHEP}

%%%%%%%%%% Start TeXmacs macros
\newcommand{\tmtextit}[1]{{\itshape{#1}}}
\newcommand{\tmtexttt}[1]{{\ttfamily{#1}}}
\newenvironment{enumeratenumeric}{\begin{enumerate}[1.] }{\end{enumerate}}
\newcommand\sss{\mathchoice%
{\displaystyle}%
{\scriptstyle}%
{\scriptscriptstyle}%
{\scriptscriptstyle}%
}

\newcommand\as{\alpha_{\sss\rm S}}
\newcommand\POWHEG{{\tt POWHEG}}
\newcommand\POWHEGBOX{{\tt POWHEG BOX}}
\newcommand\HERWIG{{\tt HERWIG}}
\newcommand\PYTHIA{{\tt PYTHIA}}
\newcommand\MCatNLO{{\tt MC@NLO}}

\newcommand\kt{k_{\sss\rm T}}

\newcommand\pt{p_{\sss\rm T}}
\newcommand\LambdaQCD{\Lambda_{\scriptscriptstyle QCD}}
%%%%%%%%%% End TeXmacs macros


\title{The POWHEG BOX user manual:\\
  $\boldsymbol{t \bar{t}+1}$~jet production} \vfill
\author{Simone Alioli\\
  Ernest Orlando Lawrence Berkeley National Laboratories\\
  University of California\\
  1 Cyclotron Road, CA94720 Berkeley, USA\\
  E-mail: \email{salioli@lbl.gov}}


\vskip -0.5truecm

\keywords{POWHEG, Shower Monte Carlo, NLO}

\abstract{This note documents the use of the package
  \POWHEGBOX{} for $t \bar{t}+1$~jet production processes.
  Results can be easily interfaced to shower Monte Carlo programs, in
  such a way that both NLO and shower accuracy are maintained.}
\preprint{\today\\ \tmtexttt{POWHEG BOX} 1.0}

\begin{document}


\section{Introduction}

The \POWHEGBOX{} program is a framework for implementing NLO
calculations in Shower Monte Carlo programs according to the
\POWHEG{} method. An explanation of the method and a discussion of
how the code is organized can be found in
refs.~\cite{Nason:2004rx,Frixione:2007vw,Alioli:2010xd}.  The code is
distributed according to the ``MCNET GUIDELINES for Event Generator
Authors and Users'' and can be found at the web page 
%
\begin{center}
 \url{http://powhegbox.mib.infn.it}.
\end{center}
%
In this manual, we describe the \POWHEG{} NLO implementation of $t \bar{t}+1$~jet
hadroproduction, as described in ref.~\cite{Alioli:2011as}


\section{Generation of events}

The executable needs FASTJET installed. If the lhapdf-config script is
already installed in a common path, then the linking should be
automatic.  Otherwise the user needs to modify the Makefile appropriately.\\

Build the executable\\
\\
\noindent
\tmtexttt{\$ cd POWHEG-BOX/ttJ \\
\$ make pwhg\_main }\\

The make process will first build a library for the virtual
amplitudes, named libvirtual.so and then link it to the main POWHEG
executable.\\

\noindent
Then do (for example) \\
\tmtexttt{\$
cd testrun-lhc\\
\$ ../pwhg\_main}\\

\noindent
At the end of the run, the file \tmtexttt{pwgevents.lhe} will contain 
events for $t \bar{t}+1$~jet hadroproduction in the Les Houches format.

\noindent
In order to shower them with \PYTHIA{} do\\
\tmtexttt{\$
cd POWHEG-BOX/ttJ \\ 
\$ make main-PYTHIA-lhef \\ 
\$ cd testrun-lhc \\
\$ ../main-PYTHIA-lhef}



\section{Process specific input parameters}
\noindent
The $t\bar{t}+1$ parton process is already divergent at the LO, so a
generation cut or a Born suppression factor is needed in order to get
finite results. We refer to the corresponding manual under
\tmtexttt{POWHEG-BOX/Docs/GenerationCut.pdf} for details.


\noindent
The decay of the $t$ quark is controlled by the token
\tmtexttt{topdecaymode}, in this way: \\
\noindent
\tmtexttt{\$ topdecaymode 20000
  ! an integer of 5 digits representing the decay mode.  }
\noindent
The top-quark is assumed to go to a $b$ and a $W$, with the $W$
decaying according to a diagonal CKM matrix.The meaning of the token
is the following: each digit represents the maximum number of the
following particles in the (parton level) decay of the $t\bar t$ pair:
$e^\pm, \mu^\pm, \tau^\pm u^\pm, c^\pm$. Thus, for example, 20000
means the $t \to e^+ \nu_e b, \bar t e^− \bar \nu _e \bar b$, 22222
means all decays, 10011 means one goes into eletron or antielectron,
and the other goes into any hadron, 00022 means fully hadronic, 00011
means fully hadronic with a single charm, 00012 fully hadronic with at
least one charm.  The value 0 means that the t and t¯ are not
decayed. Values that imply only one $t$ decay (for example 10000) are
not implemented consistently.  If the flag \tmtexttt{semileptonic} is set to 1,
only semileptonic decays are kept by the program.  In case
\tmtexttt{topdecaymode} is different from 0 more parameters are needed for the
decay kinematics, and are used exclusively for decays.\\
\noindent
The top-quark mass value can be set by\\
\tmtexttt{\$
topmass 173.2 ! top mass value in GeV.
}

\noindent
Also, in case an analysis routine is linked, the process requires some parameters to specify the jet \\
\tmtexttt{\$
R\_jet 0.5 ! jet radius.\\ \$
ptmin\_jet 25 ! jet min pt in GeV.
} 

\noindent
Since the calculation is quite intensive from a computational point of view, the calculation and the event generation can be parallelized on a cluster. We refer to the\\ \tmtexttt{POWHEG-BOX/Docs/ManySeeds.pdf} manual for details.


\begin{thebibliography}{10}

\bibitem{Nason:2004rx}
  P.~Nason,
  ``A new method for combining NLO QCD with shower Monte Carlo algorithms,''
  JHEP {\bf 0411} (2004) 040
  [arXiv:hep-ph/0409146].
  %%CITATION = JHEPA,0411,040;%%

%\cite{Frixione:2007vw}
\bibitem{Frixione:2007vw}
  S.~Frixione, P.~Nason and C.~Oleari,
``Matching NLO QCD computations with Parton Shower simulations: the POWHEG
method,''
  JHEP {\bf 0711} (2007) 070
  [arXiv:0709.2092 [hep-ph]].
  %%CITATION = JHEPA,0711,070;%%

%\cite{Alioli:2010xd}
\bibitem{Alioli:2010xd}
  S.~Alioli, P.~Nason, C.~Oleari and E.~Re,
``A general framework for implementing NLO calculations in shower Monte Carlo
  programs: the POWHEG BOX,''
  [arXiv:1002.2581 [hep-ph]].
  %%CITATION = ARXIV:1002.2581;%%

%\cite{Alioli:2008gx}
\bibitem{Alioli:2010qp} S.~Alioli, P.~Nason, C.~Oleari and E.~Re,
  ``Vector boson plus one jet production in POWHEG,''
  JHEP {\bf 1101}, 005 (2011)
  [arXiv:1009.5594 [hep-ph]].

%\cite{Alioli:2011as}
\bibitem{Alioli:2011as}
  S.~Alioli, S.~-O.~Moch and P.~Uwer,
  ``Hadronic top-quark pair-production with one jet and parton showering,''
  JHEP {\bf 1201} (2012) 137
  [arXiv:1110.5251 [hep-ph]].
  %%CITATION = ARXIV:1110.5251;%%
  %25 citations counted in INSPIRE as of 09 Sep 2013
\end{thebibliography}

\end{document}





