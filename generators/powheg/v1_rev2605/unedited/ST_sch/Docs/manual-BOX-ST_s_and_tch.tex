\documentclass[paper]{JHEP3}
\usepackage{amssymb,enumerate,url}
\bibliographystyle{JHEP}

%%%%%%%%%% Start TeXmacs macros
\newcommand{\tmtextit}[1]{{\itshape{#1}}}
\newcommand{\tmtexttt}[1]{{\ttfamily{#1}}}
\newenvironment{enumeratenumeric}{\begin{enumerate}[1.] }{\end{enumerate}}
\newcommand\sss{\mathchoice%
{\displaystyle}%
{\scriptstyle}%
{\scriptscriptstyle}%
{\scriptscriptstyle}%
}

\newcommand\as{\alpha_{\sss\rm S}}
\newcommand\POWHEG{{\tt POWHEG}}
\newcommand\HERWIG{{\tt HERWIG}}
\newcommand\PYTHIA{{\tt PYTHIA}}
\newcommand\MCatNLO{{\tt MC@NLO}}

\newcommand\kt{k_{\sss\rm T}}

\newcommand\pt{p_{\sss\rm T}}
\newcommand\LambdaQCD{\Lambda_{\scriptscriptstyle QCD}}
%%%%%%%%%% End TeXmacs macros


\title{The \mbox{POWHEG BOX} user manual:\\
  Single-top s- and t-channel processes} \vfill
\author{Simone Alioli\\
  Deutsches Elektronen-Synchrotron DESY\\
  Platanenallee 6, D-15738 Zeuthen, Germany\\
  E-mail: \email{simone.alioli@desy.de}}
\author{Paolo Nason\\
  INFN, Sezione di Milano-Bicocca,
  Piazza della Scienza 3, 20126 Milan, Italy\\
  E-mail: \email{Paolo.Nason@mib.infn.it}}
\author{Carlo Oleari\\
  Universit\`a di Milano-Bicocca and INFN, Sezione di Milano-Bicocca\\
  Piazza della Scienza 3, 20126 Milan, Italy\\
  E-mail: \email{Carlo.Oleari@mib.infn.it}}
\author{Emanuele Re\\
  Institute for Particle Physics Phenomenology, Department of Physics\\
  University of Durham, Durham, DH1 3LE, UK\\
  E-mail: \email{emanuele.re@durham.ac.uk}}

\vskip -0.5truecm

\keywords{POWHEG, Shower Monte Carlo, NLO}

\abstract{This note documents the use of the package \tmtexttt{POWHEG BOX}
  for the single-top $s$- and $t$-channel production processes.
  Results can be easily interfaced to shower Monte Carlo programs, in
  such a way that both NLO and shower accuracy are maintained.}
\preprint{\today\\ \tmtexttt{POWHEG BOX} 1.0}

\begin{document}


\section{Introduction}

The \tmtexttt{POWHEG BOX} program is a framework for implementing NLO
calculations in Shower Monte Carlo programs according to the
\POWHEG{} method. An explanation of the method and a discussion of
how the code is organized can be found in
refs.~\cite{Nason:2004rx,Frixione:2007vw,Alioli:2010xd}.  The code is
distributed according to the ``MCNET GUIDELINES for Event Generator
Authors and Users'' and can be found at the web page \\
%
\begin{center}
 \url{http://powhegbox.mib.infn.it}.
\end{center}
%
~\\
%
In the following we will focus on the implementation of single-top
($s$- and $t$-channel) production, whose source files can be found in
the \tmtexttt{POWHEG-BOX/ST\_sch} and \tmtexttt{POWHEG-BOX/ST\_tch}
subdirectories.

This program is an implementation of the NLO cross section calculated
in~\cite{Harris:2002md} in the \POWHEG{} formalism of
refs.~\cite{Nason:2004rx,Frixione:2007vw}.  A detailed description of
the implementation can be found in ref.~\cite{Alioli:2009je}. Spin
correlations of the top-quark decay products are included with a
method analogous to the one described in~\cite{Frixione:2007zp}, and
the relevant matrix elements for the full decayed amplitudes were
obtained using MadGraph~\cite{Alwall:2007st}.\footnote{The only
  difference of the BOX implementation with respect to the one
  described in~\cite{Alioli:2009je} is the treatment of finite width
  effects. In the BOX program, we decided to calculate the production
  cross section keeping always the value of the top-quark offshellness
  fixed and equal to the mass value. Finite-width effects are included
  a posteriori, at the same stage of top decay-products generation.}

In this note we give all the necessary information to run the program.

\section{Installation}
In order to run the \tmtexttt{POWHEG BOX} program, we recommend the
reader to start from the \tmtexttt{POWHEG BOX} user manual, which
contains all the information and settings that are common between all
subprocesses. In this note we focus on the settings and parameters
specific to the single-top $s$- and $t$-channel implementations.

In the following, we will describe how to run the $s$-channel
code. Same considerations hold also for the $t$-channel one: in fact,
the structure of the two codes (and the relevant files) is similar.
Explicit instructions will be given when there are relevant
differences.

\section{Generation of events and showering}
The executable is built with the following commands\\~\\
\tmtexttt{
%
  \$ cd POWHEG-BOX/ST\_sch\\
% 
  \$ make pwhg\_main\\
%
}
\\
In the \tmtexttt{testrun} folder, there are several examples of input files.
For example, you can start a run doing\\~\\
\tmtexttt{
%
  \$ cd testrun\\
% 
  \$ ../pwhg\_main\\
%
}
\\
The input file read in this case is \tmtexttt{powheg.input} and
at the end a file named \tmtexttt{pwgevents.lhe} will contain 100000
events for single top $s$-channel production at the LHC, in the Les Houches format.
To shower them with \PYTHIA{} do\\~\\
\tmtexttt{
%
\$ cd POWHEG-BOX/ST\_sch\\
%
\$ make main-PYTHIA-lhef \\
%
\$ cd testrun \\
%
\$ ../main-PYTHIA-lhef\\
%
}
\\
Similar commands will run the \HERWIG{} shower.


\section{Process specific input parameters}
\tmtexttt{
  facscfact 1 ! factorization scale factor: mufact=muref*facscfact\\
  renscfact 1 ! renormalization scale factor: muren=muref*renscfact\\
}
\\
Factorization and renormalization scale factors appearing here have to
do with the computation of the inclusive cross section (i.e.~the
$\bar{B}$ function~\cite{Nason:2004rx,Frixione:2007vw,Alioli:2009je}),
and can be varied by a factor of order 1 to study scale dependence.
The natural choice for this process is the mass of the top-quark. We
choose to perform the NLO calculation keeping these scales fixed.  The
experienced user can change this setting modifying the
\tmtexttt{set\_fac\_ren\_scales} routine.
\\~\\
It follows a description of parameters which are relevant for
this production process:
\begin{itemize}
\item If the fraction of negative weights is large, one may increase
  \tmtexttt{foldcsi}, \tmtexttt{foldy}, \tmtexttt{foldphi}.  Allowed
  values are 1, 2, 5, 10, 25, 50. The speed of the program is
  inversely proportional to the product of these numbers, so that a
  reasonable compromise should be found.  Our experiences tell us
  that, even at LHC energies, the fraction of negative weights in
  $\bar B$ calculation is such that the numbers provided in the
  examples need not to be changed.  For the $t$- channel case, it is
  recommended to leave the default foldings on the csi and y
  variables.\footnote{In all examples, the choice of the parameters
    that control the grid generation is such that a reasonably small
    fraction of negative weights is generated, so they can be run as
    they are. We remind the reader that these negative weights are
    only due to our choice of generating $\tilde{B}$ instead of $\bar
    B$.  They indeed correspond to phase space points where NLO
    corrections are bigger than LO contributions.  Had we performed
    the integration over the full radiation phase space these negative
    weights would have disappeared completely.}
\item For single-top, it is recommended to activate the
  \tmtexttt{withdamp} option, to enable the Born-zero damping factor.
\item For the $t$- channel case, it is recommended to leave the
  default value for \tmtexttt{iymax}, to obtain a good efficiency in
  the radiation generation.
\end{itemize}


Other parameters are those specifically related to single-top
processes: from revision 1.0, some of these parameters are mandatory
(the program stops if they are missing), other are optional (default
values are assigned in \tmtexttt{init\_couplings.f}, but are
overwritten if the token is found uncommented in the input file, as in
previous versions).


For the production step, the relevant
parameters are:\\
~\\
\tmtexttt{
! mandatory production parameters\\
ttype   1 \phantom{aaaaaaa} ! 1 for t, -1 for tbar\\
topmass 175.0 \phantom{a}   ! top mass\\
}
%
~\\
where the value of \tmtexttt{ttype} is used to decide if top or
antitop quarks will be produced and \tmtexttt{topmass} set the
top-quark mass.

In the current released version, top-quark decay products are always
generated by \POWHEG{}, accordingly to a procedure very similar to the
one of ref.~\cite{Frixione:2007zp}. Therefore, the following
parameters are mandatory too:\\
~\\
\tmtexttt{
! mandatory parameters used in decay generation\\
topdecaymode 10000           ! decay mode: the 5 digits correspond to the following\\
\phantom{topdecaymode 10000} ! top-decay channels (l,mu,tau,u,c) \\
\phantom{topdecaymode 10000} ! 0 means close, 1 open\\
tdec/elbranching 0.108  ! W electronic branching fraction\\
}
%
~\\
where the value of the \tmtexttt{topdecaymode} token is formed by five
digits, each representing the maximum number of the following
particles at the (parton level) decay of the $t$ ($\bar{t}$) quark:
$e^{\pm}$, $\mu^{\pm}$, $\tau^{\pm}$,
$\stackrel{\scriptscriptstyle{(-)}}{u}$,
$\stackrel{\scriptscriptstyle{(-)}}{c}$.  Thus, for example, 10000
means $t \rightarrow e^+ \nu_e b$, 11100 means all semileptonic
decays, 00011 means fully hadronic.

When all the 5 digits of \tmtexttt{topdecaymode} are set to 0, the
program will consider the top-quark as a stable particle: its decay
products will not be generated, and the event file will contain the
top-quark kinematics. Since there is no information on the top-quark
spin, obviously spin-correlation effects are lost when these events
are showered.

The optional parameters are listed below. Their meaning is
self-explanatory. We remind that it is not allowed to set any entry of
the CKM matrix exactly equal to zero.\\
~\\
\tmtexttt{
! optional production parameters \\
! (defaults defined in init\_couplings.f) \\
\#wmass 80.4                 \phantom{aaaaaaaaaaaaa} ! w mass \\
\#sthw2 0.23113              \phantom{aaaaaaaaaa} ! (sin(theta\_W))**2 \\
\#alphaem\_inv  127.011989   \phantom{a} ! 1/alphaem \\
\#CKM\_Vud 0.9740            \phantom{aaaaaaaaa} ! CKM matrix entries ...\\
\#CKM\_Vus 0.2225 \\
\#CKM\_Vub 0.000001 \\
\#CKM\_Vcd 0.2225 \\
\#CKM\_Vcs 0.9740 \\
\#CKM\_Vcb 0.000001 \\
\#CKM\_Vtd 0.000001 \\
\#CKM\_Vts 0.000001 \\
\#CKM\_Vtb 1.0\\
~\\
! optional parameters used in decay generation\\
! (defaults defined in init\_couplings.f)\\
\#topwidth         1.7      \phantom{aaaaaaa} ! top width \\
\#wwidth           2.141    \phantom{aaaaaaa} ! w width \\
\#tdec/emass       0.000511 \phantom{} ! e mass \\
\#tdec/mumass      0.1056   \phantom{a} ! mu mass \\
\#tdec/taumass     1.777    \phantom{a} ! tau mass
}

\section{Generation of a sample with $t$ and $\bar{t}$
  events}\label{sec:merging}

The user can be interested in the generation of a sample where both
top and antitop events appear. To this purpose, a script and a
dedicated executable have been included. The script is named
\tmtexttt{merge\_ttb.sh} and can be found in the directory
\tmtexttt{testrun}.  It can be run in any subfolder of
\tmtexttt{ST\_sch}. Three inputs are mandatory: the first two are the
prefixes of the input files used to generate $t$ and $\bar{t}$ events. The
third input has to be an integer and correspond to the total number of
events that the final \emph{merged} sample will contain. The script has to
be run twice, using a positive integer value at the first call and
its opposite afterward.
 Therefore,
for example, to produce a sample of 10000 events at Tevatron, starting
from the
input files \tmtexttt{tev\_st\_s\_t-powheg.input} and \tmtexttt{tev\_st\_s\_tb-powheg.input}, the invocation lines should be as follows:~\\~\\
\tmtexttt{\$ sh merge\_ttb.sh tev\_st\_s\_t tev\_st\_s\_tb 10000}\\~\\
and then~\\~\\
\tmtexttt{\$ sh merge\_ttb.sh tev\_st\_s\_t tev\_st\_s\_tb -10000}\\~

Few remarks are needed:
\begin{itemize}
\item it is responsibility of the user to check that the 2 input files
  are equal. The \tmtexttt{ttype} tokens have to be different,
  obviously.
\item the two values of \tmtexttt{numevts} are not really used: the
  program re-calculate the needed values as a function of the $t$ and
  $\bar{t}$ cross sections and of the total number of events to be
  generated.
\item the final event file is always named
  \tmtexttt{t\_tb\_sample-events.lhe}. In the header section it also
  contains a copy of the two input files used to generate it, for
  cross-checking purposes
\end{itemize}


\begin{thebibliography}{10}

\bibitem{Nason:2004rx}
  P.~Nason,
  ``A new method for combining NLO QCD with shower Monte Carlo algorithms,''
  JHEP {\bf 0411} (2004) 040
  [arXiv:hep-ph/0409146].
  %%CITATION = JHEPA,0411,040;%%

%\cite{Frixione:2007vw}
\bibitem{Frixione:2007vw}
  S.~Frixione, P.~Nason and C.~Oleari,
``Matching NLO QCD computations with Parton Shower simulations: the POWHEG
method,''
  JHEP {\bf 0711} (2007) 070
  [arXiv:0709.2092 [hep-ph]].
  %%CITATION = JHEPA,0711,070;%%

%\cite{Alioli:2010xd}
\bibitem{Alioli:2010xd}
  S.~Alioli, P.~Nason, C.~Oleari and E.~Re,
``A general framework for implementing NLO calculations in shower Monte Carlo
  programs: the POWHEG BOX,''
  [arXiv:1002.2581 [hep-ph]].
  %%CITATION = ARXIV:1002.2581;%%

%\cite{Harris:2002md}
\bibitem{Harris:2002md}
  B.~W.~Harris, E.~Laenen, L.~Phaf, Z.~Sullivan and S.~Weinzierl,
  ``The Fully differential single top quark cross-section in next to leading
  order QCD,''
  Phys.\ Rev.\  D {\bf 66}, 054024 (2002)
  [arXiv:hep-ph/0207055].
  %%CITATION = PHRVA,D66,054024;%%

% %\cite{Frixione:2005vw}
% \bibitem{Frixione:2005vw}
%   S.~Frixione, E.~Laenen, P.~Motylinski and B.~R.~Webber,
%   %``Single-top production in MC@NLO,''
%   JHEP {\bf 0603}, 092 (2006)
%   [arXiv:hep-ph/0512250].
%   %%CITATION = JHEPA,0603,092;%%

%\cite{Alioli:2009je}
\bibitem{Alioli:2009je}
  S.~Alioli, P.~Nason, C.~Oleari and E.~Re,
  ``NLO single-top production matched with shower in POWHEG: s- and t-channel
  contributions,''
  JHEP {\bf 0909}, 111 (2009)
  [arXiv:0907.4076 [hep-ph]].
  %%CITATION = JHEPA,0909,111;%%

%\cite{Frixione:2007zp}
\bibitem{Frixione:2007zp}
  S.~Frixione, E.~Laenen, P.~Motylinski and B.~R.~Webber,
  ``Angular correlations of lepton pairs from vector boson and top quark decays
  in Monte Carlo simulations,''
  JHEP {\bf 0704}, 081 (2007)
  [arXiv:hep-ph/0702198].
  %%CITATION = JHEPA,0704,081;%%

%\cite{Alwall:2007st}
\bibitem{Alwall:2007st}
  J.~Alwall {\it et al.},
  ``MadGraph/MadEvent v4: The New Web Generation,''
  JHEP {\bf 0709}, 028 (2007)
  [arXiv:0706.2334 [hep-ph]].
  %%CITATION = JHEPA,0709,028;%%

% %\cite{Cacciari:2005hq}
% \bibitem{Cacciari:2005hq}
%   M.~Cacciari and G.~P.~Salam,
%   ``Dispelling the $N^{3}$ myth for the $k_t$ jet-finder,''
%   Phys.\ Lett.\  B {\bf 641}, 57 (2006)
%   [arXiv:hep-ph/0512210].
%   %%CITATION = PHLTA,B641,57;%%

% %\cite{Boos:2001cv}
% \bibitem{Boos:2001cv}
%   E.~Boos {\it et al.},
%   ``Generic user process interface for event generators,''
%   [arXiv:hep-ph/0109068].
%   %%CITATION = HEP-PH/0109068;%%

% %\cite{Alwall:2006yp}
% \bibitem{Alwall:2006yp}
%   J.~Alwall {\it et al.},
%   ``A standard format for Les Houches event files,''
%   Comput.\ Phys.\ Commun.\  {\bf 176} (2007) 300
%   [arXiv:hep-ph/0609017].
%   %%CITATION = CPHCB,176,300;%%

% %\cite{Altarelli:1989wu}
% \bibitem{Altarelli:1989wu} T. Sj\"ostrand et~al., in
%   ``Z physics at LEP1: Event generators and software,'',  eds.
%   G.~Altarelli, R.~Kleiss and C.~Verzegnassi, Vol 3, pg. 327.
%   %%CITATION = CERN-89-08-V-3;%%

% %\cite{Whalley:2005nh}
% \bibitem{Whalley:2005nh}
%   M.~R.~Whalley, D.~Bourilkov and R.~C.~Group,
%   ``The Les Houches accord PDFs (LHAPDF) and LHAGLUE,''
%   [arXiv:hep-ph/0508110].
%   %%CITATION = HEP-PH/0508110;%%

% % %\cite{Yao:2006px}
% % \bibitem{Yao:2006px}
% %   W.~M.~Yao {\it et al.}  [Particle Data Group],
% %   %``Review of particle physics,''
% %   J.\ Phys.\ G {\bf 33}, 1 (2006).
% %   %%CITATION = JPHGB,G33,1;%%

\end{thebibliography}

\end{document}





