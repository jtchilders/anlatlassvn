\documentclass[a4paper,11pt]{article}
\usepackage{amssymb,enumerate}
\usepackage{amsmath}
\usepackage{bbm}
\usepackage{url}
\usepackage{cite}
\usepackage{graphics}
\usepackage{xspace}
\usepackage{epsfig}
\usepackage{subfigure}

\setlength\paperwidth  {210mm}%
\setlength\paperheight {300mm}%	

\textwidth 160mm%		% DEFAULT FOR LATEX209 IS a4
\textheight 230mm%

\voffset -1in
\topmargin   .05\paperheight	% FROM TOP OF PAGE TO TOP OF HEADING (0=1inch)
\headheight  .02\paperheight	% HEIGHT OF HEADING BOX.
\headsep     .03\paperheight	% VERT. SPACE BETWEEN HEAD AND TEXT.
\footskip    .07\paperheight	% FROM END OF TEX TO BASE OF FOOTER. (40pt)


\hoffset -1in				% TO ADJUST WITH PAPER:
	\oddsidemargin .13\paperwidth	% LEFT MARGIN FOR ODD PAGES (10)
	\evensidemargin .15\paperwidth	% LEFT MARGIN FOR EVEN PAGES (30)
	\marginparwidth .10\paperwidth	% TEXTWIDTH OF MARGINALNOTES
	\reversemarginpar		% BECAUSE OF TITLEPAGE.

\newcommand{\tmtextit}[1]{{\itshape{#1}}}
\newcommand{\tmtexttt}[1]{{\ttfamily{#1}}}
\newenvironment{enumeratenumeric}{\begin{enumerate}[ 1.] }{\end{enumerate}}
\newcommand\sss{\mathchoice%
{\displaystyle}%
{\scriptstyle}%
{\scriptscriptstyle}%
{\scriptscriptstyle}%
}
\newcommand\PSn{\Phi_{n}}
\newcommand{\tmop}[1]{\ensuremath{\operatorname{#1}}}

\newcommand\Lum{{\cal L}}
\newcommand\matR{{\cal R}}
\newcommand\Kinnpo{{\bf \Phi}_{n+1}}
\newcommand\Kinn{{\bf \Phi}_n}
\newcommand\PSnpo{\Phi_{n+1}}
\newcommand\as{\alpha_{\sss\rm S}}
\newcommand\asotpi{\frac{\as}{2\pi}}

\newcommand\POWHEG{{\tt POWHEG}}
\newcommand\POWHEGBOX{{\tt POWHEG\;BOX}}
\newcommand\PYTHIA{{\tt PYTHIA}}
\newcommand\POWHEGpPYTHIA{{\tt POWHEG+PYTHIA}}
\newcommand\HERWIG{{\tt HERWIG}}


\title{Manual for slepton pair production in the \POWHEGBOX{}}
\date{}
%\author{}
%\keywords{}
%\abstract{}
%\preprint{}


\begin{document}
\maketitle
%
\noindent
This document describes the settings and input parameters that are specific to
the implementation of slepton pair production at the LHC within the
\POWHEGBOX{} framework. 
%
The parameters that are common to all \POWHEGBOX{} implementations are given in
the {\tt manual-BOX.pdf} document in the {\tt POWHEG-BOX/Docs}
directory.
%\\[2ex]
If you use our program, please quote
Refs.~\cite{JMT,Alioli:2010xd,Ellis:2007qk,vanOldenborgh:1990yc,Hahn:2006nq,Sjostrand:2006za}.

\section*{Running the program}
%
Download the \POWHEGBOX{}, following the instructions at the web site 
\\[2ex]
{\tt http://powhegbox.mib.infn.it/}
\\[2ex] 
Before compiling the code make sure that:
\begin{itemize}
\item 
{\tt fastjet} is installed and {\tt fastjet-config} is in the path,
\item 
{\tt lhapdf} is installed and {\tt lhapdf-config} is in the path,
\item
{\tt gfortran} or {\tt ifort} is in the path, and the
appropriate libraries are in the environment variable {\tt
  LD\_LIBRARY\_PATH}. 
\end{itemize}
%
If {\tt LHAPDF} or {\tt fastjet} are not installed, the code can still
be run using a dummy analysis routine and built-in PDFs.
For this and other build options, see the {\tt Makefile} in the project
directory {\tt sleptons}.
%
\\[2ex]
For a convenient illustration of compiling and running the code,
we are providing a {\tt Makefile} in the sub-directory {\tt testrun-lhc}.
Go to this directory by typing
\\[2ex]
{\tt \$ cd POWHEG-BOX/sleptons/testrun-lhc}
\\[2ex]
Compile and run the code in its default mode by executing
\\[2ex]
{\tt \$ make}
%
%%%%%%%%%%%%%%%%%%%
%
\\[2ex]
Physics parameters are read from the file {\tt params.slha} which is formated 
according to the SUSY Les Houches Accord ({\tt SLHA})~\cite{Skands:2003cj,Allanach:2008qq} and can be taken from standard SUSY spectrum generators.
Technical parameters are specified in the  {\tt powheg.input} file. 
\\[2ex] 
In the last step of the run, the generated events are showered via {\tt PYTHIA}. Various settings of {\tt PYTHIA}
can be modified by editing the file {\tt setup-PYTHIA-lhef.f} prior to compilation.
%\\[2ex]
%Both initial state and final state showers are enabled by default,
%they may be switched off by setting {\tt mstp(61)=0} in the former
%and {\tt mstp(71)=0} in the latter case.
%\\[2ex]
We modified {\tt PYTHIA} in such a way that by setting the flag {\tt mstp(41)=1}, slepton decays are activated.  
{\tt PYTHIA} generates the decays according to the masses provided in {\tt params.slha}.
For the case of each slepton decaying into a lepton and the lightest neutralino, our sample analysis provides various histograms for the decay products. 
%
%%%%%%%%%%%%%%%%%%%%%%%%%%
%
\begin{thebibliography}{99}

\bibitem{JMT} B.~J\"ager, A.~von~Manteuffel, and S.~Thier, {\em
  Slepton pair production in the POWHEG BOX}, arXiv:1208.2953. 
  
\bibitem{Alioli:2010xd} S.~Alioli, P.~Nason, C.~Oleari, and E. Re, {\em
    A general framework for implementing NLO calculations in shower
    Monte Carlo programs: the POWHEG BOX}, JHEP {\bf 1006} (2010)
  043  [arXiv:1002.2581 [hep-ph]].

\bibitem{Ellis:2007qk}
  R.~K.~Ellis and G.~Zanderighi,
  {\em Scalar one-loop integrals for QCD}, 
  JHEP {\bf 0802} (2008) 002
  [arXiv:0712.1851 [hep-ph]].

\bibitem{vanOldenborgh:1990yc}
  G.~J.~van Oldenborgh,
 {\em FF: A Package to evaluate one loop Feynman diagrams}, 
  Comput.\ Phys.\ Commun.\  {\bf 66} (1991) 1.

%\cite{Hahn:2006nq}
\bibitem{Hahn:2006nq}
  T.~Hahn,
  %{\tt SLHALib\,2}
  {\em SUSY Les Houches Accord 2 I/O made easy},  
  Comput.\ Phys.\ Commun.\  {\bf 180} (2009) 1681
  [hep-ph/0605049].
  %,\url{http://www.feynarts.de/slha/}.

\bibitem{Sjostrand:2006za}
  T.~Sjostrand, S.~Mrenna, P.~Z.~Skands,
  {\em PYTHIA 6.4 Physics and Manual},
  JHEP {\bf 0605 } (2006)  026.
  [hep-ph/0603175]. 

\bibitem{Skands:2003cj}
  P.~Z.~Skands  {\it et al.},
  {\em SUSY Les Houches accord: Interfacing SUSY spectrum calculators, decay packages, and event generators},  
  % ``SUSY Les Houches accord 1,'' 
  JHEP {\bf 0407} (2004) 036
  [hep-ph/0311123].

\bibitem{Allanach:2008qq}
  B.~C.~Allanach  {\it et al.},
  {\em SUSY Les Houches Accord 2}, 
  Comput.\ Phys.\ Commun.\  {\bf 180} (2009)~8
  [arXiv:0801.0045 [hep-ph]].

\end{thebibliography}
%%%%%%%%%%%%%%%%%%
\end{document}
